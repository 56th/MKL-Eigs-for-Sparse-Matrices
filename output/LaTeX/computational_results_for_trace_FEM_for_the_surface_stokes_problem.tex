\documentclass[12pt]{article}

\usepackage{mathtools}
\usepackage{amssymb}
\usepackage{amsthm}

\usepackage[dvipsnames, table]{xcolor}
\colorlet{DarkRed}{Red!90!black}
\colorlet{LightRed}{Red!10!white}
\colorlet{DarkGreen}{Green!50!black}
\colorlet{LightGreen}{Green!10!white}
\usepackage{colortbl} % https://texblog.org/2011/04/19/highlight-table-rowscolumns-with-color/
% links
\usepackage{hyperref}
\hypersetup{
	colorlinks,
	linkcolor={DarkRed},
	citecolor={DarkRed},
	urlcolor={blue}
}

\usepackage{geometry}
\newgeometry{
	left=.8cm, right=.8cm, top=.8cm, bottom=.8cm,
	includefoot, heightrounded
}

\usepackage[parfill]{parskip} % https://tex.stackexchange.com/a/16703/135296

% sub figures / grids of pictures
\usepackage{subcaption}
\usepackage{graphicx}
\graphicspath{{img/}} % includegraphics path
% \usepackage[export]{adjustbox} % https://tex.stackexchange.com/questions/20640/how-to-add-border-for-an-image
\newcommand{\includegraphicsw}[2][1.]{\includegraphics[width=#1\linewidth]{#2}}
\newcommand{\svginput}[1]{\input{img/#1}} % pdf_tex path
\newcommand{\svginputw}[2][\linewidth]{\def\svgwidth{#1}\input{img/#2}} % pdf_tex path

% tables
\usepackage{multirow}
\usepackage{hhline}
\usepackage{float} % for H

% bold for everything
\usepackage{bm}
\newcommand{\vect}[1]{\boldsymbol{\mathbf{#1}}}

% differentials
\newcommand*\diff{\mathop{}\!\mathrm{d}}
\newcommand*\Diff[1]{\mathop{}\!\mathrm{d^#1}}

\DeclareMathOperator{\Div}{div}
\DeclareMathOperator{\Dist}{dist}
\newcommand{\sphere}{{\Gamma_{\text{sph}}}}
\newcommand{\tor}{{\Gamma_{\text{tor}}}}

\newcommand{\HOne}{{\mathbb H^1}}
\newcommand{\LTwo}{{\mathbb L^2}}
\newcommand{\LTwoSpace}[1][\Gamma]{{\mathbb L^2\left({#1}\right)}}
\newcommand{\HOneSpace}[1][\Gamma]{{\mathbb H^1\left({#1}\right)}}

\usepackage{listings}
\definecolor{mygreen}{rgb}{0,0.6,0}
\lstset{
	language=C++,
	basicstyle=\footnotesize\ttfamily,
	breaklines=true,
	commentstyle=\color{mygreen},
	frame=l,
	xleftmargin=5pt,
	tabsize=2,
	belowskip=-1pt
} 

\title{Some computational results for $\vect{P}_1$\,--\,$P_1$ and $\vect{P}_2$\,--\,$P_1$ Trace\,FEM for the surface Stokes problem}
\author{
	Alexander Zhiliakov\thanks{Department of Mathematics, University of Houston, Houston, Texas 77204 (alex@math.uh.edu).}
}

\begin{document}
	
\maketitle

\tableofcontents
\vfill
\clearpage
\let\oldtabular\tabular
\renewcommand{\tabular}[1][1.5]{\def\arraystretch{#1}\oldtabular}

\section{Preliminaries}

\subsection{Bilinear forms and matrices}

We set $n_{\vect A}$ to be the number of velocity d.o.f. and $n_{\vect S}$ to be the number of pressure d.o.f. Vector stiffness, divergence, pressure mass, normal stabilization, and full stabilization matrices resulting from Trace\,FEM discretization of the surface Stokes problem~\cite{surfstokes} are defined via
\begin{align}\begin{split}\label{mtx}
	\langle \vect A\,\vec{\vect u}, \vec{\vect v} \rangle &\approx 
		\int_{\Gamma} \big( 2\,E_s(\vect u) : E_s(\vect v) + \vect u\cdot\vect v + \tau\,(\vect u\cdot\vect n)\,(\vect v\cdot\vect n) \big) \diff{s} + 
		\rho_u \int_{\Omega_h^{\Gamma}} \frac{\partial \vect u}{\partial\vect n}\cdot\frac{\partial \vect v}{\partial\vect n} \diff{\vect x}, \quad \vect A \in \mathbb R^{n_{\vect A} \times n_{\vect A}},\\
	\langle \vect B\,\vec{\vect u}, \vec{\vect q} \rangle &\approx 
		\int_{\Gamma} \nabla_\Gamma q \cdot \vect u \diff{s}, \quad \vect B \in \mathbb R^{n_{\vect S} \times n_{\vect A}},\\
	\langle \vect M_0\,\vec{\vect p}, \vec{\vect q} \rangle &\approx
		\int_{\Gamma} p\,q \diff{s}, \quad \vect M_0 \in \mathbb R^{n_{\vect S} \times n_{\vect S}},\\
	\langle \vect C_n\,\vec{\vect p}, \vec{\vect q} \rangle &\approx
		\rho_p \int_{\Omega^{\Gamma}_h} \frac{\partial p}{\partial\vect n} \frac{\partial q}{\partial\vect n} \diff{\vect x}, \quad \vect C_n \in \mathbb R^{n_{\vect S} \times n_{\vect S}},\\
	\langle \vect C_{\text{full}}\,\vec{\vect p}, \vec{\vect q} \rangle &\approx
		\rho_p \int_{\Omega^{\Gamma}_h} \nabla p \cdot \nabla q \diff{\vect x}, \quad \vect C_{\text{full}} \in \mathbb R^{n_{\vect S} \times n_{\vect S}},		 
\end{split}\end{align}
respectively. We use notations as in~\cite{surfstokes}, in particular, $\Omega_\Gamma^h$ is the domain consisting of tetrahedra cut by the surface~$\Gamma \coloneqq \{ \vect x \in \mathbb{R}^3\::\:\phi(\vect x) = 0\}$. Here~$\vec{\vect u}$ denotes a vector of d.o.f. corresponding to a FE interpolant~$\vect u$ (analogously for $\vec{\vect p}$ and $p$). See~\eqref{mtx_exact} and~\eqref{mtx_exact_2} for the computational details. Mesh-dependent parameters are set as $\tau$, $\rho_u$, and $\rho_p$ are chosen to be proportional to some power of~$h \coloneqq$~the typical mesh size for tetrahedra from~$\Omega^{\Gamma}_h$. $\Gamma$ is chosen either as the unit sphere or torus, $\Gamma = \sphere$ or $\Gamma = \tor$ (see Figure~\ref{fig:gamma}). The background domain is chosen as~$\Omega \coloneqq (-5/3,\,5/3)^3$.

\begin{figure}[H]
	\centering
	\begin{subfigure}{.25\linewidth}
		\centering
		\includegraphicsw[.9]{{lvl1.cropped}.png}
		\caption{$h = 8.33\times10^{-1}$}
	\end{subfigure}%
	\begin{subfigure}{.25\linewidth}
		\centering
		\includegraphicsw[.9]{{lvl2.cropped}.png}
		\caption{$h = 4.17\times10^{-1}$}
	\end{subfigure}%
	\begin{subfigure}{.25\linewidth}
		\centering
		\includegraphicsw[.9]{{lvl3.cropped}.png}
		\caption{$h = 2.08\times10^{-1}$}
	\end{subfigure}
	\par\bigskip
	\begin{subfigure}{.25\linewidth}
		\centering
		\includegraphicsw[.9]{{tor_lvl3.cropped}.png}
		\caption{$h = 2.08\times10^{-1}$}
	\end{subfigure}%
	\begin{subfigure}{.25\linewidth}
		\centering
		\includegraphicsw[.9]{{tor_lvl4.cropped}.png}
		\caption{$h = 1.04\times10^{-1}$}
	\end{subfigure}%
	\begin{subfigure}{.25\linewidth}
		\centering
		\includegraphicsw[.9]{{tor_lvl5.cropped}.png}
		\caption{$h = 5.21\times10^{-2}$}
	\end{subfigure}
	\caption{First three mesh levels for~$\sphere$ (top) and $\tor$ (bottom)}
	\label{fig:gamma}		
\end{figure}

\subsection{Quadratures for bilinear forms}

We denote by~$P_h^n \subset \bar P_h^n$ spaces of continuous and discontinuous nodal~$P_n$ interpolants defined on~$\Omega_\Gamma^h$, respectively. For a function~$f$, $I_h^n(f) \in P_h^n$ is the corresponding interpolant; we will use the notation~$f_h^n$ to emphasize that~$f_h^n \in P_h^n$ and~$f_h^n$ approximates~$f$ in some sense, but~$I_h^n(f) \ne f_h^n$.

We set
\begin{align}\label{gammah}
	\Gamma_h^n &\coloneqq \{ \vect x \in \mathbb{R}^3 : \big(I_h^n(\phi)\big)(\vect x) = 0 \}, \\
	\vect n_{\Gamma_h^n} &= \frac{\nabla I_h^n(\phi)}{\|\nabla I_h^n(\phi)\|} \not\in \bar{P}_h^m\text{ for any $m$ if $n > 1$}. \label{gammah:n}
\end{align}  
	Note that~$\Gamma_h^n$ is a continuous piecewise $P_n$ surface in~$\Omega_\Gamma^h$, and $\Gamma_h^n \ne I_h^n(\Gamma)$. The unit normal~$\vect n_{\Gamma_h^n}$ is not a rational function; it is continuous in~$T \in \Omega_\Gamma^h$ and discontinuous on faces. We also define
\begin{equation}\label{gammah2}
	\Gamma_{h/m}^{2 \rightarrow 1} \coloneqq \{ \vect x \in \mathbb{R}^3 : \Big(I_{h/m}^1\big(I_h^2(\phi)\big)\Big)(\vect x) = 0 \}.
\end{equation}  
Note that~$I_{h/2}^1\big(I_{h}^2(\phi)\big) = I_{h/2}^1(\phi)$ (since in order to build both~$I_{h/2}^1$ and~$I_{h}^2$ the same values of~$\phi$ are used), and~$I_{h/m}^1\big(I_{h}^2(\phi)\big) \ne I_{h/m}^1(\phi)$ for~$m > 2$. Thus we have~$\Gamma_{h/2}^{2 \rightarrow 1} = \Gamma_{h/2}^1$, and~$\Gamma_{h/m}^{2 \rightarrow 1} \ne \Gamma_{h/m}^1$ for~$m > 2$. We refer to Figures~\ref{fig:phi_exact} and~\ref{fig:phi_inexact}. 

We implemented two options for the matrix assembly~\eqref{mtx}. The first one is
\begin{align}\begin{split}\label{mtx_exact}
	\langle \vect A\,\vec{\vect u}, \vec{\vect v} \rangle &= 
		\int^5_{\Gamma_{h/m}^{2 \to 1}} \big( 2\,E_{s,\,\Gamma_{h}^2}(\vect u) : E_{s,\,\Gamma_{h}^2}(\vect v) + \vect u\cdot\vect v + \tau\,(\vect u\cdot\vect n_{\Gamma_{h}^2})\,(\vect v\cdot\vect n_{\Gamma_{h}^2}) \big) \diff{s} \\
	&
		+ \rho_u \int^5_{\Omega_h^{\Gamma}} \frac{\partial \vect u}{\partial\vect n_{\Gamma_{h}^2}}\cdot\frac{\partial \vect v}{\partial\vect n_{\Gamma_{h}^2}} \diff{\vect x}, \quad \vect A \in \mathbb R^{n_{\vect A} \times n_{\vect A}},\\
	\langle \vect B\,\vec{\vect u}, \vec{\vect q} \rangle &= 
		\int^5_{\Gamma_{h/m}^{2 \to 1}} \nabla_{\Gamma_h^2} q \cdot \vect u \diff{s}, \quad \vect B \in \mathbb R^{n_{\vect S} \times n_{\vect A}},\\
	\langle \vect M_0\,\vec{\vect p}, \vec{\vect q} \rangle &=
		\int^5_{\Gamma_{h/m}^{2 \to 1}} p\,q \diff{s}, \quad \vect M_0 \in \mathbb R^{n_{\vect S} \times n_{\vect S}},\\
	\langle \vect C_n\,\vec{\vect p}, \vec{\vect q} \rangle &=
		\rho_p \int^5_{\Omega^{\Gamma}_h} \frac{\partial p}{\partial\vect n_{\Gamma_{h}^2}} \frac{\partial q}{\partial\vect n_{\Gamma_{h}^2}} \diff{\vect x}, \quad \vect C_n \in \mathbb R^{n_{\vect S} \times n_{\vect S}},\\
	\langle \vect C_{\text{full}}\,\vec{\vect p}, \vec{\vect q} \rangle &=
		\rho_p \int^5_{\Omega^{\Gamma}_h} \nabla p \cdot \nabla q \diff{\vect x}, \quad \vect C_{\text{full}} \in \mathbb R^{n_{\vect S} \times n_{\vect S}}.	 
\end{split}\end{align}
\begin{itemize}
	\item $\int^5_{\Gamma_{h/m}^{2 \to 1}} \cdot \diff{s}$ denotes a composite quadrature rule that is exact for~$\bar P_h^5(\Gamma_{h/m}^{2 \to 1})$, i.e. this quadrature is exact for piecewise polynomials up to degree~5 on each triangular patch~$\gamma \in \Gamma_{h/m}^{2 \to 1}$, 
	\item $\int^5_{\Omega^{\Gamma}_h} \cdot \diff{\vect x}$ denotes a composite quadrature rule that is exact for~$\bar P_h^5(\Omega^{\Gamma}_h)$, i.e. this quadrature is exact for piecewise polynomials up to degree~5 on each tetrahedron~$T \in \Omega^{\Gamma}_h$, 
	\item $E_{s,\,\Gamma_{h}^2}$ and~$\nabla_{\Gamma_h^2}$ are defined as their continuous analogues with~$\vect n_{\Gamma}$ in~$\vect P_\Gamma$ replaced with~$\vect n_{\Gamma_{h}^2}$.
\end{itemize}
The second option is
\begin{align}\begin{split}\label{mtx_exact_2}
	\langle \vect A\,\vec{\vect u}, \vec{\vect v} \rangle &= 
		\int^5_{\Gamma_{h/m}^{2 \to 1}} \big( 2\,E_{s,\,\textcolor{DarkGreen}{\Gamma_{h/m}^{2 \to 1}}}(\vect u) : E_{s,\,\textcolor{DarkGreen}{\Gamma_{h/m}^{2 \to 1}}}(\vect v) + \vect u\cdot\vect v + \tau\,(\vect u\cdot\vect n_{\Gamma_{h}^2})\,(\vect v\cdot\vect n_{\Gamma_{h}^2}) \big) \diff{s} \\
	&
		+ \rho_u \int^5_{\Omega_h^{\Gamma}} \frac{\partial \vect u}{\partial\vect n_{\Gamma_{h}^2}}\cdot\frac{\partial \vect v}{\partial\vect n_{\Gamma_{h}^2}} \diff{\vect x}, \quad \vect A \in \mathbb R^{n_{\vect A} \times n_{\vect A}},\\
	\langle \vect B\,\vec{\vect u}, \vec{\vect q} \rangle &= 
		\int^5_{\Gamma_{h/m}^{2 \to 1}} \nabla_{\textcolor{DarkGreen}{\Gamma_{h/m}^{2 \to 1}}} q \cdot \vect u \diff{s}, \quad \vect B \in \mathbb R^{n_{\vect S} \times n_{\vect A}},\\
	\langle \vect M_0\,\vec{\vect p}, \vec{\vect q} \rangle &=
		\int^5_{\Gamma_{h/m}^{2 \to 1}} p\,q \diff{s}, \quad \vect M_0 \in \mathbb R^{n_{\vect S} \times n_{\vect S}},\\
	\langle \vect C_n\,\vec{\vect p}, \vec{\vect q} \rangle &=
		\rho_p \int^5_{\Omega^{\Gamma}_h} \frac{\partial p}{\partial\vect n_{\Gamma_{h}^2}} \frac{\partial q}{\partial\vect n_{\Gamma_{h}^2}} \diff{\vect x}, \quad \vect C_n \in \mathbb R^{n_{\vect S} \times n_{\vect S}},\\
	\langle \vect C_{\text{full}}\,\vec{\vect p}, \vec{\vect q} \rangle &=
		\rho_p \int^5_{\Omega^{\Gamma}_h} \nabla p \cdot \nabla q \diff{\vect x}, \quad \vect C_{\text{full}} \in \mathbb R^{n_{\vect S} \times n_{\vect S}}.
\end{split}\end{align}

\begin{figure}[H]
	\par\bigskip
	\centering
	\begin{subfigure}{.5\linewidth}
		\centering
		\includegraphicsw[.45]{patches_2.png}
	\end{subfigure}%
	\begin{subfigure}{.5\linewidth}
		\centering
		\includegraphicsw[.45]{normals_2.png}
	\end{subfigure}%
	\par\bigskip
	\begin{subfigure}{.5\linewidth}
		\centering
		\includegraphicsw[.45]{patches_4.png}
	\end{subfigure}%
	\begin{subfigure}{.5\linewidth}
		\centering
		\includegraphicsw[.45]{normals_4.png}
	\end{subfigure}%
	\par\bigskip
	\caption{$\Gamma = \sphere$, $\phi(\vect x) = \|\vect x\|^2 - 1$, $h = 8.33\times10^{-1}$. Top-left: triangular patches~$\gamma \subset \Gamma_{h/2}^{2 \rightarrow 1} = \Gamma_{h/2}^1$ (different color corresponds to a different tetrahedron $T \in \Omega^\Gamma_h$). Top-right: a patch~$\gamma$ and its normals. Bottom-left and bottom-right: same for~$\Gamma_{h/4}^{2 \rightarrow 1} = \Gamma_{h/4}^1$. \textbf{Note that since~$\phi \in P^2$, we have that~$\Gamma_{h/m}^{2 \rightarrow 1} = \Gamma_{h/m}^1 \rightarrow \Gamma$ as $m \rightarrow \infty$ even for fixed~$h$}}
	\label{fig:phi_exact}		
\end{figure}

\begin{figure}[H]
	\centering
	\begin{subfigure}{.5\linewidth}
		\centering
		\includegraphicsw[.45]{patches_2_inexact.png}
	\end{subfigure}%
	\begin{subfigure}{.5\linewidth}
		\centering
		\includegraphicsw[.45]{patches_4_inexact.png}
	\end{subfigure}%
	\par\bigskip
	\caption{$\Gamma = \sphere$, $\phi(\vect x) = \|\vect x\|^{1/2} - 1$, $h = 8.33\times10^{-1}$. Left: triangular patches~$\gamma \subset \Gamma_{h/2}^1$ (different color corresponds to a different tetrahedron $T \in \Omega^\Gamma_h$). Right: same for~$\Gamma_{h/4}^{2 \rightarrow 1} \ne \Gamma_{h/4}^1$. \textbf{Note that since~$\phi \not\in \bar{P}^2_h$, we have that~$\Gamma_{h/m}^{2 \rightarrow 1} \ne \Gamma_{h/m}^1$ for~$m > 2$, and $\Gamma_{h/m}^{2 \rightarrow 1} \rightarrow \Gamma_h^2 \ne \Gamma$ as $m \rightarrow \infty$ for fixed~$h$}}
	\label{fig:phi_inexact}		
\end{figure}

As in~\cite{veclaplace}, we refer to~\eqref{mtx} as \textbf{inconsistent formulation}. We also consider the same formulation as in~~\eqref{mtx} but with the first term~$\vect A_s$ in the definition of~$\vect A$ changed as
\begin{equation}\label{mtx_cons}
	\langle \vect A_s\,\vec{\vect u}, \vec{\vect v} \rangle \approx \int^5_{\Gamma} 2\,\big( E_{s,\,\Gamma}(\vect u) - (\vect u\cdot\vect n_{\Gamma})\,\vect H_{\Gamma} \big) : \big( E_{s,\,\Gamma}(\vect v) - (\vect v\cdot\vect n)\,\vect H_{\Gamma} \big) \diff{s},
\end{equation}
where the shape operator is defined as~$\vect H_\Gamma \coloneqq \nabla_\Gamma \vect n_\Gamma \coloneqq \vect P_{\Gamma}\,\nabla \vect n^e_\Gamma\,\vect P_{\Gamma}$, $\vect H_\Gamma : \mathcal{O}(\Gamma) \rightarrow \mathbb{R}^3$. We refer to~\eqref{mtx_cons} as \textbf{consistent formulation}.

Similarly to~\eqref{mtx_exact} and~\eqref{mtx_exact}, we consider two discretizations of~\eqref{mtx_cons}:
\begin{equation}\label{mtx_cons_exact}
	\langle \vect A_s\,\vec{\vect u}, \vec{\vect v} \rangle = \int^5_{\Gamma_{h/m}^{2 \to 1}} 2\,\big( E_{s,\,\Gamma_{h}^{2}}(\vect u) - (\vect u\cdot\vect n_{\Gamma_h^2})\,\vect H_{\Gamma_h^2} \big) : \big( E_{s,\,\Gamma_{h}^{2}}(\vect v) - (\vect v\cdot\vect n_{\Gamma_h^2})\,\vect H_{\Gamma_h^2} \big) \diff{s} 
\end{equation}
and
\begin{equation}\label{mtx_cons_exact_2}
	\langle \vect A_s\,\vec{\vect u}, \vec{\vect v} \rangle = \int^5_{\Gamma_{h/m}^{2 \to 1}} 2\,\big( E_{s,\,\Gamma_{h/m}^{2 \to 1}}(\vect u) - (\vect u\cdot\vect n_{\Gamma_{h/m}^{2 \to 1}})\,\vect H_{\Gamma_h^2} \big) : \big( E_{s,\,\Gamma_{h/m}^{2 \to 1}}(\vect v) - (\vect v\cdot\vect n_{\Gamma_{h/m}^{2 \to 1}})\,\vect H_{\Gamma_h^2} \big) \diff{s}.
\end{equation}

Note that~$\vect n_\Gamma = \nabla \phi / \|\nabla \phi\|$ is defined in~$\mathcal{O}(\Gamma)$, so~$\nabla \vect n_\Gamma$ makes sense and
\begin{equation*}
	\nabla \vect n_\Gamma = \Big(\vect I - \frac{\nabla \phi\,\nabla \phi^T}{\|\nabla \phi\|^2}\Big)\frac{\nabla^2 \phi}{\|\nabla \phi\|} = \vect P_{\Gamma}\,\frac{\nabla^2 \phi}{\|\nabla \phi\|}.
\end{equation*}
If~$\vect n_\Gamma = \vect n^e_\Gamma$, one gets
\begin{equation}\label{H}
	\vect H_\Gamma = \vect P_{\Gamma}\,\frac{\nabla^2 \phi}{\|\nabla \phi\|}\,\vect P_{\Gamma}.
\end{equation}
Thus we define~$\vect H_{\Gamma^2_h}$ to be as in~\eqref{H} but with $\phi$ replaced with~$I^2_h(\phi)$, i.e.
\begin{equation}\label{Hh}
	\vect H_{\Gamma_h^2} \coloneqq \vect P_{\Gamma_h^2}\,\frac{\nabla^2 I_h^2(\phi)}{\|\nabla I_h^2(\phi)\|}\,\vect P_{\Gamma_h^2}.
\end{equation}
Indeed, computation of~$\vect H_{\Gamma^2_h}$ requires Hessians of shape functions.

Depending on the choice of~$\phi$, we may or may not have~$\vect n_\Gamma = \vect n^e_\Gamma$. Note that the choice~$\phi = d$ is sufficient for this, but not necessary. Consider this choices of~$\phi$ for $\sphere$:
\begin{enumerate}
	\item $\phi_1(\vect x) = \|\vect x\| - 1 = d(\vect x)$, $\nabla \phi_1 / \|\nabla \phi_1\| = \vect n_\Gamma^e$,
	\item $\phi_2(\vect x) = \|\vect x\|^2 - 1 \in P^2$, $\nabla \phi_2 / \|\nabla \phi_2\| = \nabla \phi_1 / \|\nabla \phi_1\| = \vect n_\Gamma^e$,
	\item $\phi_3(\vect x) = e^{\phi_2(\vect x)}\,x^2 + y^2 + z^2 -1$, $\nabla \phi_3 / \|\nabla \phi_3\| \ne \vect n_\Gamma^e$, i.e. $\nabla \phi_3 / \|\nabla \phi_3\| = \vect n_\Gamma$ only on~$\sphere$.
\end{enumerate}
As for the case 2: note that if~$\phi$ is piecewise quadratic in~$\Omega^\Gamma_h$ and defines a normal that is equal to its extension, then~$\vect H_{\Gamma^2_h} = \vect H_{\Gamma}$, i.e. the approximation is \textbf{exact}.

For the approach~\eqref{mtx_exact_2}, there is also an option to approximate~$\vect H$ as~$\vect P_{\Gamma_{h/m}^{2\to 1}}\,\frac{\nabla^2 I_h^2(\phi)}{\|\nabla I_h^2(\phi)\|}\,\vect P_{\Gamma_{h/m}^{2\to 1}}$ since we build~$\vect P_{\Gamma_{h/m}^{2\to 1}}$ anyway. We chose to use~\eqref{Hh} for both~\eqref{mtx_exact} and~\eqref{mtx_exact_2}.

\subsection{Error computation}

Note that~$\HOneSpace$-error (for e.g. $\vect P_2$\,--\,$P_1$ FE) can be cheaply approximated as~$\langle \vect w, \vect A_s\,\vect w \rangle^{1/2}$, $\vect w \coloneqq$ vector of d.o.f. corresponding to $\vect P^2_h$ interpolant $I_h^2(\vect u^e) - \vect u_h$, $\vect A_s \coloneqq$ matrix corresponding to the first term of~$\vect A$ in~\eqref{mtx_exact_2}. Thus the errors are approximated as
\begin{align}\begin{split}
	\| \vect u - \vect u_h \|_{\HOneSpace} &= \| I^k_h(\vect u^e) - \vect u_h \|_{\HOneSpace[\Gamma_{h/m}^{2 \to 1}]} + O(h^{k}), \\
	\| \vect u - \vect u_h \|_{\LTwoSpace} &= \| I^k_h(\vect u^e) - \vect u_h \|_{\LTwoSpace[\Gamma_{h/m}^{2 \to 1}]} + O(h^{k+1}), \\
	\| p - p_h \|_{\LTwoSpace} &= \| I^1_h(p^e) - p_h \|_{\LTwoSpace[\Gamma_{h/m}^{2 \to 1}]} + O(h^2)
\end{split}\end{align}
for~$m > 1$. Here $k = 1$ for~$\vect P_1$\,--\,$P_1$ FEM and $k = 2$ for~$\vect P_2$\,--\,$P_1$. For consistent penalty approach matrix~$\vect A_s$ is computed as in~\eqref{mtx_cons_exact} or~\eqref{mtx_cons_exact_2}.

\section{Convergence results}\label{sec:conv}

\subsection{Manufactured solution}

We solve model problem from~\cite[p.\,20]{surfstokes}, $\Gamma = \sphere$\footnotemark{}. We set
\begin{equation}\label{exact_soln}
	\tilde{\vect u}(x, y, z) \coloneqq (-z^2, y, x)^T, \quad
	\tilde p(x, y, z) \coloneqq x\,y^2 + z, \quad
	\phi(\vect x) \coloneqq \|\vect x\|^2 - 1.
\end{equation}
The exact solution on the unit sphere is chosen as 
\begin{equation}\label{exact_soln_2}
	\vect u \coloneqq \vect P\,\tilde{\vect u}^e, \quad
	p \coloneqq \tilde p^e.
\end{equation}

\footnotetext{In~\cite{surfstokes} they use~$\vect u \coloneqq \vect P\,\tilde{\vect u}$, i.e. $\vect u \ne \vect u^e$. I prefer~$\vect u \equiv \vect u^e$ as in~\cite{veclaplace}.}

Thus we have~$\int_\Gamma p \diff{\vect x} = 0$, $p \equiv p^e$, $\vect u \equiv \vect u^e$ in $\mathcal O(\Gamma)$, and $\vect u$ is a tangential field. Note that for our choice of~$\phi$ in~\eqref{exact_soln} we have
\begin{equation}\label{exact_soln_simpl}
	\vect n_{\Gamma^2_h} = \vect n_\Gamma^e\text{ in }\mathcal O(\Gamma), \quad 
	\Gamma^{2\to 1}_{h/m} = \Gamma^1_{h/m}, \quad
	\vect n_{\Gamma^{2\to 1}_{h/m}} = \vect n_{\Gamma^1_{h/m}}\text{ on }\Gamma,
\end{equation}
and 
\begin{equation}\label{exact_soln_conv}
	\vect n_{\Gamma^1_{h/m}} \rightarrow \vect n_\Gamma, \quad
	\Gamma^1_{h/m} \rightarrow \Gamma
\end{equation}
as one increases~$m$ \textbf{even for fixed~$h$}.

\begin{figure}[h]
	\centering
	\begin{subfigure}{.5\linewidth}
		\centering
		\includegraphicsw[.8]{u_soln.png}
	\end{subfigure}%
	\begin{subfigure}{.5\linewidth}
		\centering
		\includegraphicsw[.8]{p_soln.png}
	\end{subfigure}%
	\caption{Exact velocity solution (Left) and pressure solution (Right) as in~\eqref{exact_soln_2}}
	\label{fig:soln}		
\end{figure}

We have
\begin{align}
	\vect n_{\sphere}(\vect x) &= \|\vect x\|^{-1}\,\vect x = \vect n_{\sphere}^e(\vect x), \\
	\vect P_{\sphere}(\vect x) &= \|\vect x\|^{-2}\begin{pmatrix}
		y^2+z^2 & -x y & -x z \\
		-x y & x^2+z^2 & -y z \\
		-x z & -y z & x^2+y^2 
	\end{pmatrix} = \vect P^e_{\sphere}(\vect x), \\
	\vect H_{\sphere}(\vect x) &= \|\vect x\|^{-3}\begin{pmatrix}
		y^2+z^2 & -x y & -x z \\
		-x y & x^2+z^2 & -y z \\
		-x z & -y z & x^2+y^2 \\
	\end{pmatrix} \ne \vect H^e_{\sphere}(\vect x) = \vect P_{\sphere}(\vect x).
\end{align}

We consider two choices for virtual refinement: $m \propto h^{-1/2}$ and $m \propto h^{-1}$. The first choice assures $h^3$-accurate approximation of~$\Gamma$ and $h^{3/2}$ accurate approximation of the normal vector, whereas the second choice assures $h^4$- and $h^2$-approximations. We refer to Figure~\ref{fig:m}. 

\begin{figure}[H]
	\centering
	\begin{subfigure}{.5\linewidth}
		\centering
		\includegraphicsw[1.]{m_plot_new.png}
	\end{subfigure}%
	\quad
	\begin{subfigure}{.4\linewidth}
		$\vect m_{1/2} \coloneqq \{2, 3, 4, 6, 9, 12, 18, 25\}$, 
		
		$\vect m_{1} \coloneqq \{{1, 2, 4, 8, 15, 31, 61, 123}\}$.
		\vskip .3cm
		$m \in \vect m_{1/2}$ behaves as~$h^{-1/2}$, and $m \in \vect m_1$ behaves as~$h^{-1}$
	\end{subfigure}%
	\caption{Virtual refinement parameter $m$ for~$\Gamma_{h/m}^{2\to 1}$}
	\label{fig:m}		
\end{figure}

\begin{table}[H]
	\centering\footnotesize
	\caption{Errors for normals and shape operator. Please see~\eqref{exact_soln_simpl} and~\eqref{exact_soln_conv}}
	\label{tab:shape_normal_conv}
	\begin{subtable}{1.\linewidth}\centering
		\begin{tabular}[1.4]{|c||c||c|c||c|c|}
			\hline
			\multicolumn{6}{|c|}{$m \in \vect m_{1/2}$ as in Figure~\ref{fig:m}} \\
			\hline
			$h$ & $\|\vect n^e_\Gamma - \vect n_{\Gamma_{h}^2}\|_{\LTwoSpace[\Gamma_{h/m}^1]}$ & $\|\vect n^e_\Gamma - \vect n_{\Gamma_{h/m}^1} \|_{\LTwoSpace[\Gamma_{h/m}^1]}$ & Order & $\|\vect H_{\Gamma}^e - \vect H_{\Gamma^2_h}\|_{\LTwoSpace[\Gamma_{h/m}^1]}$ & Order \\
			\hline
			$8.3\times	10^{-1}$	&	$8.1\times	10^{-16}$	&	$6.4\times	10^{-1}$	&	$\text{}$	&	$2.3\times	10^{-1}$	&	$\text{}$	\\ \hline
$4.2\times	10^{-1}$	&	$1.1\times	10^{-15}$	&	$2.\times	10^{-1}$	&	$1.7$	&	$2.5\times	10^{-2}$	&	$3.2$	\\ \hline
$2.1\times	10^{-1}$	&	$2.3\times	10^{-15}$	&	$7.6\times	10^{-2}$	&	$1.4$	&	$3.5\times	10^{-3}$	&	$2.8$	\\ \hline
$1.\times	10^{-1}$	&	$5.2\times	10^{-15}$	&	$2.5\times	10^{-2}$	&	$1.6$	&	$3.9\times	10^{-4}$	&	$3.2$	\\ \hline
$5.2\times	10^{-2}$	&	$9.3\times	10^{-15}$	&	$8.4\times	10^{-3}$	&	$1.6$	&	$4.3\times	10^{-5}$	&	$3.2$	\\ \hline
		\end{tabular}
	\end{subtable}
	%	\vskip 2mm	
	%	\begin{subtable}{1.\linewidth}\centering
	%		\begin{tabular}[1.4]{|c||c||c|c||c|c|}
	%			\hline
	%			\multicolumn{6}{|c|}{$m \in \vect m_1$ as in Figure~\ref{fig:m}} \\
	%			\hline
	%			$h$ & $\|\vect n^e_\Gamma - \vect n_{\Gamma_{h}^2}\|_{\LTwoSpace[\Gamma_{h/m}^1]}$ & $\|\vect n^e_\Gamma - \vect n_{\Gamma_{h/m}^1} \|_{\LTwoSpace[\Gamma_{h/m}^1]}$ & Order & $\|\vect H_{\Gamma}^e - \vect H_{\Gamma^2_h}\|_{\LTwoSpace[\Gamma_{h/m}^1]}$ & Order \\
	%			\hline
	%			$4.2\times	10^{-1}$	&	$1.1\times	10^{-15}$	&	$2.9\times	10^{-1}$	&	$\text{}$	&	$5.7\times	10^{-2}$	&	$\text{}$	\\ \hline
$2.1\times	10^{-1}$	&	$2.3\times	10^{-15}$	&	$7.6\times	10^{-2}$	&	$2.$	&	$3.5\times	10^{-3}$	&	$4.$	\\ \hline
$1.\times	10^{-1}$	&	$5.2\times	10^{-15}$	&	$1.9\times	10^{-2}$	&	$2.$	&	$2.2\times	10^{-4}$	&	$4.$	\\ \hline
$5.2\times	10^{-2}$	&	$9.3\times	10^{-15}$	&	$4.7\times	10^{-3}$	&	$2.$	&	$1.4\times	10^{-5}$	&	$4.$	\\ \hline
$2.6\times	10^{-2}$	&	$1.9\times	10^{-14}$	&	$1.2\times	10^{-3}$	&	$2.$	&	$8.6\times	10^{-7}$	&	$4.$	\\ \hline
$1.3\times	10^{-2}$	&	$3.6\times	10^{-14}$	&	$2.9\times	10^{-4}$	&	$2.$	&	$5.4\times	10^{-8}$	&	$4.$	\\ \hline
	%		\end{tabular}
	%	\end{subtable}
\end{table}

%\subsection{$\text{P}_1$\,--\,$P_1$ Trace\,FEM}
%
%Here we compare approaches~\eqref{mtx_exact} and~\eqref{mtx_exact_2}. We use one virtual refinement for surface integrals, $m = 2$, so we have $\Gamma_{h/2}^{2 \to 1} = \Gamma_{h/2}^1$ (see~\eqref{gammah} and~\eqref{gammah2}). We use the full stabilization matrix~$\vect C_{\text{full}}$. Note that~$\phi \in P_2$ in~\eqref{exact_soln}, and hence~$\Gamma_h^2 = \Gamma$. Thus~\eqref{mtx_exact} boils down to
%\begin{align}\begin{split}\label{mtx_exact_p1}
%	\langle \vect A\,\vec{\vect u}, \vec{\vect v} \rangle &= 
%		\int^5_{\Gamma_{h/2}^1} \big( E_{s,\,\Gamma}(\vect u) : E_{s,\,\Gamma}(\vect v) + \vect u\cdot\vect v + \tau\,(\vect u\cdot\vect n_{\Gamma})\,(\vect v\cdot\vect n_{\Gamma}) \big) \diff{s} \\
%	&
%		+ \rho_u \int^5_{\Omega_h^{\Gamma}} \frac{\partial \vect u}{\partial\vect n_{\Gamma}}\cdot\frac{\partial \vect v}{\partial\vect n_{\Gamma}} \diff{\vect x}, \quad \vect A \in \mathbb R^{n_{\vect A} \times n_{\vect A}},\\
%	\langle \vect B\,\vec{\vect u}, \vec{\vect q} \rangle &= 
%		-\int^5_{\Gamma_{h/2}^1} q\,\Div_{\Gamma} \vect u \diff{s}, \quad \vect B \in \mathbb R^{n_{\vect S} \times n_{\vect A}},\\
%	\langle \vect M_0\,\vec{\vect p}, \vec{\vect q} \rangle &=
%		\int^5_{\Gamma_{h/2}^1} p\,q \diff{s}, \quad \vect M_0 \in \mathbb R^{n_{\vect S} \times n_{\vect S}},\\
%	\langle \vect C_{\text{full}}\,\vec{\vect p}, \vec{\vect q} \rangle &=
%		\rho_p \int^5_{\Omega^{\Gamma}_h} \nabla p \cdot \nabla q \diff{\vect x}, \quad \vect C_{\text{full}} \in \mathbb R^{n_{\vect S} \times n_{\vect S}},		 
%\end{split}\end{align}
%and e.g. the integrand that involves $E_{s,\,\Gamma} \equiv E_s$ is exact. Similarly, approach~\eqref{mtx_exact_2} boils down to
%\begin{align}\begin{split}\label{mtx_exact_2_p1}
%	\langle \vect A\,\vec{\vect u}, \vec{\vect v} \rangle &= 
%		\int^5_{\Gamma_{h/2}^1} \big( E_{s,\,\Gamma_{h/2}^1}(\vect u) : E_{s,\,\Gamma_{h/2}^1}(\vect v) + \vect u\cdot\vect v + \tau\,(\vect u\cdot\vect n_{\Gamma})\,(\vect v\cdot\vect n_{\Gamma}) \big) \diff{s} \\
%	&
%		+ \rho_u \int^5_{\Omega_h^{\Gamma}} \frac{\partial \vect u}{\partial\vect n_{\Gamma}}\cdot\frac{\partial \vect v}{\partial\vect n_{\Gamma}} \diff{\vect x}, \quad \vect A \in \mathbb R^{n_{\vect A} \times n_{\vect A}},\\
%	\langle \vect B\,\vec{\vect u}, \vec{\vect q} \rangle &= 
%		-\int^5_{\Gamma_{h/2}^1} q\,\Div_{\Gamma_{h/2}^1} \vect u \diff{s}, \quad \vect B \in \mathbb R^{n_{\vect S} \times n_{\vect A}},\\
%	\langle \vect M_0\,\vec{\vect p}, \vec{\vect q} \rangle &=
%		\int^5_{\Gamma_{h/2}^1} p\,q \diff{s}, \quad \vect M_0 \in \mathbb R^{n_{\vect S} \times n_{\vect S}},\\
%	\langle \vect C_{\text{full}}\,\vec{\vect p}, \vec{\vect q} \rangle &=
%		\rho_p \int^5_{\Omega^{\Gamma}_h} \nabla p \cdot \nabla q \diff{\vect x}, \quad \vect C_{\text{full}} \in \mathbb R^{n_{\vect S} \times n_{\vect S}}.
%\end{split}\end{align}
%
%For statistics: using 64 CPUs, computation of the meshlevel~6 ($h = 2.6\times10^{-2}$) takes~${\sim}14$ minutes, meshlevel~7 takes~${\sim}75$ minutes, and meshlevel~8 takes~${\sim}7.3$ hours.
%
%\begin{table}[H]
%	\centering\footnotesize
%	\caption{Convergence results. Matrices are assembled as in~\eqref{mtx_exact_p1} (top table) and~\eqref{mtx_exact_2_p1} (bottom table)}
%	\label{tab:p1p1_conv}
%	\begin{subtable}{1.\linewidth}\centering
%		\begin{tabular}[1.3]{|c||c||c|c||c|c||c|c|}
%			\hline
%			$m$ & $h$ & $\|\vect u - \vect u_h\|_{\HOneSpace}$ & Order & $\|\vect u - \vect u_h\|_{\LTwoSpace}$ & Order & $\|p - p_h\|_{\LTwoSpace}$ & Order \\
%			\hline
%			\multirow{8}{*}{2}&$8.33\times	10^{-1}$	&	$3.4$	&	$\text{}$	&	$2.2$	&	$\text{}$	&	$1.1$	&	$\text{}$	\\ \cline{2-8}
&$4.17\times	10^{-1}$	&	$1.8$	&	$9.39\times	10^{-1}$	&	$1.1$	&	$9.57\times	10^{-1}$	&	$9.3\times	10^{-1}$	&	$2.56\times	10^{-1}$	\\ \cline{2-8}
&$2.08\times	10^{-1}$	&	$7.6\times	10^{-1}$	&	$1.23$	&	$3.6\times	10^{-1}$	&	$1.65$	&	$5.1\times	10^{-1}$	&	$8.75\times	10^{-1}$	\\ \cline{2-8}
&$1.04\times	10^{-1}$	&	$3.1\times	10^{-1}$	&	$1.3$	&	$1.\times	10^{-1}$	&	$1.85$	&	$1.8\times	10^{-1}$	&	$1.46$	\\ \cline{2-8}
&$5.21\times	10^{-2}$	&	$1.3\times	10^{-1}$	&	$1.21$	&	$2.6\times	10^{-2}$	&	$1.95$	&	$5.3\times	10^{-2}$	&	$1.79$	\\ \cline{2-8}
&$2.6\times	10^{-2}$	&	$6.4\times	10^{-2}$	&	$1.05$	&	$6.5\times	10^{-3}$	&	$1.98$	&	$1.5\times	10^{-2}$	&	$1.84$	\\ \cline{2-8}
&$1.3\times	10^{-2}$	&	$3.2\times	10^{-2}$	&	$1.01$	&	$1.7\times	10^{-3}$	&	$1.97$	&	$6.6\times	10^{-3}$	&	$1.17$	\\ \cline{2-8}
\rowcolor{LightRed}
&$6.51\times	10^{-3}$	&	$1.6\times	10^{-2}$	&	$9.93\times	10^{-1}$	&	$5.1\times	10^{-4}$	&	$1.72$	&	$5.5\times	10^{-3}$	&	$2.59\times	10^{-1}$	\\ \hline
\rowcolor{LightGreen}
4&$6.51\times	10^{-3}$	&	$1.7\times	10^{-2}$	&	$9.2\times	10^{-1}$	&	$4.1\times	10^{-4}$	&	$2.02$	&	$1.6\times	10^{-3}$	&	$2.06$	\\ \hline
%		\end{tabular}
%	\end{subtable}
%	\vskip 4mm	
%	\begin{subtable}{1.\linewidth}\centering
%		\begin{tabular}[1.3]{|c||c|c||c|c||c|c|}
%			\hline
%			$h$ & $\|\vect u - \vect u_h\|_{\HOneSpace}$ & Order & $\|\vect u - \vect u_h\|_{\LTwoSpace}$ & Order & $\|p - p_h\|_{\LTwoSpace}$ & Order \\
%			\hline
%			$8.33\times	10^{-1}$	&	$3.4$	&	$\text{}$	&	$2.2$	&	$\text{}$	&	$1.1$	&	$\text{}$	\\ \hline
$4.17\times	10^{-1}$	&	$1.8$	&	$9.28\times	10^{-1}$	&	$1.1$	&	$9.3\times	10^{-1}$	&	$9.3\times	10^{-1}$	&	$2.58\times	10^{-1}$	\\ \hline
$2.08\times	10^{-1}$	&	$7.6\times	10^{-1}$	&	$1.23$	&	$3.6\times	10^{-1}$	&	$1.65$	&	$5.1\times	10^{-1}$	&	$8.62\times	10^{-1}$	\\ \hline
$1.04\times	10^{-1}$	&	$3.1\times	10^{-1}$	&	$1.29$	&	$1.\times	10^{-1}$	&	$1.85$	&	$1.9\times	10^{-1}$	&	$1.44$	\\ \hline
$5.21\times	10^{-2}$	&	$1.3\times	10^{-1}$	&	$1.22$	&	$2.6\times	10^{-2}$	&	$1.94$	&	$5.5\times	10^{-2}$	&	$1.77$	\\ \hline
$2.6\times	10^{-2}$	&	$6.4\times	10^{-2}$	&	$1.07$	&	$6.7\times	10^{-3}$	&	$1.97$	&	$1.5\times	10^{-2}$	&	$1.89$	\\ \hline
$1.3\times	10^{-2}$	&	$3.1\times	10^{-2}$	&	$1.02$	&	$1.7\times	10^{-3}$	&	$1.98$	&	$4.\times	10^{-3}$	&	$1.89$	\\ \hline
\rowcolor{LightGreen}
$6.51\times	10^{-3}$	&	$1.5\times	10^{-2}$	&	$1.02$	&	$4.3\times	10^{-4}$	&	$1.99$	&	$1.2\times	10^{-3}$	&	$1.77$	\\ \hline
%		\end{tabular}
%	\end{subtable}
%\end{table}
%
%\begin{table}[H]
%	\centering\footnotesize
%	\caption{Solver statistics for~\eqref{mtx_exact_p1} (left table) and~\eqref{mtx_exact_2_p1} (right table)}
%	\label{tab:p1p1_iters}
%	\begin{subtable}{.5\linewidth}\centering
%		\begin{tabular}[1.3]{|c|c|c|}
%			\hline
%			$h$ & Outer iterations & Residual norm \\
%			\hline
%			$8.33\times	10^{-1}$	&	$14$	&	$1.\times	10^{-8}$	\\ \hline
$4.17\times	10^{-1}$	&	$20$	&	$9.2\times	10^{-9}$	\\ \hline
$2.08\times	10^{-1}$	&	$26$	&	$8.8\times	10^{-9}$	\\ \hline
$1.04\times	10^{-1}$	&	$29$	&	$5.4\times	10^{-9}$	\\ \hline
$5.21\times	10^{-2}$	&	$29$	&	$7.1\times	10^{-9}$	\\ \hline
$2.6\times	10^{-2}$	&	$29$	&	$4.5\times	10^{-9}$	\\ \hline
$1.3\times	10^{-2}$	&	$27$	&	$9.5\times	10^{-9}$	\\ \hline
$6.51\times	10^{-3}$	&	$30$	&	$6.5\times	10^{-9}$	\\ \hline
%		\end{tabular}
%	\end{subtable}%
%	\begin{subtable}{.5\linewidth}\centering
%		\begin{tabular}[1.3]{|c|c|c|}
%			\hline
%			$h$ & Outer iterations & Residual norm \\
%			\hline
%			$8.33\times	10^{-1}$	&	$15$	&	$1.7\times	10^{-9}$	\\ \hline
$4.17\times	10^{-1}$	&	$20$	&	$7.7\times	10^{-9}$	\\ \hline
$2.08\times	10^{-1}$	&	$26$	&	$7.\times	10^{-9}$	\\ \hline
$1.04\times	10^{-1}$	&	$29$	&	$4.3\times	10^{-9}$	\\ \hline
$5.21\times	10^{-2}$	&	$29$	&	$5.2\times	10^{-9}$	\\ \hline
$2.6\times	10^{-2}$	&	$27$	&	$8.5\times	10^{-9}$	\\ \hline
$1.3\times	10^{-2}$	&	$27$	&	$3.6\times	10^{-9}$	\\ \hline
$6.51\times	10^{-3}$	&	$29$	&	$7.4\times	10^{-9}$	\\ \hline
%		\end{tabular}
%	\end{subtable}
%\end{table}
%
%%\begin{table}[h!]
%%	\centering\small
%%	\caption{$\vect P_1$\,--\,$P_1$, $\Gamma = \sphere$. Here we use $\Gamma + \alpha_h\,\vect s$ to build (refine) the bulk mesh, and then solve the actual problem using~$\Gamma$. We set~$\alpha_h \coloneqq 0.3\,h$, $\vect s \coloneqq (1, 1, 1)^T/\sqrt{3}$}
%%	\label{tab:p1p1_conv_shift}
%%	\begin{subtable}{1.\linewidth}\centering
%%		\begin{tabular}[1.3]{|c||c|c||c|c||c|c|}
%%			\hline
%%			$h$ & $\|\vect u - \vect u_h\|_{\HOneSpace}$ & Order & $\|\vect u - \vect u_h\|_{\LTwoSpace}$ & Order & $\|p - p_h\|_{\LTwoSpace}$ & Order \\
%%			\hline
%%			$8.33\times	10^{-1}$	&	$5.6$	&	$\text{}$	&	$2.4$	&	$\text{}$	&	$1.4$	&	$\text{}$	\\ \hline
$4.17\times	10^{-1}$	&	$3.8$	&	$5.39\times	10^{-1}$	&	$1.2$	&	$9.79\times	10^{-1}$	&	$9.9\times	10^{-1}$	&	$4.86\times	10^{-1}$	\\ \hline
$2.08\times	10^{-1}$	&	$2.$	&	$9.62\times	10^{-1}$	&	$3.9\times	10^{-1}$	&	$1.65$	&	$5.2\times	10^{-1}$	&	$9.24\times	10^{-1}$	\\ \hline
$1.04\times	10^{-1}$	&	$9.8\times	10^{-1}$	&	$1.$	&	$1.1\times	10^{-1}$	&	$1.86$	&	$1.9\times	10^{-1}$	&	$1.47$	\\ \hline
$5.21\times	10^{-2}$	&	$4.9\times	10^{-1}$	&	$9.97\times	10^{-1}$	&	$2.8\times	10^{-2}$	&	$1.95$	&	$5.4\times	10^{-2}$	&	$1.79$	\\ \hline
$2.6\times	10^{-2}$	&	$2.5\times	10^{-1}$	&	$1.$	&	$7.\times	10^{-3}$	&	$1.98$	&	$1.5\times	10^{-2}$	&	$1.82$	\\ \hline
$1.3\times	10^{-2}$	&	$1.2\times	10^{-1}$	&	$9.9\times	10^{-1}$	&	$1.8\times	10^{-3}$	&	$1.97$	&	$6.7\times	10^{-3}$	&	$1.19$	\\ \hline
$6.51\times	10^{-3}$	&	$6.6\times	10^{-2}$	&	$9.12\times	10^{-1}$	&	$5.3\times	10^{-4}$	&	$1.74$	&	$5.5\times	10^{-3}$	&	$2.86\times	10^{-1}$	\\ \hline
%%		\end{tabular}
%%	\end{subtable}
%%	\vskip 4mm
%%	\begin{subtable}{1.\linewidth}\centering
%%		\begin{tabular}[1.3]{|c|c|c|}
%%			\hline
%%			$h$ & Outer iterations & Residual norm \\
%%			\hline
%%			$8.33\times	10^{-1}$	&	$14$	&	$1.\times	10^{-8}$	\\ \hline
$4.17\times	10^{-1}$	&	$20$	&	$9.2\times	10^{-9}$	\\ \hline
$2.08\times	10^{-1}$	&	$26$	&	$8.7\times	10^{-9}$	\\ \hline
$1.04\times	10^{-1}$	&	$29$	&	$5.2\times	10^{-9}$	\\ \hline
$5.21\times	10^{-2}$	&	$29$	&	$7.4\times	10^{-9}$	\\ \hline
$2.6\times	10^{-2}$	&	$29$	&	$5.6\times	10^{-9}$	\\ \hline
$1.3\times	10^{-2}$	&	$29$	&	$3.4\times	10^{-9}$	\\ \hline
$6.51\times	10^{-3}$	&	$30$	&	$6.7\times	10^{-9}$	\\ \hline
%%		\end{tabular}
%%	\end{subtable}
%%\end{table}
%
%The errors in Table~\ref{tab:p1p1_conv} are computed as explained in section~\ref{subsec:err}.

\subsection{$\text{P}_2$\,--\,$P_1$ Trace\,FEM}

Next we compare inconsistent and consistent Trace\,FEM penalty formulations. 

\subsubsection{Inconsistent penalty formulation}

%We use the normal stabilization matrix~$\vect C_n$. We stick to the approach~\eqref{mtx_exact_2}, so with~$\phi$ in~\eqref{exact_soln} we have
%\begin{align}\begin{split}\label{mtx_exact_2_p2}
%	\langle \vect A\,\vec{\vect u}, \vec{\vect v} \rangle &= 
%		\int^5_{\Gamma_{h/m}^1} \big( E_{s,\,\Gamma_{h/m}^1}(\vect u) : E_{s,\,\Gamma_{h/m}^1}(\vect v) + \vect u\cdot\vect v + \tau\,(\vect u\cdot\vect n_{\Gamma})\,(\vect v\cdot\vect n_{\Gamma}) \big) \diff{s} \\
%	&
%		+ \rho_u \int^5_{\Omega_h^{\Gamma}} \frac{\partial \vect u}{\partial\vect n_{\Gamma}}\cdot\frac{\partial \vect v}{\partial\vect n_{\Gamma}} \diff{\vect x}, \quad \vect A \in \mathbb R^{n_{\vect A} \times n_{\vect A}},\\
%	\langle \vect B\,\vec{\vect u}, \vec{\vect q} \rangle &= 
%		-\int^5_{\Gamma_{h/m}^1} q\,\Div_{\Gamma_{h/m}^1} \vect u \diff{s}, \quad \vect B \in \mathbb R^{n_{\vect S} \times n_{\vect A}},\\
%	\langle \vect M_0\,\vec{\vect p}, \vec{\vect q} \rangle &=
%		\int^5_{\Gamma_{h/m}^1} p\,q \diff{s}, \quad \vect M_0 \in \mathbb R^{n_{\vect S} \times n_{\vect S}},\\
%	\langle \vect C_n\,\vec{\vect p}, \vec{\vect q} \rangle &=
%		\rho_p \int_{\Omega^{\Gamma}_h}^5 \frac{\partial p}{\partial\vect n_{\Gamma}} \frac{\partial q}{\partial\vect n_{\Gamma}} \diff{\vect x}, \quad \vect C_n \in \mathbb R^{n_{\vect S} \times n_{\vect S}}.
%\end{split}\end{align}
We use the normal stabilization matrix~$\vect C_n$. We stick to the approach~\eqref{mtx_exact}, so with~\eqref{exact_soln_simpl} we have
\begin{align}\begin{split}\label{mtx_exact_p2}
	\langle \vect A\,\vec{\vect u}, \vec{\vect v} \rangle &= 
		\int^5_{\Gamma_{h/2}^1} \big( 2\,E_{s,\,\Gamma}(\vect u) : E_{s,\,\Gamma}(\vect v) + \vect u\cdot\vect v + \tau\,(\vect u\cdot\vect n_{\Gamma})\,(\vect v\cdot\vect n_{\Gamma}) \big) \diff{s} \\
	&
		+ \rho_u \int^5_{\Omega_h^{\Gamma}} \frac{\partial \vect u}{\partial\vect n_{\Gamma}}\cdot\frac{\partial \vect v}{\partial\vect n_{\Gamma}} \diff{\vect x}, \quad \vect A \in \mathbb R^{n_{\vect A} \times n_{\vect A}},\\
	\langle \vect B\,\vec{\vect u}, \vec{\vect q} \rangle &= 
		\int^5_{\Gamma_{h/2}^1} \nabla_{\Gamma} q \cdot \vect u \diff{s}, \quad \vect B \in \mathbb R^{n_{\vect S} \times n_{\vect A}},\\
	\langle \vect M_0\,\vec{\vect p}, \vec{\vect q} \rangle &=
		\int^5_{\Gamma_{h/2}^1} p\,q \diff{s}, \quad \vect M_0 \in \mathbb R^{n_{\vect S} \times n_{\vect S}},\\
	\langle \vect C_n\,\vec{\vect p}, \vec{\vect q} \rangle &=
		\rho_p \int^5_{\Omega^{\Gamma}_h} \frac{\partial p}{\partial\vect n_\Gamma} \frac{\partial q}{\partial\vect n_\Gamma} \diff{\vect x}, \quad \vect C_n \in \mathbb R^{n_{\vect S} \times n_{\vect S}}.		 
\end{split}\end{align}

\begin{table}[H]
	\centering\footnotesize
	\caption{Convergence results. $\tau = h^{-2}$, $\rho_u = h$, $\rho_p = h$. Matrices are assembled as in~\eqref{mtx_exact_p2}}
	\label{tab:p2p1_incons_h^1}
	\begin{subtable}{1.\linewidth}\centering
		\begin{tabular}[1.3]{|c||c|c||c|c||c||c|}
			\hline
			\multicolumn{7}{|c|}{$m \in \vect m_{1/2}$ as in Figure~\ref{fig:m}} \\
			\hline
			$h$ & $\|\vect u - \vect u_h\|_{\HOne}$ & Order & $\|\vect u - \vect u_h\|_{\LTwo}$ & Order & $\|p - p_h\|_{\LTwo}$ & Order \\
			\hline
			$8.3\times	10^{-1}$	&	$3.$	&	$\text{}$	&	$1.8$	&	$\text{}$	&	$2.1$	&	$\text{}$	\\ \hline
$4.2\times	10^{-1}$	&	$1.8$	&	$7.4\times	10^{-1}$	&	$9.1\times	10^{-1}$	&	$9.7\times	10^{-1}$	&	$1.7$	&	$2.7\times	10^{-1}$	\\ \hline
$2.1\times	10^{-1}$	&	$7.\times	10^{-1}$	&	$1.4$	&	$3.4\times	10^{-1}$	&	$1.4$	&	$6.9\times	10^{-1}$	&	$1.3$	\\ \hline
$1.\times	10^{-1}$	&	$2.\times	10^{-1}$	&	$1.8$	&	$9.9\times	10^{-2}$	&	$1.8$	&	$2.\times	10^{-1}$	&	$1.8$	\\ \hline
$5.2\times	10^{-2}$	&	$5.2\times	10^{-2}$	&	$1.9$	&	$2.6\times	10^{-2}$	&	$1.9$	&	$5.2\times	10^{-2}$	&	$1.9$	\\ \hline
		\end{tabular}
	\end{subtable}
	\vskip 2mm	
	\begin{subtable}{1.\linewidth}\centering
		\begin{tabular}[1.3]{|c||c|c||c||c|}
			\hline
			$h$ & $\| \vect u_h\cdot\vect n \|_{\LTwo}$ & Order & Outer iterations & Residual norm \\
			\hline
			$8.33\times	10^{-1}$	&	$1.8$	&	$\text{}$	&	$24$	&	$6.2\times	10^{-9}$	\\ \hline
$4.17\times	10^{-1}$	&	$9.2\times	10^{-1}$	&	$9.4\times	10^{-1}$	&	$31$	&	$5.4\times	10^{-9}$	\\ \hline
$2.08\times	10^{-1}$	&	$3.5\times	10^{-1}$	&	$1.4$	&	$30$	&	$9.8\times	10^{-9}$	\\ \hline
$1.04\times	10^{-1}$	&	$9.9\times	10^{-2}$	&	$1.8$	&	$27$	&	$7.8\times	10^{-9}$	\\ \hline
$5.21\times	10^{-2}$	&	$2.6\times	10^{-2}$	&	$1.9$	&	$26$	&	$8.3\times	10^{-9}$	\\ \hline
$2.6\times	10^{-2}$	&	$6.5\times	10^{-3}$	&	$2.$	&	$26$	&	$9.6\times	10^{-9}$	\\ \hline
		\end{tabular}
	\end{subtable}
\end{table}

%\begin{table}[H]
%	\centering\footnotesize
%	\caption{Convergence results. $\tau = h^{-2}$, $\rho_u = h$, $\rho_p = h$. Matrices are assembled as in~\eqref{mtx_exact_2_p2}}
%	\label{tab:p2p1_h}
%	\begin{subtable}{1.\linewidth}\centering
%		\begin{tabular}[1.3]{|c||c|c||c|c||c||c|}
%			\hline
%			\multicolumn{7}{|c|}{$m \in \vect m_{1/2}$ as in Figure~\ref{fig:m}} \\
%			\hline
%			$h$ & $\|\vect u - \vect u_h\|_{\HOne}$ & Order & $\|\vect u - \vect u_h\|_{\LTwo}$ & Order & $\|p - p_h\|_{\LTwo}$ & Order \\
%			\hline
%			$8.3\times	10^{-1}$	&	$3.1$	&	$\text{}$	&	$1.8$	&	$\text{}$	&	$2.2$	&	$\text{}$	\\ \hline
$2.1\times	10^{-1}$	&	$7.\times	10^{-1}$	&	$1.1$	&	$3.5\times	10^{-1}$	&	$1.2$	&	$7.\times	10^{-1}$	&	$8.3\times	10^{-1}$	\\ \hline
$1.\times	10^{-1}$	&	$2.\times	10^{-1}$	&	$1.8$	&	$1.\times	10^{-1}$	&	$1.8$	&	$2.\times	10^{-1}$	&	$1.8$	\\ \hline
$5.2\times	10^{-2}$	&	$5.4\times	10^{-2}$	&	$1.9$	&	$2.6\times	10^{-2}$	&	$1.9$	&	$4.9\times	10^{-2}$	&	$2.$	\\ \hline
%		\end{tabular}
%	\end{subtable}
%	\vskip 2mm	
%	\begin{subtable}{1.\linewidth}\centering
%		\begin{tabular}[1.3]{|c||c|c||c||c|}
%			\hline
%			$h$ & $\| \vect u_h\cdot\vect n \|_{\LTwo}$ & Order & Outer iterations & Residual norm \\
%			\hline
%			$8.33\times	10^{-1}$	&	$1.7$	&	$\text{}$	&	$25$	&	$6.1\times	10^{-9}$	\\ \hline
$2.08\times	10^{-1}$	&	$3.5\times	10^{-1}$	&	$1.2$	&	$30$	&	$8.7\times	10^{-9}$	\\ \hline
$1.04\times	10^{-1}$	&	$1.\times	10^{-1}$	&	$1.8$	&	$27$	&	$6.6\times	10^{-9}$	\\ \hline
$5.21\times	10^{-2}$	&	$2.6\times	10^{-2}$	&	$1.9$	&	$25$	&	$9.\times	10^{-9}$	\\ \hline
%		\end{tabular}
%	\end{subtable}
%\end{table}

%We use the normal stabilization matrix~$\vect C_n$. We choose~$m = O(h^{-1/2})$ so that the geometric error is~$O(h^3)$. Thus $m = 2, 2, 4, 4, 6, 8, 10, 14$ for meshlevel~1, 2 and so forth, respectively. We stick to approach~\eqref{mtx_exact_2}, so we have
%\begin{align}\begin{split}\label{mtx_exact_2_p2}
%	\langle \vect A\,\vec{\vect u}, \vec{\vect v} \rangle &= 
%		\int^5_{\Gamma_{h/m}^1} \big( E_{s,\,\Gamma_{h/m}^1}(\vect u) : E_{s,\,\Gamma_{h/m}^1}(\vect v) + \vect u\cdot\vect v + \tau\,(\vect u\cdot\vect n_{\Gamma})\,(\vect v\cdot\vect n_{\Gamma}) \big) \diff{s} \\
%	&
%		+ \rho_u \int^5_{\Omega_h^{\Gamma}} \frac{\partial \vect u}{\partial\vect n_{\Gamma}}\cdot\frac{\partial \vect v}{\partial\vect n_{\Gamma}} \diff{\vect x}, \quad \vect A \in \mathbb R^{n_{\vect A} \times n_{\vect A}},\\
%	\langle \vect B\,\vec{\vect u}, \vec{\vect q} \rangle &= 
%		-\int^5_{\Gamma_{h/m}^1} q\,\Div_{\Gamma_{h/m}^1} \vect u \diff{s}, \quad \vect B \in \mathbb R^{n_{\vect S} \times n_{\vect A}},\\
%	\langle \vect M_0\,\vec{\vect p}, \vec{\vect q} \rangle &=
%		\int^5_{\Gamma_{h/m}^1} p\,q \diff{s}, \quad \vect M_0 \in \mathbb R^{n_{\vect S} \times n_{\vect S}},\\
%	\langle \vect C_n\,\vec{\vect p}, \vec{\vect q} \rangle &=
%		\rho_p \int_{\Omega^{\Gamma}_h}^5 \frac{\partial p}{\partial\vect n_{\Gamma}} \frac{\partial q}{\partial\vect n_{\Gamma}} \diff{\vect x}, \quad \vect C_n \in \mathbb R^{n_{\vect S} \times n_{\vect S}}.
%\end{split}\end{align}
%
%\begin{table}[h!]
%	\centering\footnotesize
%	\caption{Convergence results. $\tau = h^{-2}$, $\rho_u = \rho_p = h$}
%	\label{tab:p2p1_conv_old}
%	\begin{subtable}{1.\linewidth}\centering
%		\begin{tabular}[1.3]{|c||c|c||c|c||c||c|c|}
%			\hline
%			$h$ & $\|\vect u - \vect u_h\|_{\HOneSpace}$ & Order & $\|\vect u - \vect u_h\|_{\LTwoSpace}$ & Order & $\|p - p_h\|_{\LTwoSpace}$ & Order \\
%			\hline
%			$8.33\times	10^{-1}$	&	$2.8$	&	$\text{}$	&	$1.7$	&	$\text{}$	&	$2.1$	&	$\text{}$	\\ \hline
$4.17\times	10^{-1}$	&	$1.9$	&	$5.74\times	10^{-1}$	&	$9.\times	10^{-1}$	&	$9.15\times	10^{-1}$	&	$1.7$	&	$2.57\times	10^{-1}$	\\ \hline
$2.08\times	10^{-1}$	&	$7.2\times	10^{-1}$	&	$1.37$	&	$3.4\times	10^{-1}$	&	$1.39$	&	$7.1\times	10^{-1}$	&	$1.31$	\\ \hline
$1.04\times	10^{-1}$	&	$2.2\times	10^{-1}$	&	$1.7$	&	$1.\times	10^{-1}$	&	$1.78$	&	$2.1\times	10^{-1}$	&	$1.76$	\\ \hline
$5.21\times	10^{-2}$	&	$6.2\times	10^{-2}$	&	$1.85$	&	$2.6\times	10^{-2}$	&	$1.92$	&	$5.1\times	10^{-2}$	&	$2.04$	\\ \hline
$2.6\times	10^{-2}$	&	$1.8\times	10^{-2}$	&	$1.81$	&	$6.5\times	10^{-3}$	&	$2.01$	&	$1.3\times	10^{-2}$	&	$1.91$	\\ \hline
%		\end{tabular}
%	\end{subtable}
%	\vskip 2mm	
%	\begin{subtable}{1.\linewidth}\centering
%		\begin{tabular}[1.3]{|c||c|c||c||c|}
%			\hline
%			$h$ & $\| \vect u_h\cdot\vect n \|_{\LTwoSpace}$ & Order & Outer iterations & Residual norm \\
%			\hline
%			$8.33\times	10^{-1}$	&	$1.7$	&	$\text{}$	&	$24$	&	$9.9\times	10^{-9}$	\\ \hline
$4.17\times	10^{-1}$	&	$8.9\times	10^{-1}$	&	$8.87\times	10^{-1}$	&	$31$	&	$4.6\times	10^{-9}$	\\ \hline
$2.08\times	10^{-1}$	&	$3.4\times	10^{-1}$	&	$1.38$	&	$31$	&	$4.5\times	10^{-9}$	\\ \hline
$1.04\times	10^{-1}$	&	$1.\times	10^{-1}$	&	$1.78$	&	$29$	&	$3.5\times	10^{-9}$	\\ \hline
$5.21\times	10^{-2}$	&	$2.6\times	10^{-2}$	&	$1.95$	&	$27$	&	$6.9\times	10^{-9}$	\\ \hline
$2.6\times	10^{-2}$	&	$6.5\times	10^{-3}$	&	$1.99$	&	$28$	&	$7.3\times	10^{-9}$	\\ \hline
%		\end{tabular}
%	\end{subtable}
%\end{table}
%
%\begin{table}[h!]
%	\centering\footnotesize
%	\caption{Convergence results. $\tau = h^{-3}$, $\rho_u = \rho_p = h$}
%	\label{tab:p2p1_conv_new}
%	\begin{subtable}{1.\linewidth}\centering
%		\begin{tabular}[1.3]{|c||c|c||c|c||c||c|c|}
%			\hline
%			$h$ & $\|\vect u - \vect u_h\|_{\HOneSpace}$ & Order & $\|\vect u - \vect u_h\|_{\LTwoSpace}$ & Order & $\|p - p_h\|_{\LTwoSpace}$ & Order \\
%			\hline
%			$8.33\times	10^{-1}$	&	$2.7$	&	$\text{}$	&	$1.6$	&	$\text{}$	&	$2.$	&	$\text{}$	\\ \hline
$4.17\times	10^{-1}$	&	$1.2$	&	$1.2$	&	$5.3\times	10^{-1}$	&	$1.61$	&	$1.$	&	$9.82\times	10^{-1}$	\\ \hline
$2.08\times	10^{-1}$	&	$2.4\times	10^{-1}$	&	$2.26$	&	$8.5\times	10^{-2}$	&	$2.66$	&	$1.9\times	10^{-1}$	&	$2.44$	\\ \hline
$1.04\times	10^{-1}$	&	$7.5\times	10^{-2}$	&	$1.68$	&	$1.2\times	10^{-2}$	&	$2.87$	&	$2.8\times	10^{-2}$	&	$2.74$	\\ \hline
$5.21\times	10^{-2}$	&	$3.\times	10^{-2}$	&	$1.32$	&	$5.1\times	10^{-3}$	&	$1.19$	&	$4.6\times	10^{-3}$	&	$2.62$	\\ \hline
$2.6\times	10^{-2}$	&	$1.1\times	10^{-2}$	&	$1.43$	&	$3.\times	10^{-4}$	&	$4.09$	&	$1.4\times	10^{-3}$	&	$1.69$	\\ \hline
$1.3\times	10^{-2}$	&	$4.7\times	10^{-3}$	&	$1.24$	&	$6.6\times	10^{-5}$	&	$2.18$	&	$6.2\times	10^{-4}$	&	$1.18$	\\ \hline
%		\end{tabular}
%	\end{subtable}
%	\vskip 2mm	
%	\begin{subtable}{1.\linewidth}\centering
%		\begin{tabular}[1.3]{|c||c|c||c||c|}
%			\hline
%			$h$ & $\| \vect u_h\cdot\vect n \|_{\LTwoSpace}$ & Order & Outer iterations & Residual norm \\
%			\hline
%			$8.33\times	10^{-1}$	&	$1.6$	&	$\text{}$	&	$25$	&	$4.4\times	10^{-9}$	\\ \hline
$4.17\times	10^{-1}$	&	$5.2\times	10^{-1}$	&	$1.61$	&	$32$	&	$3.6\times	10^{-9}$	\\ \hline
$2.08\times	10^{-1}$	&	$8.4\times	10^{-2}$	&	$2.63$	&	$31$	&	$5.9\times	10^{-9}$	\\ \hline
$1.04\times	10^{-1}$	&	$1.1\times	10^{-2}$	&	$2.91$	&	$28$	&	$9.6\times	10^{-9}$	\\ \hline
$5.21\times	10^{-2}$	&	$1.4\times	10^{-3}$	&	$3.01$	&	$27$	&	$7.4\times	10^{-9}$	\\ \hline
$2.6\times	10^{-2}$	&	$1.7\times	10^{-4}$	&	$3.01$	&	$27$	&	$3.3\times	10^{-9}$	\\ \hline
$1.3\times	10^{-2}$	&	$2.1\times	10^{-5}$	&	$3.$	&	$33$	&	$8.8\times	10^{-9}$	\\ \hline
%		\end{tabular}
%	\end{subtable}
%\end{table}
%
%\begin{table}[h!]
%	\centering\footnotesize
%	\caption{Convergence results. $\tau = 0.1\,h^{-3}$, $\rho_u = \rho_p = h$}
%	\label{tab:p2p1_conv}
%	\begin{subtable}{1.\linewidth}\centering
%		\begin{tabular}[1.3]{|c||c|c||c|c||c||c|c|}
%			\hline
%			$h$ & $\|\vect u - \vect u_h\|_{\HOneSpace}$ & Order & $\|\vect u - \vect u_h\|_{\LTwoSpace}$ & Order & $\|p - p_h\|_{\LTwoSpace}$ & Order \\
%			\hline
%			$8.33\times	10^{-1}$	&	$3.4$	&	$\text{}$	&	$2.1$	&	$\text{}$	&	$2.6$	&	$\text{}$	\\ \hline
$4.17\times	10^{-1}$	&	$3.$	&	$1.96\times	10^{-1}$	&	$1.5$	&	$5.02\times	10^{-1}$	&	$2.9$	&	$-1.44\times	10^{-1}$	\\ \hline
$2.08\times	10^{-1}$	&	$1.2$	&	$1.29$	&	$6.\times	10^{-1}$	&	$1.32$	&	$1.2$	&	$1.25$	\\ \hline
$1.04\times	10^{-1}$	&	$2.3\times	10^{-1}$	&	$2.42$	&	$1.\times	10^{-1}$	&	$2.53$	&	$2.2\times	10^{-1}$	&	$2.5$	\\ \hline
$5.21\times	10^{-2}$	&	$4.2\times	10^{-2}$	&	$2.46$	&	$1.4\times	10^{-2}$	&	$2.85$	&	$2.6\times	10^{-2}$	&	$3.05$	\\ \hline
$2.6\times	10^{-2}$	&	$1.2\times	10^{-2}$	&	$1.83$	&	$1.7\times	10^{-3}$	&	$3.06$	&	$3.9\times	10^{-3}$	&	$2.73$	\\ \hline
$1.3\times	10^{-2}$	&	$4.7\times	10^{-3}$	&	$1.31$	&	$2.2\times	10^{-4}$	&	$2.96$	&	$7.9\times	10^{-4}$	&	$2.31$	\\ \hline
%		\end{tabular}
%	\end{subtable}
%	\vskip 2mm	
%	\begin{subtable}{1.\linewidth}\centering
%		\begin{tabular}[1.3]{|c||c|c||c||c|}
%			\hline
%			$h$ & $\| \vect u_h\cdot\vect n \|_{\LTwoSpace}$ & Order & Outer iterations & Residual norm \\
%			\hline
%			$8.33\times	10^{-1}$	&	$2.1$	&	$\text{}$	&	$25$	&	$5.8\times	10^{-9}$	\\ \hline
$4.17\times	10^{-1}$	&	$1.5$	&	$4.69\times	10^{-1}$	&	$30$	&	$4.8\times	10^{-9}$	\\ \hline
$2.08\times	10^{-1}$	&	$6.\times	10^{-1}$	&	$1.32$	&	$30$	&	$5.6\times	10^{-9}$	\\ \hline
$1.04\times	10^{-1}$	&	$1.\times	10^{-1}$	&	$2.53$	&	$29$	&	$3.5\times	10^{-9}$	\\ \hline
%		\end{tabular}
%	\end{subtable}
%\end{table}

For statistics: using 64 CPUs, computation of the meshlevel~3 ($h = 2.08\times10^{-1}$) takes~${\sim}1$ minute, meshlevel~4 takes~${\sim}7$ minutes, meshlevel~5 takes~${\sim}50$ minutes, meshlevel~6 takes~$4.8$ hours, and meshlevel~7 takes~${\sim}21.3$ hours.

\subsubsection{Consistent penalty formulation}

%We consider the same formulation as in~\eqref{mtx_exact_2_p2}, but with the first term in the definition of~$\vect A$ changed as
%\begin{equation}\label{mtx_exact_2_cons}
%	\int^5_{\Gamma_{h/m}^{2 \to 1}} \big( E_{s,\,\Gamma_{h/m}^{2 \to 1}}(\vect u) - (\vect u\cdot\vect n_{\Gamma_{h/m}^{2 \to 1}})\,\vect H_{\Gamma_h^2} \big) : \big( E_{s,\,\Gamma_{h/m}^{2 \to 1}}(\vect v) - (\vect v\cdot\vect n_{\Gamma_{h/m}^{2 \to 1}})\,\vect H_{\Gamma_h^2} \big) \diff{s},
%\end{equation}
%Shape operator~$\vect H_{\Gamma_h^2}$ is computed as explained in Section~\ref{sec:shape}. Given~$\phi$ in~\eqref{exact_soln}, \eqref{mtx_exact_2_cons} boils down to
%\begin{equation*}
%	\int^5_{\Gamma_{h/m}^1} \big( E_{s,\,\Gamma_{h/m}^1}(\vect u) - (\vect u\cdot\vect n_{\Gamma_{h/m}^1})\,\vect H_{\Gamma} \big) : \big( E_{s,\,\Gamma_{h/m}^1}(\vect v) - (\vect v\cdot\vect n_{\Gamma_{h/m}^1})\,\vect H_{\Gamma} \big) \diff{s}.
%\end{equation*}
%Convergence results for~\eqref{mtx_exact_2_cons} are presented in Tables~\ref{tab:p2p1_cons_h}\,--\,\ref{tab:p2p1_cons_h^-1}.
%
%We also consider the modified version of~\eqref{mtx_exact_2_cons}
%\begin{equation}\label{mtx_exact_2_cons_upd}
%	\int^5_{\Gamma_{h/m}^{2 \to 1}} \big( E_{s,\,\Gamma_{h/m}^{2 \to 1}}(\vect u) - (\vect u\cdot\vect n_{\Gamma_h^2})\,\vect H_{\Gamma_h^2} \big) : \big( E_{s,\,\Gamma_{h/m}^{2 \to 1}}(\vect v) - (\vect v\cdot\vect n_{\Gamma_h^2})\,\vect H_{\Gamma_h^2} \big) \diff{s}.
%\end{equation}
%Convergence results for~\eqref{mtx_exact_2_cons_upd} are presented in Table~\ref{tab:p2p1_cons_upd_h^-1}.
%
%For statistics: using 80 CPUs, computation of the meshlevel~7 ($h = 1.3\times10^{-2}$) takes~${\sim}74.3$ hours with $m = 64$ (computation also involves errors for normals and the shape operator).
We consider the same formulation as in~\eqref{mtx_exact_p2}, but with the first~$\vect A_s$ term in the definition of~$\vect A$ changed according to~\eqref{mtx_cons_exact}. Thus with~\eqref{exact_soln_simpl} we have
\begin{align}\label{mtx_exact_p2_cons}
	\langle \vect A_s\,\vec{\vect u}, \vec{\vect v} \rangle = \int^5_{\Gamma_{h/m}^1} 2\,\big( E_{s,\,\Gamma}(\vect u) - (\vect u\cdot\vect n_{\Gamma})\,\vect H_{\Gamma} \big) : \big( E_{s,\,\Gamma}(\vect v) - (\vect v\cdot\vect n_{\Gamma})\,\vect H_{\Gamma} \big) \diff{s}.
\end{align}

\begin{table}[H]
	\centering\footnotesize
	\caption{Convergence results. $\tau = h^{-2}$, $\rho_u = h^{-1}$, $\rho_p = h$. Matrices are assembled as in~\eqref{mtx_exact_p2}\,--\,\eqref{mtx_exact_p2_cons}}
	\label{tab:p2p1_cons_h^-1}
	\begin{subtable}{1.\linewidth}\centering
		\begin{tabular}[1.3]{|c||c|c||c|c||c||c|}
			\hline
			\multicolumn{7}{|c|}{$m \in \vect m_{1/2}$ as in Figure~\ref{fig:m}} \\
			\hline
			$h$ & $\|\vect u - \vect u_h\|_{\HOne}$ & Order & $\|\vect u - \vect u_h\|_{\LTwo}$ & Order & $\|p - p_h\|_{\LTwo}$ & Order \\
			\hline
			$8.3\times	10^{-1}$	&	$1.2$	&	$\text{}$	&	$4.8\times	10^{-1}$	&	$\text{}$	&	$4.2\times	10^{-1}$	&	$\text{}$	\\ \hline
$4.2\times	10^{-1}$	&	$3.7\times	10^{-1}$	&	$1.8$	&	$6.1\times	10^{-2}$	&	$3.$	&	$1.1\times	10^{-1}$	&	$2.$	\\ \hline
$2.1\times	10^{-1}$	&	$9.2\times	10^{-2}$	&	$2.$	&	$5.8\times	10^{-3}$	&	$3.4$	&	$2.5\times	10^{-2}$	&	$2.1$	\\ \hline
$1.\times	10^{-1}$	&	$2.2\times	10^{-2}$	&	$2.1$	&	$5.6\times	10^{-4}$	&	$3.4$	&	$6.3\times	10^{-3}$	&	$2.$	\\ \hline
$5.2\times	10^{-2}$	&	$5.4\times	10^{-3}$	&	$2.$	&	$5.2\times	10^{-5}$	&	$3.4$	&	$1.7\times	10^{-3}$	&	$1.8$	\\ \hline
		\end{tabular}
	\end{subtable}
	\vskip 2mm	
	\begin{subtable}{1.\linewidth}\centering
		\begin{tabular}[1.3]{|c||c|c||c||c|}
			\hline
			$h$ & $\| \vect u_h\cdot\vect n \|_{\LTwo}$ & Order & Outer iterations & Residual norm \\
			\hline
			$8.33\times	10^{-1}$	&	$3.4\times	10^{-1}$	&	$\text{}$	&	$26$	&	$7.3\times	10^{-9}$	\\ \hline
$4.17\times	10^{-1}$	&	$5.3\times	10^{-2}$	&	$2.7$	&	$33$	&	$4.8\times	10^{-9}$	\\ \hline
$2.08\times	10^{-1}$	&	$4.9\times	10^{-3}$	&	$3.4$	&	$31$	&	$6.\times	10^{-9}$	\\ \hline
$1.04\times	10^{-1}$	&	$5.\times	10^{-4}$	&	$3.3$	&	$27$	&	$8.3\times	10^{-9}$	\\ \hline
$5.21\times	10^{-2}$	&	$4.9\times	10^{-5}$	&	$3.4$	&	$26$	&	$8.6\times	10^{-9}$	\\ \hline
$2.6\times	10^{-2}$	&	$5.\times	10^{-6}$	&	$3.3$	&	$26$	&	$7.5\times	10^{-9}$	\\ \hline
$1.3\times	10^{-2}$	&	$5.9\times	10^{-7}$	&	$3.1$	&	$34$	&	$8.\times	10^{-9}$	\\ \hline
		\end{tabular}
	\end{subtable}
\end{table}

For statistics: using 80 CPUs, computation of the meshlevel~7 ($h = 1.3\times10^{-2}$) takes~${\sim}74.3$ hours with $m = 64$ (computation also involves errors for normals and the shape operator).

%\begin{table}[h!]
%	\centering\footnotesize
%	\caption{Convergence results. $\tau = h^{-2}$, $\rho_u = h^{-1}$, $\rho_p = h$. Matrices are assembled as in~\eqref{mtx_exact_2_cons}}
%	\label{tab:p2p1_cons_h^-1}
%	\begin{subtable}{1.\linewidth}\centering
%		\begin{tabular}[1.3]{|c||c|c||c|c||c||c|}
%			\hline
%			\multicolumn{7}{|c|}{$m \in \vect m_{1/2}$ as in Figure~\ref{fig:m}} \\
%			\hline
%			$h$ & $\|\vect u - \vect u_h\|_{\HOne}$ & Order & $\|\vect u - \vect u_h\|_{\LTwo}$ & Order & $\|p - p_h\|_{\LTwo}$ & Order \\
%			\hline
%			$8.3\times	10^{-1}$	&	$1.5$	&	$\text{}$	&	$6.5\times	10^{-1}$	&	$\text{}$	&	$4.5\times	10^{-1}$	&	$\text{}$	\\ \hline
$2.1\times	10^{-1}$	&	$9.3\times	10^{-2}$	&	$2.$	&	$1.1\times	10^{-2}$	&	$2.9$	&	$2.5\times	10^{-2}$	&	$2.1$	\\ \hline
$1.\times	10^{-1}$	&	$2.2\times	10^{-2}$	&	$2.1$	&	$1.7\times	10^{-3}$	&	$2.7$	&	$6.1\times	10^{-3}$	&	$2.$	\\ \hline
$5.2\times	10^{-2}$	&	$8.1\times	10^{-1}$	&	$-5.2$	&	$7.\times	10^{-1}$	&	$-8.7$	&	$1.2$	&	$-7.6$	\\ \hline
%		\end{tabular}
%	\end{subtable}
%	\vskip 2mm	
%	\begin{subtable}{1.\linewidth}\centering
%		\begin{tabular}[1.3]{|c||c|c||c||c|}
%			\hline
%			$h$ & $\| \vect u_h\cdot\vect n \|_{\LTwo}$ & Order & Outer iterations & Residual norm \\
%			\hline
%			$8.33\times	10^{-1}$	&	$5.7\times	10^{-1}$	&	$\text{}$	&	$25$	&	$6.3\times	10^{-9}$	\\ \hline
$2.08\times	10^{-1}$	&	$1.1\times	10^{-2}$	&	$2.9$	&	$31$	&	$5.5\times	10^{-9}$	\\ \hline
$1.04\times	10^{-1}$	&	$1.7\times	10^{-3}$	&	$2.7$	&	$27$	&	$6.7\times	10^{-9}$	\\ \hline
$5.21\times	10^{-2}$	&	$7.5\times	10^{-4}$	&	$1.2$	&	$34$	&	$6.5\times	10^{-9}$	\\ \hline
%		\end{tabular}
%	\end{subtable}
%	%	\vskip 4mm	
%	%	\begin{subtable}{1.\linewidth}\centering
%	%		\begin{tabular}[1.3]{|c||c|c||c|c||c||c|}
%	%			\hline
%	%			\multicolumn{7}{|c|}{$m \in \vect m_1$ as in Figure~\ref{fig:m}} \\
%	%			\hline
%	%			$h$ & $\|\vect u - \vect u_h\|_{\HOne}$ & Order & $\|\vect u - \vect u_h\|_{\LTwo}$ & Order & $\|p - p_h\|_{\LTwo}$ & Order \\
%	%			\hline
%	%			$4.2\times	10^{-1}$	&	$7.3\times	10^{-1}$	&	$\text{}$	&	$5.\times	10^{-1}$	&	$\text{}$	&	$5.3\times	10^{-1}$	&	$\text{}$	\\ \hline
$2.1\times	10^{-1}$	&	$9.6\times	10^{-2}$	&	$2.9$	&	$1.3\times	10^{-2}$	&	$5.3$	&	$2.9\times	10^{-2}$	&	$4.2$	\\ \hline
$1.\times	10^{-1}$	&	$2.2\times	10^{-2}$	&	$2.1$	&	$2.\times	10^{-3}$	&	$2.7$	&	$8.4\times	10^{-3}$	&	$1.8$	\\ \hline
\rowcolor{LightRed}
$5.2\times	10^{-2}$	&	$8.1\times	10^{-1}$	&	$-5.2$	&	$7.\times	10^{-1}$	&	$-8.4$	&	$1.2$	&	$-7.1$	\\ \hline
$2.6\times	10^{-2}$	&	$1.6\times	10^{-3}$	&	$8.9$	&	$1.2\times	10^{-4}$	&	$1.3\times	10^1$	&	$1.4\times	10^{-3}$	&	$9.7$	\\ \hline
$1.3\times	10^{-2}$	&	$5.6\times	10^{-4}$	&	$1.6$	&	$3.\times	10^{-5}$	&	$2.$	&	$6.4\times	10^{-4}$	&	$1.1$	\\ \hline
%	%		\end{tabular}
%	%	\end{subtable}
%	%	\vskip 2mm	
%	%	\begin{subtable}{1.\linewidth}\centering
%	%		\begin{tabular}[1.3]{|c||c|c||c|c||c||c|}
%	%			\hline
%	%			$h$ & $\| \vect u_h\cdot\vect n \|_{\LTwo}$ & Order & $\| \vect P\,\vect u_h - \vect u \|_{\LTwo}$ & Order & Outer iterations & Residual norm \\
%	%			\hline
%	%			$4.17\times	10^{-1}$	&	$1.4\times	10^{-1}$	&	$\text{}$	&	$4.8\times	10^{-1}$	&	$\text{}$	&	$33$	&	$5.4\times	10^{-9}$	\\ \hline
$2.08\times	10^{-1}$	&	$1.1\times	10^{-2}$	&	$3.7$	&	$6.8\times	10^{-3}$	&	$6.2$	&	$31$	&	$7.2\times	10^{-9}$	\\ \hline
$1.04\times	10^{-1}$	&	$8.7\times	10^{-4}$	&	$3.6$	&	$1.8\times	10^{-3}$	&	$1.9$	&	$28$	&	$9.7\times	10^{-9}$	\\ \hline
$5.21\times	10^{-2}$	&	$7.\times	10^{-4}$	&	$3.2\times	10^{-1}$	&	$7.\times	10^{-1}$	&	$-8.6$	&	$34$	&	$7.4\times	10^{-9}$	\\ \hline
$2.6\times	10^{-2}$	&	$5.8\times	10^{-6}$	&	$6.9$	&	$1.2\times	10^{-4}$	&	$1.3\times	10^1$	&	$27$	&	$4.5\times	10^{-9}$	\\ \hline
$1.3\times	10^{-2}$	&	$6.1\times	10^{-7}$	&	$3.2$	&	$3.\times	10^{-5}$	&	$2.$	&	$34$	&	$9.7\times	10^{-9}$	\\ \hline
%	%		\end{tabular}
%	%	\end{subtable}
%\end{table}

%We consider the same formulations as in~\eqref{mtx_exact}--\eqref{mtx_exact_2}, but with the first term in the definition of~$\vect A$ changed as
%\begin{equation}\label{mtx_exact_2_cons}
%	\int^5_{\Gamma_{h/m}^{2 \to 1}} \big( E_{s,\,\Gamma_{h/m}^{2 \to 1}}(\vect u) - (\vect u\cdot\vect n_{\Gamma_{h/m}^{2 \to 1}})\,\vect H_{\Gamma_h^2} \big) : \big( E_{s,\,\Gamma_{h/m}^{2 \to 1}}(\vect v) - (\vect v\cdot\vect n_{\Gamma_{h/m}^{2 \to 1}})\,\vect H_{\Gamma_h^2} \big) \diff{s},
%\end{equation}
%or 
%\begin{equation}\label{mtx_exact_cons}
%	\int^5_{\Gamma_{h/m}^{2 \to 1}} \big( E_{s,\,\Gamma_h^2}(\vect u) - (\vect u\cdot\vect n_{\Gamma_h^2})\,\vect H_{\Gamma_h^2} \big) : \big( E_{s,\,\Gamma_h^2}(\vect v) - (\vect v\cdot\vect n_{\Gamma_h^2})\,\vect H_{\Gamma_h^2} \big) \diff{s}.
%\end{equation}
%Shape operator is computed as explained in Section~\ref{sec:shape}. Approach~\eqref{mtx_exact_2_cons} corresponds to~\eqref{mtx_exact_2}, and~\eqref{mtx_exact_cons} corresponds to~\eqref{mtx_exact}. Given~$\phi$ in~\eqref{exact_soln}, \eqref{mtx_exact_2_cons} and \eqref{mtx_exact_cons} boil down to
%\begin{equation*}
%	\int^5_{\Gamma_{h/m}^1} \big( E_{s,\,\Gamma_{h/m}^1}(\vect u) - (\vect u\cdot\vect n_{\Gamma_{h/m}^1})\,\vect H_{\Gamma} \big) : \big( E_{s,\,\Gamma_{h/m}^1}(\vect v) - (\vect v\cdot\vect n_{\Gamma_{h/m}^1})\,\vect H_{\Gamma} \big) \diff{s}
%\end{equation*}
%and
%\begin{equation*}
%	\int^5_{\Gamma_{h/m}^1} \big( E_{s,\,\Gamma}(\vect u) - (\vect u\cdot\vect n_{\Gamma})\,\vect H_{\Gamma} \big) : \big( E_{s,\,\Gamma}(\vect v) - (\vect v\cdot\vect n_{\Gamma})\,\vect H_{\Gamma} \big) \diff{s},
%\end{equation*}
%respectively. Convergence results for~\eqref{mtx_exact_2_cons} are presented in Tables~\ref{tab:p2p1_cons_h}\,--\,\ref{tab:p2p1_cons_h^-1}.
%
%We also consider the modified version of~\eqref{mtx_exact_2_cons}
%\begin{equation}\label{mtx_exact_2_cons_upd}
%\int^5_{\Gamma_{h/m}^{2 \to 1}} \big( E_{s,\,\Gamma_{h/m}^{2 \to 1}}(\vect u) - (\vect u\cdot\vect n_{\Gamma_h^2})\,\vect H_{\Gamma_h^2} \big) : \big( E_{s,\,\Gamma_{h/m}^{2 \to 1}}(\vect v) - (\vect v\cdot\vect n_{\Gamma_h^2})\,\vect H_{\Gamma_h^2} \big) \diff{s}.
%\end{equation}
%Convergence results for~\eqref{mtx_exact_2_cons_upd} are presented in Table~\ref{tab:p2p1_cons_upd_h^-1}.

%\begin{table}[h!]
%	\centering\footnotesize
%	\caption{Convergence results. $\tau = h^{-2}$, $\rho_u = \rho_p = h$. Matrices are assembled as in~\eqref{mtx_exact_2_cons}, i.e. both~$\vect n_{\Gamma^2_h}$ and $\vect n_{\Gamma^{2\to 1}_{h/m}}$ are used}
%	\label{tab:p2p1_cons_h}
%	\begin{subtable}{1.\linewidth}\centering
%		\begin{tabular}[1.3]{|c||c|c||c|c||c||c|}
%			\hline
%			\multicolumn{7}{|c|}{$m \in \vect m_{1/2}$ as in Figure~\ref{fig:m}} \\
%			\hline
%			$h$ & $\|\vect u - \vect u_h\|_{\HOne}$ & Order & $\|\vect u - \vect u_h\|_{\LTwo}$ & Order & $\|p - p_h\|_{\LTwo}$ & Order \\
%			\hline
%			$8.3\times	10^{-1}$	&	$1.5$	&	$\text{}$	&	$6.3\times	10^{-1}$	&	$\text{}$	&	$4.5\times	10^{-1}$	&	$\text{}$ \\ \hline
$2.1\times	10^{-1}$	&	$1.5\times	10^{-1}$	&	$1.7$	&	$2.6\times	10^{-2}$	&	$2.3$	&	$3.4\times	10^{-2}$	&	$1.9$ \\ \hline
$1.\times	10^{-1}$	&	$4.8\times	10^{-2}$	&	$1.7$	&	$5.1\times	10^{-3}$	&	$2.3$	&	$9.\times	10^{-3}$	&	$1.9$ \\ \hline
\rowcolor{LightRed}
$5.2\times	10^{-2}$	&	$2.2\times	10^{-2}$	&	$1.1$	&	$4.7\times	10^{-3}$	&	$1.2\times	10^{-1}$	&	$4.8\times	10^{-3}$	&	$0.9$ \\ \hline
$2.6\times	10^{-2}$	&	$7.3\times	10^{-3}$	&	$1.6$	&	$2.1\times	10^{-4}$	&	$4.5$	&	$1.3\times	10^{-3}$	&	$1.8$ \\ \hline
$1.3\times	10^{-2}$	&	$3.2\times	10^{-3}$	&	$1.2$	&	$4.\times	10^{-5}$	&	$2.4$	&	$6.3\times	10^{-4}$	&	$1.1$ \\ \hline
%		\end{tabular}
%	\end{subtable}
%	\vskip 2mm	
%	\begin{subtable}{1.\linewidth}\centering
%		\begin{tabular}[1.3]{|c||c|c||c|c||c||c|}
%			\hline
%			$h$ & $\| \vect u_h\cdot\vect n \|_{\LTwo}$ & Order & $\| \vect P\,\vect u_h - \vect u \|_{\LTwo}$ & Order & Outer iterations & Residual norm \\
%			\hline
%			$8.33\times	10^{-1}$	&	$5.6\times	10^{-1}$	&	$\text{}$	&	$2.9\times	10^{-1}$	&	$\text{}$	&	$25$	&	$6.1\times	10^{-9}$	\\ \hline
$2.08\times	10^{-1}$	&	$2.4\times	10^{-2}$	&	$2.3$	&	$7.4\times	10^{-3}$	&	$2.6$	&	$31$	&	$7.\times	10^{-9}$	\\ \hline
$1.04\times	10^{-1}$	&	$4.8\times	10^{-3}$	&	$2.3$	&	$1.8\times	10^{-3}$	&	$2.$	&	$28$	&	$9.6\times	10^{-9}$	\\ \hline
$5.21\times	10^{-2}$	&	$1.1\times	10^{-3}$	&	$2.1$	&	$4.6\times	10^{-3}$	&	$-1.3$	&	$27$	&	$7.6\times	10^{-9}$	\\ \hline
$2.6\times	10^{-2}$	&	$1.8\times	10^{-4}$	&	$2.7$	&	$1.2\times	10^{-4}$	&	$5.3$	&	$28$	&	$7.6\times	10^{-9}$	\\ \hline
$1.3\times	10^{-2}$	&	$2.6\times	10^{-5}$	&	$2.8$	&	$3.\times	10^{-5}$	&	$2.$	&	$35$	&	$9.3\times	10^{-9}$	\\ \hline
%		\end{tabular}
%	\end{subtable}
%	\vskip 4mm	
%	\begin{subtable}{1.\linewidth}\centering
%		\begin{tabular}[1.3]{|c||c|c||c|c||c||c|}
%			\hline
%			\multicolumn{7}{|c|}{$m \in \vect m_1$ as in Figure~\ref{fig:m}} \\
%			\hline
%			$h$ & $\|\vect u - \vect u_h\|_{\HOne}$ & Order & $\|\vect u - \vect u_h\|_{\LTwo}$ & Order & $\|p - p_h\|_{\LTwo}$ & Order \\
%			\hline
%			$4.2\times	10^{-1}$	&	$6.6\times	10^{-1}$	&	$\text{}$	&	$2.6\times	10^{-1}$	&	$\text{}$	&	$1.7\times	10^{-1}$	&	$\text{}$ \\ \hline
$2.1\times	10^{-1}$	&	$1.5\times	10^{-1}$	&	$2.1$	&	$2.6\times	10^{-2}$	&	$3.3$	&	$3.4\times	10^{-2}$	&	$2.4$ \\ \hline
$1.\times	10^{-1}$	&	$4.5\times	10^{-2}$	&	$1.8$	&	$3.2\times	10^{-3}$	&	$3.$	&	$8.6\times	10^{-3}$	&	$2.$ \\ \hline
\rowcolor{LightRed}
$5.2\times	10^{-2}$	&	$2.\times	10^{-2}$	&	$1.1$	&	$4.6\times	10^{-3}$	&	$-0.54$	&	$4.8\times	10^{-3}$	&	$8.4\times	10^{-1}$ \\ \hline
%		\end{tabular}
%	\end{subtable}
%	\vskip 2mm	
%	\begin{subtable}{1.\linewidth}\centering
%		\begin{tabular}[1.3]{|c||c|c||c|c||c||c|}
%			\hline
%			$h$ & $\| \vect u_h\cdot\vect n \|_{\LTwo}$ & Order & $\| \vect P\,\vect u_h - \vect u \|_{\LTwo}$ & Order & Outer iterations & Residual norm \\
%			\hline
%			$4.17\times	10^{-1}$	&	$2.2\times	10^{-1}$	&	$\text{}$	&	$1.4\times	10^{-1}$	&	$\text{}$	&	$33$	&	$3.3\times	10^{-9}$	\\ \hline
$2.08\times	10^{-1}$	&	$2.4\times	10^{-2}$	&	$3.2$	&	$7.4\times	10^{-3}$	&	$4.2$	&	$31$	&	$7.\times	10^{-9}$	\\ \hline
$1.04\times	10^{-1}$	&	$2.6\times	10^{-3}$	&	$3.2$	&	$1.9\times	10^{-3}$	&	$2.$	&	$28$	&	$9.6\times	10^{-9}$	\\ \hline
$5.21\times	10^{-2}$	&	$2.6\times	10^{-4}$	&	$3.3$	&	$4.6\times	10^{-3}$	&	$-1.3$	&	$27$	&	$7.9\times	10^{-9}$	\\ \hline
%		\end{tabular}
%	\end{subtable}
%\end{table}

%\begin{table}[h!]
%	\centering\footnotesize
%	\caption{Convergence results. $\tau = h^{-2}$, $\rho_u = h^{-1}$, $\rho_p = h$. Matrices are assembled as in~\eqref{mtx_exact_2_cons}}
%	\label{tab:p2p1_cons_h^-1}
%	\begin{subtable}{1.\linewidth}\centering
%		\begin{tabular}[1.3]{|c||c|c||c|c||c||c|}
%			\hline
%			\multicolumn{7}{|c|}{$m \in \vect m_{1/2}$ as in Figure~\ref{fig:m}} \\
%			\hline
%			$h$ & $\|\vect u - \vect u_h\|_{\HOne}$ & Order & $\|\vect u - \vect u_h\|_{\LTwo}$ & Order & $\|p - p_h\|_{\LTwo}$ & Order \\
%			\hline
%			$8.3\times	10^{-1}$	&	$1.5$	&	$\text{}$	&	$6.5\times	10^{-1}$	&	$\text{}$	&	$4.5\times	10^{-1}$	&	$\text{}$	\\ \hline
$2.1\times	10^{-1}$	&	$9.3\times	10^{-2}$	&	$2.$	&	$1.1\times	10^{-2}$	&	$2.9$	&	$2.5\times	10^{-2}$	&	$2.1$	\\ \hline
$1.\times	10^{-1}$	&	$2.2\times	10^{-2}$	&	$2.1$	&	$1.7\times	10^{-3}$	&	$2.7$	&	$6.1\times	10^{-3}$	&	$2.$	\\ \hline
$5.2\times	10^{-2}$	&	$8.1\times	10^{-1}$	&	$-5.2$	&	$7.\times	10^{-1}$	&	$-8.7$	&	$1.2$	&	$-7.6$	\\ \hline
%		\end{tabular}
%	\end{subtable}
%	\vskip 2mm	
%	\begin{subtable}{1.\linewidth}\centering
%		\begin{tabular}[1.3]{|c||c|c||c||c|}
%			\hline
%			$h$ & $\| \vect u_h\cdot\vect n \|_{\LTwo}$ & Order & Outer iterations & Residual norm \\
%			\hline
%			$8.33\times	10^{-1}$	&	$5.7\times	10^{-1}$	&	$\text{}$	&	$25$	&	$6.3\times	10^{-9}$	\\ \hline
$2.08\times	10^{-1}$	&	$1.1\times	10^{-2}$	&	$2.9$	&	$31$	&	$5.5\times	10^{-9}$	\\ \hline
$1.04\times	10^{-1}$	&	$1.7\times	10^{-3}$	&	$2.7$	&	$27$	&	$6.7\times	10^{-9}$	\\ \hline
$5.21\times	10^{-2}$	&	$7.5\times	10^{-4}$	&	$1.2$	&	$34$	&	$6.5\times	10^{-9}$	\\ \hline
%		\end{tabular}
%	\end{subtable}
%%	\vskip 4mm	
%%	\begin{subtable}{1.\linewidth}\centering
%%		\begin{tabular}[1.3]{|c||c|c||c|c||c||c|}
%%			\hline
%%			\multicolumn{7}{|c|}{$m \in \vect m_1$ as in Figure~\ref{fig:m}} \\
%%			\hline
%%			$h$ & $\|\vect u - \vect u_h\|_{\HOne}$ & Order & $\|\vect u - \vect u_h\|_{\LTwo}$ & Order & $\|p - p_h\|_{\LTwo}$ & Order \\
%%			\hline
%%			$4.2\times	10^{-1}$	&	$7.3\times	10^{-1}$	&	$\text{}$	&	$5.\times	10^{-1}$	&	$\text{}$	&	$5.3\times	10^{-1}$	&	$\text{}$	\\ \hline
$2.1\times	10^{-1}$	&	$9.6\times	10^{-2}$	&	$2.9$	&	$1.3\times	10^{-2}$	&	$5.3$	&	$2.9\times	10^{-2}$	&	$4.2$	\\ \hline
$1.\times	10^{-1}$	&	$2.2\times	10^{-2}$	&	$2.1$	&	$2.\times	10^{-3}$	&	$2.7$	&	$8.4\times	10^{-3}$	&	$1.8$	\\ \hline
\rowcolor{LightRed}
$5.2\times	10^{-2}$	&	$8.1\times	10^{-1}$	&	$-5.2$	&	$7.\times	10^{-1}$	&	$-8.4$	&	$1.2$	&	$-7.1$	\\ \hline
$2.6\times	10^{-2}$	&	$1.6\times	10^{-3}$	&	$8.9$	&	$1.2\times	10^{-4}$	&	$1.3\times	10^1$	&	$1.4\times	10^{-3}$	&	$9.7$	\\ \hline
$1.3\times	10^{-2}$	&	$5.6\times	10^{-4}$	&	$1.6$	&	$3.\times	10^{-5}$	&	$2.$	&	$6.4\times	10^{-4}$	&	$1.1$	\\ \hline
%%		\end{tabular}
%%	\end{subtable}
%%	\vskip 2mm	
%%	\begin{subtable}{1.\linewidth}\centering
%%		\begin{tabular}[1.3]{|c||c|c||c|c||c||c|}
%%			\hline
%%			$h$ & $\| \vect u_h\cdot\vect n \|_{\LTwo}$ & Order & $\| \vect P\,\vect u_h - \vect u \|_{\LTwo}$ & Order & Outer iterations & Residual norm \\
%%			\hline
%%			$4.17\times	10^{-1}$	&	$1.4\times	10^{-1}$	&	$\text{}$	&	$4.8\times	10^{-1}$	&	$\text{}$	&	$33$	&	$5.4\times	10^{-9}$	\\ \hline
$2.08\times	10^{-1}$	&	$1.1\times	10^{-2}$	&	$3.7$	&	$6.8\times	10^{-3}$	&	$6.2$	&	$31$	&	$7.2\times	10^{-9}$	\\ \hline
$1.04\times	10^{-1}$	&	$8.7\times	10^{-4}$	&	$3.6$	&	$1.8\times	10^{-3}$	&	$1.9$	&	$28$	&	$9.7\times	10^{-9}$	\\ \hline
$5.21\times	10^{-2}$	&	$7.\times	10^{-4}$	&	$3.2\times	10^{-1}$	&	$7.\times	10^{-1}$	&	$-8.6$	&	$34$	&	$7.4\times	10^{-9}$	\\ \hline
$2.6\times	10^{-2}$	&	$5.8\times	10^{-6}$	&	$6.9$	&	$1.2\times	10^{-4}$	&	$1.3\times	10^1$	&	$27$	&	$4.5\times	10^{-9}$	\\ \hline
$1.3\times	10^{-2}$	&	$6.1\times	10^{-7}$	&	$3.2$	&	$3.\times	10^{-5}$	&	$2.$	&	$34$	&	$9.7\times	10^{-9}$	\\ \hline
%%		\end{tabular}
%%	\end{subtable}
%\end{table}
%
%\begin{table}[h!]
%	\centering\footnotesize
%	\caption{Convergence results for shifted sphere, $\sphere + 0.3\,\vect s$, $\vect s = (1, 1, 1)^T / \sqrt{3}$. $\tau = h^{-2}$, $\rho_u = h^{-1}$, $\rho_p = h$. Matrices are assembled as in~\eqref{mtx_exact_2_cons}}
%	\label{tab:p2p1_cons_upd_h^-1_shift}
%	\begin{subtable}{1.\linewidth}\centering
%		\begin{tabular}[1.3]{|c||c|c||c|c||c||c|}
%			\hline
%			\multicolumn{7}{|c|}{$m \in \vect m_{1/2}$ as in Figure~\ref{fig:m}} \\
%			\hline
%			$h$ & $\|\vect u - \vect u_h\|_{\HOne}$ & Order & $\|\vect u - \vect u_h\|_{\LTwo}$ & Order & $\|p - p_h\|_{\LTwo}$ & Order \\
%			\hline
%			$8.3\times	10^{-1}$	&	$1.5$	&	$\text{}$	&	$6.1\times	10^{-1}$	&	$\text{}$	&	$4.3\times	10^{-1}$	&	$\text{}$	\\ \hline
$2.1\times	10^{-1}$	&	$9.4\times	10^{-2}$	&	$2.$	&	$1.1\times	10^{-2}$	&	$2.9$	&	$2.5\times	10^{-2}$	&	$2.$	\\ \hline
$1.\times	10^{-1}$	&	$2.2\times	10^{-2}$	&	$2.1$	&	$1.6\times	10^{-3}$	&	$2.8$	&	$6.\times	10^{-3}$	&	$2.$	\\ \hline
$5.2\times	10^{-2}$	&	$5.5\times	10^{-3}$	&	$2.$	&	$2.7\times	10^{-4}$	&	$2.6$	&	$1.6\times	10^{-3}$	&	$1.9$	\\ \hline
$2.6\times	10^{-2}$	&	$1.3\times	10^{-3}$	&	$2.$	&	$3.3\times	10^{-5}$	&	$3.1$	&	$4.1\times	10^{-4}$	&	$2.$	\\ \hline
%		\end{tabular}
%	\end{subtable}
%	\vskip 2mm	
%	\begin{subtable}{1.\linewidth}\centering
%		\begin{tabular}[1.3]{|c||c|c||c||c|}
%			\hline
%			$h$ & $\| \vect u_h\cdot\vect n \|_{\LTwo}$ & Order & Outer iterations & Residual norm \\
%			\hline
%			$8.33\times	10^{-1}$	&	$5.5\times	10^{-1}$	&	$\text{}$	&	$27$	&	$4.2\times	10^{-9}$	\\ \hline
$2.08\times	10^{-1}$	&	$1.1\times	10^{-2}$	&	$2.8$	&	$31$	&	$6.1\times	10^{-9}$	\\ \hline
$1.04\times	10^{-1}$	&	$1.6\times	10^{-3}$	&	$2.7$	&	$27$	&	$7.\times	10^{-9}$	\\ \hline
$5.21\times	10^{-2}$	&	$2.7\times	10^{-4}$	&	$2.6$	&	$25$	&	$5.6\times	10^{-9}$	\\ \hline
$2.6\times	10^{-2}$	&	$3.3\times	10^{-5}$	&	$3.1$	&	$26$	&	$4.3\times	10^{-9}$	\\ \hline
%		\end{tabular}
%	\end{subtable}
%	\vskip 2mm	
%	\begin{subtable}{1.\linewidth}\centering
%		\begin{tabular}[1.3]{|c||c|c||c|c||c||c|}
%			\hline
%			\multicolumn{7}{|c|}{$m \in \vect m_1$ as in Figure~\ref{fig:m}} \\
%			\hline
%			$h$ & $\|\vect u - \vect u_h\|_{\HOne}$ & Order & $\|\vect u - \vect u_h\|_{\LTwo}$ & Order & $\|p - p_h\|_{\LTwo}$ & Order \\
%			\hline
%			$4.2\times	10^{-1}$	&	$4.7\times	10^{-1}$	&	$\text{}$	&	$1.3\times	10^{-1}$	&	$\text{}$	&	$1.3\times	10^{-1}$	&	$\text{}$	\\ \hline
$2.1\times	10^{-1}$	&	$9.4\times	10^{-2}$	&	$2.3$	&	$1.1\times	10^{-2}$	&	$3.5$	&	$2.5\times	10^{-2}$	&	$2.4$	\\ \hline
$1.\times	10^{-1}$	&	$2.2\times	10^{-2}$	&	$2.1$	&	$8.6\times	10^{-4}$	&	$3.7$	&	$6.\times	10^{-3}$	&	$2.1$	\\ \hline
$5.2\times	10^{-2}$	&	$5.4\times	10^{-3}$	&	$2.$	&	$6.8\times	10^{-5}$	&	$3.7$	&	$1.6\times	10^{-3}$	&	$1.9$	\\ \hline
$2.6\times	10^{-2}$	&	$1.3\times	10^{-3}$	&	$2.$	&	$5.9\times	10^{-6}$	&	$3.5$	&	$4.1\times	10^{-4}$	&	$2.$	\\ \hline
%		\end{tabular}
%	\end{subtable}
%	\vskip 2mm	
%	\begin{subtable}{1.\linewidth}\centering
%		\begin{tabular}[1.3]{|c||c|c||c||c|}
%			\hline
%			$h$ & $\| \vect u_h\cdot\vect n \|_{\LTwo}$ & Order & Outer iterations & Residual norm \\
%			\hline
%			$4.17\times	10^{-1}$	&	$1.3\times	10^{-1}$	&	$\text{}$	&	$33$	&	$3.9\times	10^{-9}$	\\ \hline
$2.08\times	10^{-1}$	&	$1.1\times	10^{-2}$	&	$3.5$	&	$31$	&	$6.1\times	10^{-9}$	\\ \hline
$1.04\times	10^{-1}$	&	$8.5\times	10^{-4}$	&	$3.7$	&	$27$	&	$7.\times	10^{-9}$	\\ \hline
$5.21\times	10^{-2}$	&	$6.7\times	10^{-5}$	&	$3.7$	&	$25$	&	$5.6\times	10^{-9}$	\\ \hline
$2.6\times	10^{-2}$	&	$5.8\times	10^{-6}$	&	$3.5$	&	$26$	&	$4.3\times	10^{-9}$	\\ \hline
%		\end{tabular}
%	\end{subtable}
%\end{table}

%\begin{table}[h!]
%	\centering\footnotesize
%	\caption{Convergence results. $\tau = h^{-2}$, $\rho_u = h^{-1}$, $\rho_p = h$. Matrices are assembled as in~\eqref{mtx_exact_2_cons_upd}}
%	\label{tab:p2p1_cons_upd_h^-1}
%	\begin{subtable}{1.\linewidth}\centering
%		\begin{tabular}[1.3]{|c||c|c||c|c||c||c|}
%			\hline
%			\multicolumn{7}{|c|}{$m \in \vect m_1$ as in Figure~\ref{fig:m}} \\
%			\hline
%			$h$ & $\|\vect u - \vect u_h\|_{\HOne}$ & Order & $\|\vect u - \vect u_h\|_{\LTwo}$ & Order & $\|p - p_h\|_{\LTwo}$ & Order \\
%			\hline
%			$4.2\times	10^{-1}$	&	$7.4\times	10^{-1}$	&	$\text{}$	&	$5.\times	10^{-1}$	&	$\text{}$	&	$5.2\times	10^{-1}$	&	$\text{}$	\\ \hline
$2.1\times	10^{-1}$	&	$9.7\times	10^{-2}$	&	$2.9$	&	$1.3\times	10^{-2}$	&	$5.3$	&	$2.9\times	10^{-2}$	&	$4.2$	\\ \hline
$1.\times	10^{-1}$	&	$2.2\times	10^{-2}$	&	$2.1$	&	$2.\times	10^{-3}$	&	$2.7$	&	$8.4\times	10^{-3}$	&	$1.8$	\\ \hline
\rowcolor{LightRed}
$5.2\times	10^{-2}$	&	$8.1\times	10^{-1}$	&	$-5.2$	&	$7.\times	10^{-1}$	&	$-8.4$	&	$1.2$	&	$-7.1$	\\ \hline
%		\end{tabular}
%	\end{subtable}
%	\vskip 2mm	
%	\begin{subtable}{1.\linewidth}\centering
%		\begin{tabular}[1.3]{|c||c|c||c|c||c||c|}
%			\hline
%			$h$ & $\| \vect u_h\cdot\vect n \|_{\LTwo}$ & Order & $\| \vect P\,\vect u_h - \vect u \|_{\LTwo}$ & Order & Outer iterations & Residual norm \\
%			\hline
%			$4.17\times	10^{-1}$	&	$1.4\times	10^{-1}$	&	$\text{}$	&	$4.8\times	10^{-1}$	&	$\text{}$	&	$33$	&	$5.4\times	10^{-9}$	\\ \hline
$2.08\times	10^{-1}$	&	$1.1\times	10^{-2}$	&	$3.7$	&	$6.7\times	10^{-3}$	&	$6.2$	&	$31$	&	$7.2\times	10^{-9}$	\\ \hline
$1.04\times	10^{-1}$	&	$8.8\times	10^{-4}$	&	$3.6$	&	$1.8\times	10^{-3}$	&	$1.9$	&	$28$	&	$9.7\times	10^{-9}$	\\ \hline
$5.21\times	10^{-2}$	&	$7.\times	10^{-4}$	&	$3.3\times	10^{-1}$	&	$7.\times	10^{-1}$	&	$-8.6$	&	$34$	&	$7.4\times	10^{-9}$	\\ \hline
%		\end{tabular}
%	\end{subtable}
%\end{table}

%\begin{table}[h!]
%	\centering\footnotesize
%	\caption{Convergence results. $\tau = h^{-2}$, $\rho_u = \rho_p = h$, $m$ as in Figure~\ref{fig:m} (Left). Matrices are assembled as in~\eqref{mtx_exact_2}}
%	\label{tab:p2p1_conv_cons}
%	\begin{subtable}{1.\linewidth}\centering
%		\begin{tabular}[1.3]{|c||c|c||c|c||c||c||c|c|c|}
%			\hline
%			$h$ & $\|\vect u - \vect u_h\|_{\HOne}$ & Order & $\|\vect u - \vect u_h\|_{\LTwo}$ & Order & $\|p - p_h\|_{\LTwo}$ & Order & $\|\vect n_\Gamma - \vect n_{\Gamma_{h}^2}\|_{\LTwo}$ & $\|\vect n_\Gamma - \vect n_{\Gamma_{h/m}^1} \|_{\LTwo}$ & Order \\
%			\hline
%			$8.3\times	10^{-1}$	&	$1.5$	&	$\text{}$	&	$6.3\times	10^{-1}$	&	$\text{}$	&	$4.5\times	10^{-1}$	&	$\text{}$	&	$8.\times	10^{-16}$	&	$6.3\times	10^{-1}$	&	$\text{}$	\\ \hline
$4.2\times	10^{-1}$	&	$6.6\times	10^{-1}$	&	$1.2$	&	$2.6\times	10^{-1}$	&	$1.3$	&	$1.7\times	10^{-1}$	&	$1.3$	&	$1.1\times	10^{-15}$	&	$2.9\times	10^{-1}$	&	$1.1$	\\ \hline
$2.1\times	10^{-1}$	&	$1.5\times	10^{-1}$	&	$2.1$	&	$2.6\times	10^{-2}$	&	$3.3$	&	$3.4\times	10^{-2}$	&	$2.4$	&	$2.3\times	10^{-15}$	&	$7.6\times	10^{-2}$	&	$2.$	\\ \hline
\rowcolor{LightRed}
$1.\times	10^{-1}$	&	$6.6\times	10^{-2}$	&	$1.2$	&	$1.2\times	10^{-2}$	&	$1.1$	&	$1.3\times	10^{-2}$	&	$1.4$	&	$5.2\times	10^{-15}$	&	$3.8\times	10^{-2}$	&	$1.$	\\ \hline
$5.2\times	10^{-2}$	&	$2.4\times	10^{-2}$	&	$1.5$	&	$5.\times	10^{-3}$	&	$1.2$	&	$4.6\times	10^{-3}$	&	$1.5$	&	$9.3\times	10^{-15}$	&	$1.3\times	10^{-2}$	&	$1.6$	\\ \hline
%		\end{tabular}
%	\end{subtable}
%	\vskip 4mm	
%	\begin{subtable}{1.\linewidth}\centering
%		\begin{tabular}[1.3]{|c||c|c||c|c||c||c|}
%			\hline
%			$h$ & $\| \vect u_h\cdot\vect n \|_{\LTwo}$ & Order & $\| \vect P\,\vect u_h - \vect u \|_{\LTwo}$ & Order & Outer iterations & Residual norm \\
%			\hline
%			$8.33\times	10^{-1}$	&	$5.6\times	10^{-1}$	&	$\text{}$	&	$2.9\times	10^{-1}$	&	$\text{}$	&	$25$	&	$6.1\times	10^{-9}$	\\ \hline
$4.17\times	10^{-1}$	&	$2.2\times	10^{-1}$	&	$1.3$	&	$1.4\times	10^{-1}$	&	$1.1$	&	$33$	&	$3.3\times	10^{-9}$	\\ \hline
$2.08\times	10^{-1}$	&	$2.4\times	10^{-2}$	&	$3.2$	&	$7.4\times	10^{-3}$	&	$4.2$	&	$31$	&	$7.\times	10^{-9}$	\\ \hline
$1.04\times	10^{-1}$	&	$1.2\times	10^{-2}$	&	$1.1$	&	$1.8\times	10^{-3}$	&	$2.$	&	$29$	&	$4.3\times	10^{-9}$	\\ \hline
$5.21\times	10^{-2}$	&	$2.\times	10^{-3}$	&	$2.5$	&	$4.6\times	10^{-3}$	&	$-1.3$	&	$27$	&	$7.3\times	10^{-9}$	\\ \hline
%		\end{tabular}
%	\end{subtable}
%\end{table}

%\begin{table}[h!]
%	\centering\footnotesize
%	\caption{Convergence results. Same as Table~\ref{tab:p2p1_conv_cons}, but with $m$ doubled, i.e. $\Gamma_{h/2m}^1$ is used instead of $\Gamma_{h/m}^1$}
%	\label{tab:p2p1_conv_cons_ref}
%	\begin{subtable}{1.\linewidth}\centering
%		\begin{tabular}[1.3]{|c||c|c||c|c||c||c||c|c|c|}
%			\hline
%			$h$ & $\|\vect u - \vect u_h\|_{\HOne}$ & Order & $\|\vect u - \vect u_h\|_{\LTwo}$ & Order & $\|p - p_h\|_{\LTwo}$ & Order & $\|\vect n_\Gamma - \vect n_{\Gamma_{h}^2}\|_{\LTwo}$ & $\|\vect n_\Gamma - \vect n_{\Gamma_{h/m}^1} \|_{\LTwo}$ & Order \\
%			\hline
%			$8.3\times	10^{-1}$	&	$1.2$	&	$\text{}$	&	$5.3\times	10^{-1}$	&	$\text{}$	&	$4.4\times	10^{-1}$	&	$\text{}$	&	$8.4\times	10^{-16}$	&	$2.9\times	10^{-1}$	&	$\text{}$	\\ \hline
$4.2\times	10^{-1}$	&	$4.4\times	10^{-1}$	&	$1.5$	&	$1.6\times	10^{-1}$	&	$1.7$	&	$1.3\times	10^{-1}$	&	$1.7$	&	$1.1\times	10^{-15}$	&	$1.5\times	10^{-1}$	&	$9.6\times	10^{-1}$	\\ \hline
$2.1\times	10^{-1}$	&	$1.3\times	10^{-1}$	&	$1.7$	&	$1.2\times	10^{-2}$	&	$3.8$	&	$3.\times	10^{-2}$	&	$2.2$	&	$2.3\times	10^{-15}$	&	$3.8\times	10^{-2}$	&	$2.$	\\ \hline
\rowcolor{LightRed}
$1.\times	10^{-1}$	&	$4.5\times	10^{-2}$	&	$1.6$	&	$3.2\times	10^{-3}$	&	$1.9$	&	$8.6\times	10^{-3}$	&	$1.8$	&	$5.2\times	10^{-15}$	&	$1.9\times	10^{-2}$	&	$1.$	\\ \hline
%		\end{tabular}
%	\end{subtable}
%	\vskip 4mm	
%	\begin{subtable}{1.\linewidth}\centering
%		\begin{tabular}[1.3]{|c||c|c||c|c||c||c|}
%			\hline
%			$h$ & $\| \vect u_h\cdot\vect n \|_{\LTwo}$ & Order & $\| \vect P\,\vect u_h - \vect u \|_{\LTwo}$ & Order & Outer iterations & Residual norm \\
%			\hline
%			$8.33\times	10^{-1}$	&	$4.2\times	10^{-1}$	&	$\text{}$	&	$3.3\times	10^{-1}$	&	$\text{}$	&	$25$	&	$9.1\times	10^{-9}$	\\ \hline
$4.17\times	10^{-1}$	&	$6.2\times	10^{-2}$	&	$2.8$	&	$1.5\times	10^{-1}$	&	$1.1$	&	$33$	&	$4.4\times	10^{-9}$	\\ \hline
$2.08\times	10^{-1}$	&	$8.5\times	10^{-3}$	&	$2.9$	&	$8.3\times	10^{-3}$	&	$4.2$	&	$31$	&	$7.7\times	10^{-9}$	\\ \hline
$1.04\times	10^{-1}$	&	$2.6\times	10^{-3}$	&	$1.7$	&	$1.9\times	10^{-3}$	&	$2.1$	&	$28$	&	$9.6\times	10^{-9}$	\\ \hline
%		\end{tabular}
%	\end{subtable}
%\end{table}

%\begin{table}[h!]
%	\centering\footnotesize
%	\caption{Convergence results. $\tau = h^{-2}$, $\rho_u = \rho_p = h$, $m$ as in Figure~\ref{fig:m} (Left), Matrices are assembled as in~\eqref{mtx_exact}, i.e. $I^2_h(\phi)$ is used to define all the normal vectors}
%	\label{tab:p2p1_conv_cons_p2}
%	\begin{subtable}{1.\linewidth}\centering
%		\begin{tabular}[1.3]{|c||c|c||c|c||c||c||c|c|c|}
%			\hline
%			$h$ & $\|\vect u - \vect u_h\|_{\HOne}$ & Order & $\|\vect u - \vect u_h\|_{\LTwo}$ & Order & $\|p - p_h\|_{\LTwo}$ & Order & $\|\vect n_\Gamma - \vect n_{\Gamma_{h}^2}\|_{\LTwo}$ & $\|\vect n_\Gamma - \vect n_{\Gamma_{h/m}^1} \|_{\LTwo}$ & Order \\
%			\hline
%			$8.3\times	10^{-1}$	&	$1.3$	&	$\text{}$	&	$4.4\times	10^{-1}$	&	$\text{}$	&	$4.1\times	10^{-1}$	&	$\text{}$	&	$8.1\times	10^{-16}$	&	$6.4\times	10^{-1}$	&	$\text{}$	\\ \hline
$4.2\times	10^{-1}$	&	$4.8\times	10^{-1}$	&	$1.4$	&	$1.4\times	10^{-1}$	&	$1.7$	&	$1.7\times	10^{-1}$	&	$1.2$	&	$1.1\times	10^{-15}$	&	$2.9\times	10^{-1}$	&	$1.1$	\\ \hline
$2.1\times	10^{-1}$	&	$1.2\times	10^{-1}$	&	$1.9$	&	$1.1\times	10^{-2}$	&	$3.7$	&	$2.8\times	10^{-2}$	&	$2.6$	&	$2.3\times	10^{-15}$	&	$7.6\times	10^{-2}$	&	$2.$	\\ \hline
\rowcolor{LightRed}
$1.\times	10^{-1}$	&	$4.\times	10^{-2}$	&	$1.6$	&	$1.8\times	10^{-3}$	&	$2.6$	&	$8.8\times	10^{-3}$	&	$1.7$	&	$5.2\times	10^{-15}$	&	$3.8\times	10^{-2}$	&	$1.$	\\ \hline
%		\end{tabular}
%	\end{subtable}
%	\vskip 4mm	
%	\begin{subtable}{1.\linewidth}\centering
%		\begin{tabular}[1.3]{|c||c|c||c|c||c||c|}
%			\hline
%			$h$ & $\| \vect u_h\cdot\vect n \|_{\LTwo}$ & Order & $\| \vect P\,\vect u_h - \vect u \|_{\LTwo}$ & Order & Outer iterations & Residual norm \\
%			\hline
%			$8.33\times	10^{-1}$	&	$3.1\times	10^{-1}$	&	$\text{}$	&	$3.2\times	10^{-1}$	&	$\text{}$	&	$26$	&	$6.8\times	10^{-9}$	\\ \hline
$4.17\times	10^{-1}$	&	$4.7\times	10^{-2}$	&	$2.7$	&	$1.3\times	10^{-1}$	&	$1.3$	&	$33$	&	$4.9\times	10^{-9}$	\\ \hline
$2.08\times	10^{-1}$	&	$8.7\times	10^{-3}$	&	$2.4$	&	$6.8\times	10^{-3}$	&	$4.3$	&	$31$	&	$7.4\times	10^{-9}$	\\ \hline
$1.04\times	10^{-1}$	&	$1.\times	10^{-3}$	&	$3.1$	&	$1.5\times	10^{-3}$	&	$2.2$	&	$29$	&	$4.9\times	10^{-9}$	\\ \hline
%		\end{tabular}
%	\end{subtable}
%\end{table}

\section{Inf-sup stability: pressure Schur complement generalized eigenvalues}

\subsection{Solution description}

We define matrices 
\begin{equation}
	\vect C_0 \coloneqq \vect 0,\quad
	\vect M_n \coloneqq \vect M_0 + \vect C_n,\quad
	\vect M_{\text{full}} \coloneqq \vect M_0 + \vect C_{\text{full}}.
\end{equation}
We are interested in (generalized) extreme eigenvalues of the pressure Schur complement matrices
\begin{align}\label{schur}
	\vect S_0 \coloneqq \vect B\,\vect A^{-1}\,\vect B^{T},\quad
	\vect S_n \coloneqq \vect S_0 + \vect C_n,\quad
	\vect S_{\text{full}} \coloneqq \vect S_0 + \vect C_{\text{full}},
\end{align}
i.e. in solving
\begin{equation}\label{problem}
	\vect S_\star\,\vect x = \lambda\,\vect M_\star\,\vect x,
\end{equation}
where ``$\star$'' stands for ``$0$,'' ``$n$,'' or ``full.'' We denote by~$0 = \lambda_1 < \lambda_2 \le \dots \le \lambda_{n_{\vect S}} = O(1)$ the spectrum of~\eqref{problem}.

Computing $\vect A^{-1}$ in~\eqref{schur} becomes troublesome already for $h = 5.21\times10^{-2}$ ($n_{\vect A} = 32736$ for $\vect u \in \vect P_1$ FE space): although $\vect A$ is sparse, $\vect A^{-1}$ is dense and consumes 8.5+ GB in double-precision arithmetic. A quick research \href{https://mathematica.stackexchange.com/questions/189620/matrix-free-arnoldi-method-for-eigensystems}{showed} that \texttt{Mathematica} has no built-in matrix-free eigenvalue routines. \texttt{Intel MKL}'s FEAST algorithm for computing (generalized) eigenvalues in an interval \href{https://software.intel.com/sites/default/files/mkl-2019-developer-reference-c.pdf#_OPENTOPIC_TOC_PROCESSING_d62e853651}{is suitable for matrix-free implementations}; however, it requires some expensive operations to be implemented (e.g. matrix-matrix multiplications $\vect Y \leftarrow \vect S_\star\,\vect X$, $\vect Y \leftarrow \vect M_\star\,\vect X$ and approximating the action of inverses in the form $\vect y \leftarrow (\sigma\,\vect M_\star - \vect S_\star)^{-1}\,\vect x$).

Taking this into account, instead of~\eqref{problem} we consider a perturbed\footnotemark{} problem
\begin{equation}\label{problem_pert}
	\underbrace{\begin{bmatrix}
		\vect A & \phantom{-}\vect B^T \\
		\vect B & -\vect C_\star \\
		\end{bmatrix}}_{\mathcal A_\star \coloneqq}
	\begin{bmatrix}
	\vect x \\
	\vect y
	\end{bmatrix}
	=
	\mu
	\underbrace{\begin{bmatrix}
		\epsilon\,\vect A & \\
		& \vect M_\star
		\end{bmatrix}}_{\mathcal M^\epsilon_\star \coloneqq}
	\begin{bmatrix}
	\vect x \\
	\vect y
	\end{bmatrix}
\end{equation}
with $0 < \epsilon \ll 1$. For $\mathcal A_0$ and $\mathcal M^\epsilon_0$ we have
\begin{equation}
	\mu = -\lambda + o(1)\quad\text{or}\quad\epsilon^{-1} + \lambda + o(1),\qquad\epsilon \rightarrow 0.
\end{equation}
This makes it easy to pick only ``correct'' eigenvalues. To ease the computation further we replace the $(1, 1)$-block of~$\mathcal M^\epsilon_\star$ with $\epsilon\,\vect I$. 

To make sure that results are consistent we solve~\eqref{problem_pert} for~$\epsilon = 10^{-5}$ and~$\epsilon = 10^{-6}$; for the coarse mesh levels we also check that the dense solver for~\eqref{problem} and the iterative one for~\eqref{problem_pert} give solutions that coincide.  

\footnotetext{The majority of generalized eigenvalue solvers require left-hand-side matrix to be Hermitian and right-hand-side matrix to be Hermitian \textbf{positive definite}; that's why we need to introduce $\epsilon > 0$.}     

\clearpage

\subsection{Dependency of the spectrum on the mesh size}

\begin{table}[h]
	\centering\small
	\caption{Spectrum of~\eqref{problem} for inconsistent $\vect P_1$\,--\,$P_1$, $\tau = h^{-2}$, $\rho_u = h$, $\rho_p = h$, $m \equiv 2$} 
	\label{tab:p1p1}
	\begin{subtable}{1.\linewidth}
		\centering
		\caption{$\Gamma = \sphere$}
		\label{tab:p1p1:sph}
		\begin{tabular}[1.3]{|c|c|c|c|c|c|c|c|c|}
			\hline
			\multirow{2}{*}{$h$} & \multirow{2}{*}{$n_{\vect A}$} & \multirow{2}{*}{$n_{\vect S}$} & \multicolumn{2}{c|}{$\vect S_0$} & \multicolumn{2}{c|}{$\vect S_n$} & \multicolumn{2}{c|}{$\vect S_{\text{full}}$} \\ 
			\cline{4-9}
			& & & $\lambda_2$ & $\lambda_{n_{\vect S}}$ & $\lambda_2$ & $\lambda_{n_{\vect S}}$ & $\lambda_2$ & $\lambda_{n_{\vect S}}$ \\ 
			\hline
			$8.33\times	10^{-1}$	&	$153$	&	$51$	&	$1.32\times	10^{-2}$	&	$1.42$	&	$7.48\times	10^{-1}$	&	$1.13$	&	$9.58\times	10^{-1}$	&	$1.06$	\\ \hline
$4.17\times	10^{-1}$	&	$570$	&	$190$	&	$5.12\times	10^{-3}$	&	$1.04$	&	$5.77\times	10^{-1}$	&	$1.$	&	$8.54\times	10^{-1}$	&	$1.$	\\ \hline
$2.08\times	10^{-1}$	&	$1992$	&	$664$	&	$4.4\times	10^{-3}$	&	$7.93\times	10^{-1}$	&	$3.87\times	10^{-1}$	&	$1.$	&	$6.71\times	10^{-1}$	&	$1.$	\\ \hline
$1.04\times	10^{-1}$	&	$8292$	&	$2764$	&	$2.01\times	10^{-3}$	&	$7.75\times	10^{-1}$	&	$2.19\times	10^{-1}$	&	$1.$	&	$5.82\times	10^{-1}$	&	$1.$	\\ \hline
$5.21\times	10^{-2}$	&	$32736$	&	$10912$	&	$6.04\times	10^{-5}$	&	$9.81\times	10^{-1}$	&	$1.17\times	10^{-1}$	&	$1.$	&	$5.37\times	10^{-1}$	&	$1.$	\\ \hline
$2.6\times	10^{-2}$	&	$131592$	&	$43864$	&	$3.53\times	10^{-5}$	&	$8.67\times	10^{-1}$	&	$5.72\times	10^{-2}$	&	$1.$	&	$5.16\times	10^{-1}$	&	$1.$	\\ \hline
$1.3\times	10^{-2}$	&	$525864$	&	$175288$	&	$2.16\times	10^{-6}$	&	$7.34\times	10^{-1}$	&	$2.84\times	10^{-2}$	&	$1.$	&	$5.04\times	10^{-1}$	&	$1.$	\\ \hline
			%		\multirow{2}{*}{$h$} & \multirow{2}{*}{$n_{\vect A}$} & \multirow{2}{*}{$n_{\vect S}$} & \multicolumn{2}{c|}{$\vect S_0$} & \multicolumn{2}{c|}{$\vect S_n$} & \multicolumn{2}{c|}{$\vect S_{\text{full}}$} \\ 
			%		\cline{4-9}
			%		& & & $r_2$ & $r_{n_{\vect S}}$ & $r_2$ & $r_{n_{\vect S}}$ & $r_2$ & $r_{n_{\vect S}}$ \\ 
			%		\hline
			%		$2.08\times	10^{-1}$	&	$972$	&	$324$	&	$3.\times	10^{-10}$	&	$3.\times	10^{-17}$	&	$9.\times	10^{-13}$	&	$5.\times	10^{-8}$	&	$1.\times	10^{-13}$	&	$3.\times	10^{-7}$	\\ \hline
$1.04\times	10^{-1}$	&	$4740$	&	$1580$	&	$1.\times	10^{-15}$	&	$2.\times	10^{-18}$	&	$5.\times	10^{-11}$	&	$5.\times	10^{-8}$	&	$4.\times	10^{-10}$	&	$8.\times	10^{-8}$	\\ \hline
$5.21\times	10^{-2}$	&	$19704$	&	$6568$	&	$3.\times	10^{-15}$	&	$1.\times	10^{-18}$	&	$5.\times	10^{-14}$	&	$1.\times	10^{-5}$	&	$1.\times	10^{-10}$	&	$2.\times	10^{-4}$	\\ \hline
$2.6\times	10^{-2}$	&	$80808$	&	$26936$	&	$3.\times	10^{-19}$	&	$4.\times	10^{-19}$	&	$5.\times	10^{-13}$	&	$7.\times	10^{-5}$	&	$9.\times	10^{-13}$	&	$7.\times	10^{-4}$	\\ \hline
$1.3\times	10^{-2}$	&	$327036$	&	$109012$	&	$2.\times	10^{-20}$	&	$2.\times	10^{-22}$	&	$9.\times	10^{-14}$	&	$2.\times	10^{-4}$	&	$1.\times	10^{-12}$	&	$7.\times	10^{-4}$	\\ \hline
		\end{tabular}
	\end{subtable}%
	\vskip 2mm
	\begin{subtable}{1.\linewidth}
		\centering
		\caption{$\Gamma = \tor$}
		\label{tab:p1p1:tor}
		\begin{tabular}[1.3]{|c|c|c|c|c|c|c|c|c|}
			\hline
			\multirow{2}{*}{$h$} & \multirow{2}{*}{$n_{\vect A}$} & \multirow{2}{*}{$n_{\vect S}$} & \multicolumn{2}{c|}{$\vect S_0$} & \multicolumn{2}{c|}{$\vect S_n$} & \multicolumn{2}{c|}{$\vect S_{\text{full}}$} \\ 
			\cline{4-9}
			& & & $\lambda_2$ & $\lambda_{n_{\vect S}}$ & $\lambda_2$ & $\lambda_{n_{\vect S}}$ & $\lambda_2$ & $\lambda_{n_{\vect S}}$ \\ 
			\hline
			$2.08\times	10^{-1}$	&	$972$	&	$324$	&	$5.04\times	10^{-2}$	&	$4.93$	&	$2.84\times	10^{-1}$	&	$1.35$	&	$3.64\times	10^{-1}$	&	$1.19$	\\ \hline
$1.04\times	10^{-1}$	&	$4740$	&	$1580$	&	$2.99\times	10^{-3}$	&	$3.83$	&	$1.58\times	10^{-1}$	&	$1.02$	&	$3.35\times	10^{-1}$	&	$1.01$	\\ \hline
			%		\multirow{2}{*}{$h$} & \multirow{2}{*}{$n_{\vect A}$} & \multirow{2}{*}{$n_{\vect S}$} & \multicolumn{2}{c|}{$\vect S_0$} & \multicolumn{2}{c|}{$\vect S_n$} & \multicolumn{2}{c|}{$\vect S_{\text{full}}$} \\ 
			%		\cline{4-9}
			%		& & & $r_2$ & $r_{n_{\vect S}}$ & $r_2$ & $r_{n_{\vect S}}$ & $r_2$ & $r_{n_{\vect S}}$ \\ 
			%		\hline
			%		$2.08\times	10^{-1}$	&	$972$	&	$324$	&	$3.\times	10^{-10}$	&	$3.\times	10^{-17}$	&	$9.\times	10^{-13}$	&	$5.\times	10^{-8}$	&	$1.\times	10^{-13}$	&	$3.\times	10^{-7}$	\\ \hline
$1.04\times	10^{-1}$	&	$4740$	&	$1580$	&	$1.\times	10^{-15}$	&	$2.\times	10^{-18}$	&	$5.\times	10^{-11}$	&	$5.\times	10^{-8}$	&	$4.\times	10^{-10}$	&	$8.\times	10^{-8}$	\\ \hline
$5.21\times	10^{-2}$	&	$19704$	&	$6568$	&	$3.\times	10^{-15}$	&	$1.\times	10^{-18}$	&	$5.\times	10^{-14}$	&	$1.\times	10^{-5}$	&	$1.\times	10^{-10}$	&	$2.\times	10^{-4}$	\\ \hline
$2.6\times	10^{-2}$	&	$80808$	&	$26936$	&	$3.\times	10^{-19}$	&	$4.\times	10^{-19}$	&	$5.\times	10^{-13}$	&	$7.\times	10^{-5}$	&	$9.\times	10^{-13}$	&	$7.\times	10^{-4}$	\\ \hline
$1.3\times	10^{-2}$	&	$327036$	&	$109012$	&	$2.\times	10^{-20}$	&	$2.\times	10^{-22}$	&	$9.\times	10^{-14}$	&	$2.\times	10^{-4}$	&	$1.\times	10^{-12}$	&	$7.\times	10^{-4}$	\\ \hline
		\end{tabular}
	\end{subtable}
\end{table}
\vfill
\begin{figure}[h]
	\centering\small
	\begin{subfigure}{.49\linewidth}
		\centering
		\includegraphicsw{sphere_2_P1P1.png}
		%\caption{$\vect P_1$\,--\,$P_1$ for $\sphere$}
	\end{subfigure}%
	\hfill
	\begin{subfigure}{.49\linewidth}
		\centering
		\includegraphicsw{torus_P1P1.png}
		%\caption{$\vect P_1$\,--\,$P_1$ for $\tor$}
	\end{subfigure}
	\caption{Log-log plot of~$\lambda_2$ for Tables~\ref{tab:p1p1}\subref{tab:p1p1:sph} (left) and~\ref{tab:p1p1}\subref{tab:p1p1:tor} (right)}
\end{figure}
\vfill
\clearpage
\vfill
\begin{table}[h]
	\centering\small
	\caption{Spectrum of~\eqref{problem} for $\vect P_2$\,--\,$P_1$, $\Gamma = \sphere$, $m \in \vect m_{1/2}$ as in Figure~\ref{fig:m}} 
	\label{tab:p2p1}
	\begin{subtable}{1.\linewidth}
		\centering
		\caption{Consistent approach, $\tau = h^{-2}$, $\rho_u = h^{-1}$, $\rho_p = h$}
		\label{tab:p2p1:sph}
		\begin{tabular}[1.3]{|c|c|c|c|c|c|c|c|c|}
			\hline
			\multirow{2}{*}{$h$} & \multirow{2}{*}{$n_{\vect A}$} & \multirow{2}{*}{$n_{\vect S}$} & \multicolumn{2}{c|}{$\vect S_0$} & \multicolumn{2}{c|}{$\vect S_n$} & \multicolumn{2}{c|}{$\vect S_{\text{full}}$} \\ 
			\cline{4-9}
			& & & $\lambda_2$ & $\lambda_{n_{\vect S}}$ & $\lambda_2$ & $\lambda_{n_{\vect S}}$ & $\lambda_2$ & $\lambda_{n_{\vect S}}$ \\ 
			\hline
			$8.33\times	10^{-1}$	&	$789$	&	$51$	&	$2.33\times	10^{-1}$	&	$1.07$	&	$6.3\times	10^{-1}$	&	$1.$	&	$8.81\times	10^{-1}$	&	$1.$	\\ \hline
$4.17\times	10^{-1}$	&	$3276$	&	$190$	&	$4.72\times	10^{-2}$	&	$6.97\times	10^{-1}$	&	$5.29\times	10^{-1}$	&	$1.$	&	$7.64\times	10^{-1}$	&	$1.$	\\ \hline
$2.08\times	10^{-1}$	&	$11718$	&	$664$	&	$7.93\times	10^{-2}$	&	$6.7\times	10^{-1}$	&	$5.09\times	10^{-1}$	&	$1.$	&	$6.39\times	10^{-1}$	&	$1.$	\\ \hline
$1.04\times	10^{-1}$	&	$48762$	&	$2764$	&	$3.71\times	10^{-2}$	&	$6.69\times	10^{-1}$	&	$5.03\times	10^{-1}$	&	$1.$	&	$5.73\times	10^{-1}$	&	$1.$	\\ \hline
$5.21\times	10^{-2}$	&	$193086$	&	$10912$	&	$1.81\times	10^{-3}$	&	$6.68\times	10^{-1}$	&	$4.98\times	10^{-1}$	&	$1.$	&	$5.36\times	10^{-1}$	&	$1.$	\\ \hline
$2.6\times	10^{-2}$	&	$775998$	&	$43864$	&	$6.65\times	10^{-4}$	&	$6.65\times	10^{-1}$	&	$4.92\times	10^{-1}$	&	$1.$	&	$5.17\times	10^{-1}$	&	$1.$	\\ \hline
		\end{tabular}
	\end{subtable}%
%	\vskip 3mm
%	\begin{subtable}{1.\linewidth}
%		\centering
%		\caption{$\Gamma = \tor$}
%		\label{tab:p2p1:tor}
%		\begin{tabular}[1.3]{|c|c|c|c|c|c|c|c|c|}
%			\hline
%			\multirow{2}{*}{$h$} & \multirow{2}{*}{$n_{\vect A}$} & \multirow{2}{*}{$n_{\vect S}$} & \multicolumn{2}{c|}{$\vect S_0$} & \multicolumn{2}{c|}{$\vect S_n$} & \multicolumn{2}{c|}{$\vect S_{\text{full}}$} \\ 
%			\cline{4-9}
%			& & & $\lambda_2$ & $\lambda_{n_{\vect S}}$ & $\lambda_2$ & $\lambda_{n_{\vect S}}$ & $\lambda_2$ & $\lambda_{n_{\vect S}}$ \\ 
%			\hline
%			$2.08\times	10^{-1}$	&	$5184$	&	$324$	&	$9.92\times	10^{-2}$	&	$3.89$	&	$1.33\times	10^{-1}$	&	$1.37$	&	$1.75\times	10^{-1}$	&	$1.19$	\\ \hline
$1.04\times	10^{-1}$	&	$27906$	&	$1580$	&	$1.46\times	10^{-2}$	&	$4.35$	&	$2.84\times	10^{-1}$	&	$1.04$	&	$2.99\times	10^{-1}$	&	$1.02$	\\ \hline
$5.21\times	10^{-2}$	&	$116568$	&	$6568$	&	$6.08\times	10^{-3}$	&	$4.85$	&	$3.19\times	10^{-1}$	&	$1.01$	&	$3.24\times	10^{-1}$	&	$1.01$	\\ \hline
$2.6\times	10^{-2}$	&	$477660$	&	$26936$	&	$1.36\times	10^{-3}$	&	$4.92$	&	$3.14\times	10^{-1}$	&	$1.01$	&	$3.16\times	10^{-1}$	&	$1.$	\\ \hline
%		\end{tabular}
%	\end{subtable}
\end{table}
\vfill
\begin{figure}[h]
	\centering\small
	\begin{subfigure}{.49\linewidth}
		\centering
		\includegraphicsw{sphere_2_P2P1_consistent.png}
	\end{subfigure}%
%	\hfill
%	\begin{subfigure}{.49\linewidth}
%		\centering
%		\includegraphicsw{torus_P2P1.png}
%	\end{subfigure}
	\caption{Log-log plot of~$\lambda_2$ for Table~\ref{tab:p2p1}\subref{tab:p2p1:sph}}
\end{figure}
\vfill
\subsection{Sensitivity of the spectrum to levelset shifts}
TBA
\clearpage
%\clearpage
%\vfill
%\begin{table}[h]
%	\centering\small
%	\caption{Spectrum of~\eqref{problem} for $\vect P_2$\,--\,$P_1$} 
%	\label{tab:p2p1}
%	\begin{subtable}{1.\linewidth}
%		\centering
%		\caption{$\Gamma = \sphere$}
%		\label{tab:p2p1:sph}
%		\begin{tabular}[1.3]{|c|c|c|c|c|c|c|c|c|}
%			\hline
%			\multirow{2}{*}{$h$} & \multirow{2}{*}{$n_{\vect A}$} & \multirow{2}{*}{$n_{\vect S}$} & \multicolumn{2}{c|}{$\vect S_0$} & \multicolumn{2}{c|}{$\vect S_n$} & \multicolumn{2}{c|}{$\vect S_{\text{full}}$} \\ 
%			\cline{4-9}
%			& & & $\lambda_2$ & $\lambda_{n_{\vect S}}$ & $\lambda_2$ & $\lambda_{n_{\vect S}}$ & $\lambda_2$ & $\lambda_{n_{\vect S}}$ \\ 
%			\hline
%			$8.33\times	10^{-1}$	&	$789$	&	$51$	&	$3.22\times	10^{-1}$	&	$1.73$	&	$8.27\times	10^{-1}$	&	$1.17$	&	$9.68\times	10^{-1}$	&	$1.07$	\\ \hline
$4.17\times	10^{-1}$	&	$3240$	&	$190$	&	$9.17\times	10^{-2}$	&	$1.08$	&	$6.45\times	10^{-1}$	&	$1.$	&	$8.56\times	10^{-1}$	&	$1.$	\\ \hline
$2.08\times	10^{-1}$	&	$11718$	&	$664$	&	$1.78\times	10^{-1}$	&	$8.31\times	10^{-1}$	&	$5.49\times	10^{-1}$	&	$1.$	&	$6.75\times	10^{-1}$	&	$1.$	\\ \hline
$1.04\times	10^{-1}$	&	$48762$	&	$2764$	&	$1.04\times	10^{-1}$	&	$8.35\times	10^{-1}$	&	$5.14\times	10^{-1}$	&	$1.$	&	$5.82\times	10^{-1}$	&	$1.$	\\ \hline
$5.21\times	10^{-2}$	&	$193014$	&	$10912$	&	$2.99\times	10^{-3}$	&	$9.89\times	10^{-1}$	&	$5.02\times	10^{-1}$	&	$1.$	&	$5.34\times	10^{-1}$	&	$1.$	\\ \hline
$2.6\times	10^{-2}$	&	$775998$	&	$43864$	&	$1.17\times	10^{-3}$	&	$7.9\times	10^{-1}$	&	$4.96\times	10^{-1}$	&	$1.$	&	$5.17\times	10^{-1}$	&	$1.$	\\ \hline
%		\end{tabular}
%	\end{subtable}%
%	\vskip 3mm
%	\begin{subtable}{1.\linewidth}
%		\centering
%		\caption{$\Gamma = \tor$}
%		\label{tab:p2p1:tor}
%		\begin{tabular}[1.3]{|c|c|c|c|c|c|c|c|c|}
%			\hline
%			\multirow{2}{*}{$h$} & \multirow{2}{*}{$n_{\vect A}$} & \multirow{2}{*}{$n_{\vect S}$} & \multicolumn{2}{c|}{$\vect S_0$} & \multicolumn{2}{c|}{$\vect S_n$} & \multicolumn{2}{c|}{$\vect S_{\text{full}}$} \\ 
%			\cline{4-9}
%			& & & $\lambda_2$ & $\lambda_{n_{\vect S}}$ & $\lambda_2$ & $\lambda_{n_{\vect S}}$ & $\lambda_2$ & $\lambda_{n_{\vect S}}$ \\ 
%			\hline
%			$2.08\times	10^{-1}$	&	$5184$	&	$324$	&	$9.92\times	10^{-2}$	&	$3.89$	&	$1.33\times	10^{-1}$	&	$1.37$	&	$1.75\times	10^{-1}$	&	$1.19$	\\ \hline
$1.04\times	10^{-1}$	&	$27906$	&	$1580$	&	$1.46\times	10^{-2}$	&	$4.35$	&	$2.84\times	10^{-1}$	&	$1.04$	&	$2.99\times	10^{-1}$	&	$1.02$	\\ \hline
$5.21\times	10^{-2}$	&	$116568$	&	$6568$	&	$6.08\times	10^{-3}$	&	$4.85$	&	$3.19\times	10^{-1}$	&	$1.01$	&	$3.24\times	10^{-1}$	&	$1.01$	\\ \hline
$2.6\times	10^{-2}$	&	$477660$	&	$26936$	&	$1.36\times	10^{-3}$	&	$4.92$	&	$3.14\times	10^{-1}$	&	$1.01$	&	$3.16\times	10^{-1}$	&	$1.$	\\ \hline
%		\end{tabular}
%	\end{subtable}
%\end{table}
%\vfill
%\begin{figure}[h]
%	\centering\small
%	\begin{subfigure}{.49\linewidth}
%		\centering
%		\includegraphicsw{sphere_2_P2P1.png}
%	\end{subfigure}%
%	\hfill
%	\begin{subfigure}{.49\linewidth}
%		\centering
%		\includegraphicsw{torus_P2P1.png}
%	\end{subfigure}
%	\caption{Log-log plot of~$\lambda_2$ for Tables~\ref{tab:p2p1}\subref{tab:p2p1:sph} (left) and~\ref{tab:p2p1}\subref{tab:p2p1:tor} (right)}
%\end{figure}
%\vfill
%\clearpage
%\subsection{Sensitivity of the spectrum to levelset shifts}
%
%In this section we investigate the sensitivity of the spectrum to levelset shifts
%\begin{equation}\label{shift}
%\Gamma \mapsto \Gamma + \alpha\,\vect s,
%\end{equation}
%for some $\alpha \in \mathbb R$ and $\vect s \in \mathbb R^3$, $\|\vect s\| = 1$. 
%
%We construct the bulk mesh~$\Omega_h^\Gamma$ and then perform the assembly of matrices~\eqref{mtx} using the shifted levelset~\eqref{shift}. That is, the refinement of~$\Omega_h^\Gamma$ is performed using~$\Gamma$, not~$\Gamma + \alpha\,\vect s$, and~$\Omega_h^{\Gamma + \alpha\,\vect s}$ is never constructed. We choose $\alpha \in [0, h]$ to guarantee the appearance of ``small cuts'' in~$\Omega_h^\Gamma$.
%
%\begin{figure}[h]
%	\centering
%	\begin{subfigure}{.5\linewidth}
%		\centering
%		\includegraphicsw[.7]{{shift_0.0.cropped}.png}
%		\caption{$\sphere$}
%	\end{subfigure}%
%	\begin{subfigure}{.5\linewidth}
%		\centering
%		\includegraphicsw[.7]{{shift_0.2.cropped}.png}
%		\caption{$\sphere + \alpha\,\vect s$}
%	\end{subfigure}%
%	\caption{The unit sphere (left) and the shifted unit sphere (right). Here $\vect s = (0, 1, 1)^T/\sqrt{2}$, $\alpha = 0.2$, and $h = 2.08\times10^{-1}$. The bulk mesh~$\Omega_\Gamma^h$ is computed for~$\sphere$ and then used for~$\sphere + \alpha\,\vect s$}
%	\label{fig:shift}		
%\end{figure}
%
%\begin{table}[h]
%	\centering
%	\caption{Spectrum of~\eqref{problem} for perturbed levelset $\sphere + \alpha\,\vect s$. Here $\vect s = (1, 1, 1)^T/\sqrt{3}$, $h = 1.04\times10^{-1}$} 
%	\label{tab:p1p1_shift_h=0.104167}
%	\small
%	\begin{subtable}{1.\linewidth}
%		\centering
%		\caption{$\vect P_1$\,--\,$P_1$}
%		\begin{tabular}[1.3]{|c|c|c|c|c|c|c|}
%			\hline
%			\multirow{2}{*}{Surface} & \multicolumn{2}{c|}{$\vect S_0$} & \multicolumn{2}{c|}{$\vect S_n$} & \multicolumn{2}{c|}{$\vect S_{\text{full}}$} \\ 
%			\cline{2-7}
%			& $\lambda_2$ & $\lambda_{n_{\vect S}}$ & $\lambda_2$ & $\lambda_{n_{\vect S}}$ & $\lambda_2$ & $\lambda_{n_{\vect S}}$ \\ 
%			\hline
%			$\sphere$\phantom{ + 0.1\,h$\,\vect s$}	&	$2.006\times	10^{-3}$	&	$7.79\times	10^{-1}$	&	$2.19\times	10^{-1}$	&	$1.$	&	$5.818\times	10^{-1}$	&	$1.$	\\ \hline
$\sphere + 0.1\,h\,\vect s$	&	$4.832\times	10^{-4}$	&	$8.01\times	10^{-1}$	&	$2.195\times	10^{-1}$	&	$1.$	&	$5.818\times	10^{-1}$	&	$1.$	\\ \hline
$\sphere + 0.3\,h\,\vect s$	&	$7.278\times	10^{-4}$	&	$8.17\times	10^{-1}$	&	$2.203\times	10^{-1}$	&	$1.$	&	$5.818\times	10^{-1}$	&	$1.$	\\ \hline
$\sphere + 0.5\,h\,\vect s$	&	$3.121\times	10^{-4}$	&	$8.67\times	10^{-1}$	&	$2.221\times	10^{-1}$	&	$1.$	&	$5.82\times	10^{-1}$	&	$1.$	\\ \hline
$\sphere + 0.7\,h\,\vect s$	&	$1.438\times	10^{-3}$	&	$1.51$	&	$2.254\times	10^{-1}$	&	$1.$	&	$5.82\times	10^{-1}$	&	$1.$	\\ \hline
$\sphere + \phantom{0.1}\,h\,\vect s$	&	$1.79\times	10^{-3}$	&	$2.07$	&	$2.332\times	10^{-1}$	&	$1.$	&	$5.827\times	10^{-1}$	&	$1.$	\\ \hline
%		\end{tabular}
%	\end{subtable}
%	\vskip 4mm
%	\begin{subtable}{1.\linewidth}
%		\centering
%		\caption{$\vect P_2$\,--\,$P_1$}
%		\begin{tabular}[1.3]{|c|c|c|c|c|c|c|}
%			\hline
%			\multirow{2}{*}{Surface} & \multicolumn{2}{c|}{$\vect S_0$} & \multicolumn{2}{c|}{$\vect S_n$} & \multicolumn{2}{c|}{$\vect S_{\text{full}}$} \\ 
%			\cline{2-7}
%			& $\lambda_2$ & $\lambda_{n_{\vect S}}$ & $\lambda_2$ & $\lambda_{n_{\vect S}}$ & $\lambda_2$ & $\lambda_{n_{\vect S}}$ \\ 
%			\hline
%			$\sphere$\phantom{ + 0.1\,h$\,\vect s$}	&	$1.041\times	10^{-1}$	&	$8.35\times	10^{-1}$	&	$5.138\times	10^{-1}$	&	$1.$	&	$5.841\times	10^{-1}$	&	$1.$	\\ \hline
$\sphere + 0.1\,h\,\vect s$	&	$1.705\times	10^{-3}$	&	$8.58\times	10^{-1}$	&	$5.138\times	10^{-1}$	&	$1.$	&	$5.841\times	10^{-1}$	&	$1.$	\\ \hline
$\sphere + 0.3\,h\,\vect s$	&	$2.293\times	10^{-3}$	&	$8.81\times	10^{-1}$	&	$5.137\times	10^{-1}$	&	$1.$	&	$5.841\times	10^{-1}$	&	$1.$	\\ \hline
$\sphere + 0.5\,h\,\vect s$	&	$5.63\times	10^{-3}$	&	$9.35\times	10^{-1}$	&	$5.138\times	10^{-1}$	&	$1.$	&	$5.844\times	10^{-1}$	&	$1.$	\\ \hline
$\sphere + 0.7\,h\,\vect s$	&	$8.188\times	10^{-3}$	&	$1.77$	&	$5.138\times	10^{-1}$	&	$1.$	&	$5.843\times	10^{-1}$	&	$1.$	\\ \hline
$\sphere + \phantom{0.1}\,h\,\vect s$	&	$1.93\times	10^{-2}$	&	$2.21$	&	$5.142\times	10^{-1}$	&	$1.$	&	$5.852\times	10^{-1}$	&	$1.$	\\ \hline
%		\end{tabular}%
%	\end{subtable}
%\end{table}

%\section{Notes on DROPS implementation}
%
%\subsection{Notations}\label{subsec:not}
%
%We denote by~$P_h^n \subset \bar P_h^n$ spaces of continuous and discontinuous nodal~$P_n$ interpolants defined on~$\Omega_\Gamma^h$, respectively. For a function~$f$, $I_h^n(f) \in P_h^n$ is the corresponding interpolant; we will use the notation~$f_h^n$ to emphasize that~$f_h^n \in P_h^n$ and~$f_h^n$ approximates~$f$ in some sense, but~$I_h^n(f) \ne f_h^n$.
%
%We set
%\begin{align}\label{gammah}
%	\Gamma_h^n &\coloneqq \{ \vect x \in \mathbb{R}^3 : \big(I_h^n(\phi)\big)(\vect x) = 0 \}, \\
%	\vect n_{\Gamma_h^n} &= \frac{\nabla I_h^n(\phi)}{\|\nabla I_h^n(\phi)\|} \not\in \bar{P}_h^m\text{ for any $m$ if $n > 1$}. \label{gammah:n}
%\end{align}  
%Note that~$\Gamma_h^n$ is a continuous piecewise $P_n$ surface in~$\Omega_\Gamma^h$, and $\Gamma_h^n \ne I_h^n(\Gamma)$. The unit normal~$\vect n_{\Gamma_h^n}$ is not a rational function; it is continuous in~$T \in \Omega_\Gamma^h$ and discontinuous on faces. We also define
%\begin{equation}\label{gammah2}
%	\Gamma_{h/m}^{2 \rightarrow 1} \coloneqq \{ \vect x \in \mathbb{R}^3 : \Big(I_{h/m}^1\big(I_h^2(\phi)\big)\Big)(\vect x) = 0 \}.
%\end{equation}  
%Note that~$I_{h/2}^1\big(I_{h}^2(\phi)\big) = I_{h/2}^1(\phi)$ (since in order to build both~$I_{h/2}^1$ and~$I_{h}^2$ the same values of~$\phi$ are used), and~$I_{h/m}^1\big(I_{h}^2(\phi)\big) \ne I_{h/m}^1(\phi)$ for~$m > 2$. Thus we have~$\Gamma_{h/2}^{2 \rightarrow 1} = \Gamma_{h/2}^1$, and~$\Gamma_{h/m}^{2 \rightarrow 1} \ne \Gamma_{h/m}^1$ for~$m > 2$.
%%Note also that~$\Gamma_{h/m}^{2 \rightarrow 1} \ne I_{h/m}^1(\Gamma_h^2)$ since $\Gamma_{h/m}^{2 \rightarrow 1}$ connects roots of piecewise \textbf{linear} functions, and~$I_{h/m}^1(\Gamma_h^2)$ connects roots  
%
%\subsection{Approximation of integrands involving~$\text{n}_\Gamma$}\label{subsec:app}
%
%We start with description of the continuous levelset~$\phi$ of~$\Gamma = \left\{ \vect x \in \mathbb{R}^3 : \phi(\vect x) = 0 \right\}$. It is stored in~\texttt{levelset\_fun} variable. For example, for the unit sphere we have:  
%\begin{lstlisting}
%// surfnavierstokes_funcs.h
%DROPS::Point3DCL sphere_2_shift(0.);
%double sphere_2 (const DROPS::Point3DCL& p, double) {
%	return pow(p[0] - sphere_2_shift[0], 2.) + 
%	       pow(p[1] - sphere_2_shift[1], 2.) + 
%	       pow(p[2] - sphere_2_shift[2], 2.) - 1.;
%}
%
%// surfnavierstokes.cpp
%instat_scalar_fun_ptr levelset_fun;
%// ...
%levelset_fun = &sphere_2;
%\end{lstlisting}
%\vskip .2cm
%Continuous piecewise $P_2$ interpolant~$I_h^2(\phi)$ of $\phi$ is built on~$\Omega_\Gamma^h$ via iterating over vertices and edges of~$\Omega_\Gamma^h$. It is stored in~\texttt{lset} object:
%\begin{lstlisting}
%// levelset.cpp
%void LevelsetP2ContCL::Init( instat_scalar_fun_ptr phi0, double t) {
%	const Uint lvl= Phi.GetLevel(),
%	idx= Phi.RowIdx->GetIdx();
%	for (auto it = MG_.GetTriangVertexBegin(lvl), end = MG_.GetTriangVertexEnd(lvl); it != end; ++it) {
%		if (it->Unknowns.Exist(idx))
%			Phi.Data[it->Unknowns(idx)]= phi0( it->GetCoord(), t);
%	}
%	for (auto it = MG_.GetTriangEdgeBegin(lvl), end = MG_.GetTriangEdgeEnd(lvl); it != end; ++it) {
%		if (it->Unknowns.Exist(idx))
%			Phi.Data[it->Unknowns(idx)]= phi0( GetBaryCenter( *it), t);
%	}
%}
%
%// surfnavierstokes.cpp
%DROPS::LevelsetP2CL& lset(*DROPS::LevelsetP2CL::Create(mg, lsbnd, sf));
%// ...
%lset.Init(levelset_fun);
%\end{lstlisting}
%\vskip .2cm
%In order to assemble matrices in~\eqref{mtx} for e.g. $\vect P_1$\,--\,$P_1$ elements, one calls
%\begin{lstlisting}
%SetupNavierStokesIF_P1P1(mg, &A, /* ... */ lset.Phi, /* ... */);
%\end{lstlisting}
%(\textbf{Interestingly enough}, this function does not get~\texttt{lset} object that represents the interpolant; it gets only~\texttt{lset.Phi}, which is the object of type~\texttt{VecDescCL}. \texttt{lset.Phi} is essentially just a vector of values of~$\phi$ at interpolation points (i.e. vertices and edges' centroids of~$\Omega_\Gamma^h$). That is, the assembling function above has no idea what~\texttt{lset.Phi} actually represents: one may interpret it as an element of~$P_h^2$ or e.g. $P_{h/2}^1$. Who knows?..)
% 
%\textbf{No} interpolation is built explicitly for~$\vect n_{\Gamma_h^2}$ in~\eqref{gammah:n}; it is implicitly represented via \texttt{qnormal} data field:
%\begin{lstlisting}
%// ifacetransp.cpp
%class LocalStokesCL {
%	// ...
%	GridFunctionCL<Point3DCL> qnormal;
%	// ...
%}
%\end{lstlisting}
%\texttt{qnormal} object is essentially a set of values of type~\texttt{Point3DCL} which are obtained by mapping a (vector valued) function to suitable quadrature nodes. This is how it is constructed:   
%\begin{lstlisting}
%// ifacetransp.cpp
%void LocalStokesCL::Get_Normals(const LocalP2CL<>& ls, LocalP1CL<Point3DCL>& Normals) {
%	for(int i=0; i<10 ; ++i)
%		Normals+=ls[i]*P2Grad[i];
%}
%// ...
%void LocalStokesCL::calcIntegrands(const SMatrixCL<3,3>& T, const LocalP2CL<>& ls, const TetraCL& tet) {
%	// ...
%	LocalP1CL<Point3DCL> Normals;
%	Get_Normals(ls, Normals);
%	resize_and_evaluate_on_vertexes (Normals, q2Ddomain, qnormal);
%	for(Uint i=0; i<qnormal.size(); ++i) 
%		qnormal[i]= qnormal[i]/qnormal[i].norm();
%	// ...
%}
%\end{lstlisting}
%First $\nabla I_h^2(\phi|_T)$ is built (locally for a tetrahedron~$T \in \Omega_\Gamma^h$ represented by~\texttt{tet}) and saved to \texttt{Normals} object. \texttt{ls[i]} gives the value of~$\phi|_T$ at $i$th node (vertices and edges' centroids\,---\,there are 10 of them for tetrahedra), and \texttt{P2Grad[i]} represents the gradient of quadratic basis function which itself is linear. (Actually, it is sufficient to have $4 < 10$ linear functions to represent $\nabla I_h^2(\phi|_T)$, but this is how it is implemented here.) Finally, \texttt{qnormal} object is built via evaluating \texttt{Normals} at quadrature nodes and normalization.
%
%Objects \texttt{qnormal} for surface integrals and \texttt{q3Dnormal} for volume integrals are used in approximation of~$\vect P = \vect I - \vect n\,\vect n^T$, normal derivatives, and taking-normal-components in~\eqref{mtx}. \texttt{q3Dnormal} is constructed as~\texttt{qnormal} but for quadrature points of tetrahedrons, not triangles. 
%
%For one, $\vect P\,\nabla f_h^2$, $f_h^2 \coloneqq P_2$ basis function on~$\Omega_\Gamma^h$, is approximated via~\texttt{qsurfP2grad} object:
%\begin{lstlisting}
%// ifacetransp.cpp
%void LocalStokesCL::calcIntegrands(/* ... */) {
%    // ...
%    for(int j=0; j<10 ;++j) {
%		resize_and_evaluate_on_vertexes(P2Grad[j], q2Ddomain, qsurfP2grad[j]);
%		qsurfP2grad[j]-= dot(qsurfP2grad[j], qnormal)*qnormal;
%	}
%	// ...
%}
%\end{lstlisting}  
%The term~$\int_{\Omega^{\Gamma}_h} \frac{\partial p}{\partial\vect n} \frac{\partial q}{\partial\vect n} \diff{\vect x}$ in~\eqref{mtx} is computed as
%\begin{lstlisting}
%// ifacetransp.cpp
%void LocalStokesCL::setupA_P1_stab(double A_P1_stab[4][4], double absdet) {
%	for (int i=0; i<4; ++i) 
%		for (int j=0; j<4; ++j) 
%			A_P1_stab[i][j] = quad(dot(q3Dnormal, q3DP1Grad[i])*dot(q3Dnormal, q3DP1Grad[j]), absdet, q3Ddomain, AllTetraC);
%}
%\end{lstlisting}
%
%\subsection{Quadrature rules for~$\int_{\Gamma}$ and $\int_{\Omega_\Gamma^h}$}\label{subsec:integrals}
%
%All the 3D integrals in~\eqref{mtx} are computed via iteration over~$T \in \Omega_\Gamma^h$ without any virtual refinements. \texttt{q3Ddomain} object represents the set of quadrature nodes and weights:
%\begin{lstlisting}
%// ifacetransp.cpp
%void LocalStokesCL::calc3DIntegrands(/* ... */) {
%	make_SimpleQuadDomain<Quad5DataCL> (q3Ddomain, AllTetraC);
%	// ...
%}
%\end{lstlisting} 
%It is used e.g. in~\texttt{setupA\_P1\_stab} above. 15 nodes and weights are used, and the quadrature is exact for functions in~$\bar P_h^5$.
%
%All the surface integrals are also computed via iteration over~$T \in \Omega_\Gamma^h$, but using~$\Gamma_{h/2}^1$. One extra ``virtual'' refinement is achieved via setting 
%\begin{lstlisting}
%// ifacetransp.cpp
%LocalStokesCL(bool fullGradient) 
%	: lat(PrincipalLatticeCL::instance(2))
%	, /* ... */ { /* ... */ }
%\end{lstlisting}
%\texttt{PrincipalLatticeCL::instance(2)} means that each edge of the tetrahedron~$T \in \Omega_\Gamma^h$ is split into 2 edges, and~$T$ is split into 8 smaller tetrahedrons. Changing 2 to 4 will give us~$\Gamma_{h/4}^{2 \rightarrow 1}$ from~\eqref{gammah2} and so forth. \texttt{q2Ddomain} object represents the set of quadrature nodes and weights:
%\begin{lstlisting}
%// ifacetransp.cpp
%void LocalStokesCL::calcIntegrands(/* ... */) {
%	// ...
%	evaluate_on_vertexes( ls, lat, Addr( ls_loc));
%	spatch.make_patch<MergeCutPolicyCL>( lat, ls_loc);
%	make_CompositeQuad5Domain2D ( q2Ddomain, spatch, tet);
%	// ...
%}
%\end{lstlisting} 
%Each linear subsurface in~$T \in \Omega_\Gamma^h$ has 7 quadrature nodes and weights, and the quadrature rule is again exact for functions in~$\bar P_h^5$.
%
%\texttt{spatch} represents a set of triangles that form~$\Gamma_{h/2}^1$ inside~$T$. That is, in order to approximate zeros of~$\phi$, $I_{h/2}^1\big(I_{h}^2(\phi)\big) = I_{h/2}^1(\phi)$ is used:
%\begin{lstlisting}
%// subtriangulation.h
%// ...
%const double edge_bary1_cut= ls0/(ls0 - ls1); // the root of the level set function on the edge
%// ...
%\end{lstlisting}
%Here~$l(x) := \texttt{ls0}\,(1-x) + \texttt{ls1}\,x$ is a linear function defined on the master edge~$[0, 1]$. Indeed, its root is~$x = \texttt{ls0/(ls0 - ls1)}$.
%
%\subsection{Approximation of the shape operator~\textbf{H}}\label{sec:shape}
%
%%With~$\vect P : \mathcal{O}(\Gamma) \rightarrow \mathbb R^3$ defined as~$\vect P \coloneqq \vect I - \vect n^e\,(\vect n^e)^T$, we have. 
%%\begin{equation*}
%%	\vect H \coloneqq \nabla_{\Gamma} \vect n \coloneqq \vect P\,\nabla \vect n^e\,\vect P = \vect P\,\nabla \vect n^e,
%%\end{equation*}
%%$\vect H_\Gamma : \mathcal{O}(\Gamma) \rightarrow \mathbb{R}^3$. Note that~$\vect n_\phi \coloneqq \nabla \phi / \|\nabla \phi\|$ is defined in~$\mathcal{O}(\Gamma)$, so~$\nabla \vect n_\phi$ makes sense and
%%\begin{equation*}
%%	\nabla \vect n_\phi = \Big(\vect I - \frac{\nabla \phi\,\nabla \phi^T}{\|\nabla \phi\|^2}\Big)\frac{\nabla^2 \phi}{\|\nabla \phi\|} = \big(\vect I - \vect n_\phi\,\vect n_\phi^T\big)\,\frac{\nabla^2 \phi}{\|\nabla \phi\|} = \vect P_\phi\,\frac{\nabla^2 \phi}{\|\nabla \phi\|}.
%%\end{equation*}
%%Note that~$\vect n_\phi = \vect n$ on~$\Gamma$ for any smooth levelset~$\phi$, so
%%\begin{equation*}
%%	\vect H = \vect P\,\nabla \vect n^e_\phi.
%%\end{equation*}
%%Depending on the choice of~$\phi$, we may or may not have~$\vect n_\phi = \vect n^e_\phi$ in~$\mathcal{O}(\Gamma)$. Note that the choice~$\phi = d$ is sufficient for this, but \textbf{not necessary}. Consider these choices of~$\phi$ for $\sphere$:
%%\begin{enumerate}
%%	\item $\phi_1(\vect x) = \|\vect x\| - 1 = d(\vect x)$, $\vect n_{\phi_1} = \vect n^e$ in~$\mathcal{O}(\Gamma)$,
%%	\item $\phi_2(\vect x) = \|\vect x\|^2 - 1 \in P^2$, $\vect n_{\phi_2} = \vect n_{\phi_1} = \vect n^e$ in~$\mathcal{O}(\Gamma)$,
%%	\item $\phi_3(\vect x) = e^{\phi_2(\vect x)}\,x^2 + y^2 + z^2 -1$, $\vect n_{\phi_3} = \vect n$ only on~$\sphere$, i.e.~$\vect n_{\phi_3} \ne \vect n^e$ in~$\mathcal{O}(\Gamma)$.
%%\end{enumerate}
%%
%%For a smooth function~$\vect f : \mathcal{O}(\Gamma) \rightarrow \mathbb R^3$, one has~$\nabla \vect f^e = \nabla (\vect f \circ \vect p) = \big(\nabla \vect p\big)^T\,\big(\nabla \vect f\big)^e$, so
%%\begin{equation*}
%%	\vect H = \vect P\,\nabla \vect n_\phi^e = \vect P\,\big(\nabla \vect p\big)^T\,\big(\nabla \vect n_\phi \big)^e = \vect P\,\big(\nabla \vect p\big)^T\,\Big(\vect P_\phi\,\frac{\nabla^2 \phi}{\|\nabla \phi\|} \Big)^e = \vect P\,\big(\nabla \vect p\big)^T\,\vect P\,\Big(\frac{\nabla^2 \phi}{\|\nabla \phi\|} \Big)^e.
%%\end{equation*}
%%
%%\begin{equation}\label{H}
%%\vect H_\Gamma = \vect P_{\Gamma}\,\frac{\nabla^2 \phi}{\|\nabla \phi\|}\,\vect P_{\Gamma}.
%%\end{equation}
%%Thus we define~$\vect H_{\Gamma^2_h}$ to be as in~\eqref{H} but with $\phi$ replaced with~$I^2_h(\phi)$. Indeed, computation of~$\vect H_{\Gamma^2_h}$ requires Hessians of shape functions, but luckily this routine was implemented in DROPS.
%%
%%Note that if~$\phi$ is piecewise quadratic in~$\Omega^\Gamma_h$, then~$\vect H_{\Gamma^2_h}$ is represented \textbf{exactly}. I actually checked it via computing~$\vect H_{\Gamma^2_h}$ at several points in DROPS and compared it with~$\vect H$ for~$\sphere$. 
%%
%%For the approach~\eqref{mtx_exact_2}, there is also an option to approximate~$\vect H$ as~$\vect P_{\Gamma_{h/m}^{2\to 1}}\,\frac{\nabla^2 I_h^2(\phi)}{\|\nabla I_h^2(\phi)\|}\,\vect P_{\Gamma_{h/m}^{2\to 1}}$ since we build~$\vect P_{\Gamma_{h/m}^{2\to 1}}$ anyway. I chose to use~$\vect H_{\Gamma_h^2} \coloneqq \vect P_{\Gamma_h^2}\,\frac{\nabla^2 I_h^2(\phi)}{\|\nabla I_h^2(\phi)\|}\,\vect P_{\Gamma_h^2}$ for both~\eqref{mtx_exact} and ~\eqref{mtx_exact_2}.
%
%We have~$\vect H_\Gamma \coloneqq \nabla_\Gamma \vect n_\Gamma \coloneqq \vect P_{\Gamma}\,\nabla \vect n^e_\Gamma\,\vect P_{\Gamma}$, $\vect H_\Gamma : \mathcal{O}(\Gamma) \rightarrow \mathbb{R}^3$. Note that~$\vect n_\Gamma = \nabla \phi / \|\nabla \phi\|$ is defined in~$\mathcal{O}(\Gamma)$, so~$\nabla \vect n_\Gamma$ makes sense and
%\begin{equation*}
%	\nabla \vect n_\Gamma = \Big(\vect I - \frac{\nabla \phi\,\nabla \phi^T}{\|\nabla \phi\|^2}\Big)\frac{\nabla^2 \phi}{\|\nabla \phi\|} = \vect P_{\Gamma}\,\frac{\nabla^2 \phi}{\|\nabla \phi\|}.
%\end{equation*}
%If~$\vect n_\Gamma = \vect n^e_\Gamma$, one gets
%\begin{equation}\label{H}
%	\vect H_\Gamma = \vect P_{\Gamma}\,\frac{\nabla^2 \phi}{\|\nabla \phi\|}\,\vect P_{\Gamma}.
%\end{equation}
%Thus we define~$\vect H_{\Gamma^2_h}$ to be as in~\eqref{H} but with $\phi$ replaced with~$I^2_h(\phi)$:
%\begin{equation}\label{Hh}
%	\vect H_{\Gamma_h^2} \coloneqq \vect P_{\Gamma_h^2}\,\frac{\nabla^2 I_h^2(\phi)}{\|\nabla I_h^2(\phi)\|}\,\vect P_{\Gamma_h^2}.
%\end{equation}
%Indeed, computation of~$\vect H_{\Gamma^2_h}$ requires Hessians of shape functions, but luckily this routine was implemented in DROPS.
%
%Depending on the choice of~$\phi$, we may or may not have~$\vect n_\Gamma = \vect n^e_\Gamma$. Note that the choice~$\phi = d$ is sufficient for this, but not necessary. Consider this choices of~$\phi$ for $\sphere$:
%\begin{enumerate}
%	\item $\phi_1(\vect x) = \|\vect x\| - 1 = d(\vect x)$, $\nabla \phi_1 / \|\nabla \phi_1\| = \vect n_\Gamma^e$,
%	\item $\phi_2(\vect x) = \|\vect x\|^2 - 1 \in P^2$, $\nabla \phi_2 / \|\nabla \phi_2\| = \nabla \phi_1 / \|\nabla \phi_1\| = \vect n_\Gamma^e$,
%	\item $\phi_3(\vect x) = e^{\phi_2(\vect x)}\,x^2 + y^2 + z^2 -1$, $\nabla \phi_3 / \|\nabla \phi_3\| \ne \vect n_\Gamma^e$, i.e. $\nabla \phi_3 / \|\nabla \phi_3\| = \vect n_\Gamma$ only on~$\sphere$.
%\end{enumerate}
%As for the case 2: note that if~$\phi$ is piecewise quadratic in~$\Omega^\Gamma_h$ and defines a normal that is equal to its extension, then~$\vect H_{\Gamma^2_h} = \vect H_{\Gamma}$, i.e. the approximation is \textbf{exact}.
%
%For the approach~\eqref{mtx_exact_2}, there is also an option to approximate~$\vect H$ as~$\vect P_{\Gamma_{h/m}^{2\to 1}}\,\frac{\nabla^2 I_h^2(\phi)}{\|\nabla I_h^2(\phi)\|}\,\vect P_{\Gamma_{h/m}^{2\to 1}}$ since we build~$\vect P_{\Gamma_{h/m}^{2\to 1}}$ anyway. I chose to use~\eqref{Hh} for both~\eqref{mtx_exact} and~\eqref{mtx_exact_2}.
%
%%We have~$\vect H_\Gamma \coloneqq \nabla \vect n_\Gamma$. Using~$\vect n_\Gamma = \nabla \phi / \|\nabla \phi\|$, one gets
%%\begin{equation}\label{H}
%%	\vect H_\Gamma = \Big(\vect I - \frac{\nabla \phi\,\nabla \phi^T}{\|\nabla \phi\|^2}\Big)\frac{\nabla^2 \phi}{\|\nabla \phi\|} = \vect P_{\Gamma}\,\frac{\nabla^2 \phi}{\|\nabla \phi\|}.
%%\end{equation}
%%Thus we define~$\vect H_{\Gamma^2_h}$ to be as in~\eqref{H} but with $\phi$ replaced with~$I^2_h(\phi)$. Indeed, computation of~$\vect H_{\Gamma^2_h}$ requires Hessians of shape functions, but luckily this routine was implemented in DROPS.
%%
%%Note that if~$\phi$ is piecewise quadratic on~$\Omega_\Gamma^h$, then~$\vect H_{\Gamma^2_h}$ is represented \textbf{exactly}. I actually checked it via computing~$\vect H_{\Gamma^2_h}$ at several points in DROPS and compared it to~$\vect H$ for~$\sphere$. 
%%
%%Comments:
%%
%%\begin{itemize}
%%	\item For the approach~\eqref{mtx_exact_2}, there is also an option to approximate~$\vect H$ as~$\vect P_{\Gamma_{h/m}^{2\to 1}}\,\frac{\nabla^2 I_h^2(\phi)}{\|\nabla I_h^2(\phi)\|}$ since we build~$\vect P_{\Gamma_{h/m}^{2\to 1}}$ anyway. I chose to use~$\vect H_{\Gamma_h^2} \coloneqq \vect P_{\Gamma_h^2}\,\frac{\nabla^2 I_h^2(\phi)}{\|\nabla I_h^2(\phi)\|}$ for both~\eqref{mtx_exact} and ~\eqref{mtx_exact_2}.
%%	\item From the identity~$E_{s,\,\Gamma}(\vect a) = E_{s,\,\Gamma}(\vect a_T) + a_N\,\vect H_{\Gamma}$ we have that $\vect H = E_{s,\,\Gamma}(\vect n_{\Gamma}) = \vect P_{\Gamma}\,\frac{\nabla^2 \phi}{\|\nabla \phi\|}\,\vect P_{\Gamma} = \nabla \vect n_\Gamma\,\vect P_{\Gamma}$. If $\vect n_\Gamma^e \equiv \vect n_\Gamma$, then we have~\eqref{H}.
%%	\item But it maybe the case that for~$\phi \ne d$, e.g.~$\phi(x, y, z) = 2 x^2 + y^2 +z^2 - 1$, we have that~$\vect n_{\Gamma} = \nabla \phi / \|\nabla \phi\| \ne \vect n^e_{\Gamma}$, and~$\nabla \vect n_{\Gamma}$ is not symmetric. Is not it better to take~$\vect H = \vect P_{\Gamma}\,\frac{\nabla^2 \phi}{\|\nabla \phi\|}\,\vect P_{\Gamma}$ to guarantee its symmetry? In the end of the day $\vect H = E_{s,\,\Gamma}(\vect n_{\Gamma}) = \vect H^T$.
%%	\item Is it true if $\Dist(\Gamma, \Gamma_h) \lesssim h^3$, then $\| \vect n - \vect n_h \|_{\LTwo} \lesssim h^2$? I do not really see $h^2$ convergence of the normal in Table~\ref{tab:p2p1_conv_cons_mright}... I figured out that computing~$\Dist(\sphere, \sphere_{h/m}^1)$ is easy since one just need to take maximum of distance function $\|\vect x\| - 1$ over all vertices of the patch. I may add it real quick to see if~$\Dist(\sphere, \sphere_{h/m}^1) \lesssim h^3$ for $m = O(h^{-1/2})$.
%%\end{itemize}
% 
%\subsection{Summary on the matrix assembly}
%
%The matrices in~\eqref{mtx} are assembled as
%\begin{align}\begin{split}\label{mtx_exact}
%	\langle \vect A\,\vec{\vect u}, \vec{\vect v} \rangle &= 
%		\int^5_{\Gamma_{h/m}^{2 \to 1}} \big( 2\,E_{s,\,\Gamma_{h}^2}(\vect u) : E_{s,\,\Gamma_{h}^2}(\vect v) + \vect u\cdot\vect v + \tau\,(\vect u\cdot\vect n_{\Gamma_{h}^2})\,(\vect v\cdot\vect n_{\Gamma_{h}^2}) \big) \diff{s} \\
%	&
%		+ \rho_u \int^5_{\Omega_h^{\Gamma}} \frac{\partial \vect u}{\partial\vect n_{\Gamma_{h}^2}}\cdot\frac{\partial \vect v}{\partial\vect n_{\Gamma_{h}^2}} \diff{\vect x}, \quad \vect A \in \mathbb R^{n_{\vect A} \times n_{\vect A}},\\
%	\langle \vect B\,\vec{\vect u}, \vec{\vect q} \rangle &= 
%		\int^5_{\Gamma_{h/m}^{2 \to 1}} \nabla_{\Gamma_h^2} q \cdot \vect u \diff{s}, \quad \vect B \in \mathbb R^{n_{\vect S} \times n_{\vect A}},\\
%	\langle \vect M_0\,\vec{\vect p}, \vec{\vect q} \rangle &=
%		\int^5_{\Gamma_{h/m}^{2 \to 1}} p\,q \diff{s}, \quad \vect M_0 \in \mathbb R^{n_{\vect S} \times n_{\vect S}},\\
%	\langle \vect C_n\,\vec{\vect p}, \vec{\vect q} \rangle &=
%		\rho_p \int^5_{\Omega^{\Gamma}_h} \frac{\partial p}{\partial\vect n_{\Gamma_{h}^2}} \frac{\partial q}{\partial\vect n_{\Gamma_{h}^2}} \diff{\vect x}, \quad \vect C_n \in \mathbb R^{n_{\vect S} \times n_{\vect S}},\\
%	\langle \vect C_{\text{full}}\,\vec{\vect p}, \vec{\vect q} \rangle &=
%		\rho_p \int^5_{\Omega^{\Gamma}_h} \nabla p \cdot \nabla q \diff{\vect x}, \quad \vect C_{\text{full}} \in \mathbb R^{n_{\vect S} \times n_{\vect S}},		 
%\end{split}\end{align}
%Comments:
%\begin{itemize}
%	\item $\int^5_{\Gamma_{h/m}^{2 \to 1}} \cdot \diff{s}$ denotes a composite quadrature rule that is exact for~$\bar P_h^5(\Gamma_{h/m}^{2 \to 1})$, i.e. this quadrature is exact for piecewise polynomials up to degree~5 on each triangular patch~$\gamma \in \Gamma_{h/m}^{2 \to 1}$, 
%	\item $\int^5_{\Omega^{\Gamma}_h} \cdot \diff{\vect x}$ denotes a composite quadrature rule that is exact for~$\bar P_h^5(\Omega^{\Gamma}_h)$, i.e. this quadrature is exact for piecewise polynomials up to degree~5 on each tetrahedron~$T \in \Omega^{\Gamma}_h$, 
%	\item $E_{s,\,\Gamma_{h}^2}$ and~$\nabla_{\Gamma_h^2}$ are defined as their continuous analogues with~$\vect n_{\Gamma}$ in~$\vect P$ replaced with~$\vect n_{\Gamma_{h}^2}$,
%	\item It is always the case that integrands use~$\vect n_{\Gamma_{h}^2} \ne \vect n_{\Gamma_{h/m}^{2 \to 1}}$, and the actual domain of integration is~$\Gamma_{h/m}^{2 \to 1} \ne \Gamma_{h}^2$,
%	\item $\vect n_{\Gamma_{h}^2}$ is defined in~\eqref{gammah:n} and it is not a polynomial even locally, thus quadrature rules are never exact (although for $\vect P_2$\,--\,$P_1$ shape functions alone these quadratures are exact).
%	%\item In our examples we know~$\phi$ and~$\nabla \phi$ exactly, and thus it is super easy to feed the \textbf{exact} normal~$\vect n_{\Gamma} \ne \vect n_{\Gamma_{h}^2}$ to quadratures. Shall we do this?
%\end{itemize}
%
%\subsection{Using exact normals in integrands of~$\int_{\Gamma_{h/m}^{2 \to 1}}$ (updated summary)}
%
%It was quite easy to update~\eqref{mtx_exact} such that the exact normals w.r.t. piecewise linear surface domain of integration are used. \texttt{spatch} (section~\ref{subsec:integrals}) has a member function that gives \textbf{physical} normals to its triangles straightaway. Thus implementation of updated quadratures boiled down to constructing~\texttt{GridFunction} object (described in section~\ref{subsec:app}) out of these physical normals. Details of the implementation can be found in commit~\texttt{\href{https://github.com/56th/drops/commit/dacc440587a3f1ea56186fd8c1ff6b6e3ea4b730}{dacc440}}.
%
%Comments:
%\begin{itemize}
%	\item It is also easy to extend this approach s.t. $\Gamma_{h/m}^1 \ne \Gamma_{h/m}^{2 \rightarrow 1}$  is used (please see section~\ref{subsec:not} and then figures~\ref{fig:phi_exact} and~\ref{fig:phi_inexact}). There is no difference between~$\Gamma_{h/2}^{2 \rightarrow 1}$ and $\Gamma_{h/2}^1$. For $m > 2$, there is no difference between~$\Gamma_{h/m}^{2 \rightarrow 1}$ and $\Gamma_{h/m}^1$ if $\phi \in P^2$. Right now~$\Gamma_{h/m}^{2 \rightarrow 1}$ is implemented.
%	\item It is \textbf{not} that easy to use the exact normal w.r.t. piecewise linear surface domain of integration in volume integrals. Our guess is that it shall not be a problem (we can leave~$\vect n_{\Gamma_h^2}$ there),
%	\item It is \textbf{not} easy to build~$I_{h}^n(\phi)$ for~$n > 2$. It is \textbf{not} implemented in DROPS as for now.
%\end{itemize}
%As for now, the matrices in~\eqref{mtx} can also be assembled as
%\begin{align}\begin{split}\label{mtx_exact_2}
%	\langle \vect A\,\vec{\vect u}, \vec{\vect v} \rangle &= 
%		\int^5_{\Gamma_{h/m}^{2 \to 1}} \big( 2\,E_{s,\,\textcolor{DarkGreen}{\Gamma_{h/m}^{2 \to 1}}}(\vect u) : E_{s,\,\textcolor{DarkGreen}{\Gamma_{h/m}^{2 \to 1}}}(\vect v) + \vect u\cdot\vect v + \tau\,(\vect u\cdot\vect n_{\Gamma_{h}^2})\,(\vect v\cdot\vect n_{\Gamma_{h}^2}) \big) \diff{s} \\
%	&
%		+ \rho_u \int^5_{\Omega_h^{\Gamma}} \frac{\partial \vect u}{\partial\vect n_{\Gamma_{h}^2}}\cdot\frac{\partial \vect v}{\partial\vect n_{\Gamma_{h}^2}} \diff{\vect x}, \quad \vect A \in \mathbb R^{n_{\vect A} \times n_{\vect A}},\\
%	\langle \vect B\,\vec{\vect u}, \vec{\vect q} \rangle &= 
%		\int^5_{\Gamma_{h/m}^{2 \to 1}} \nabla_{\textcolor{DarkGreen}{\Gamma_{h/m}^{2 \to 1}}} q \cdot \vect u \diff{s}, \quad \vect B \in \mathbb R^{n_{\vect S} \times n_{\vect A}},\\
%	\langle \vect M_0\,\vec{\vect p}, \vec{\vect q} \rangle &=
%		\int^5_{\Gamma_{h/m}^{2 \to 1}} p\,q \diff{s}, \quad \vect M_0 \in \mathbb R^{n_{\vect S} \times n_{\vect S}},\\
%	\langle \vect C_n\,\vec{\vect p}, \vec{\vect q} \rangle &=
%		\rho_p \int^5_{\Omega^{\Gamma}_h} \frac{\partial p}{\partial\vect n_{\Gamma_{h}^2}} \frac{\partial q}{\partial\vect n_{\Gamma_{h}^2}} \diff{\vect x}, \quad \vect C_n \in \mathbb R^{n_{\vect S} \times n_{\vect S}},\\
%	\langle \vect C_{\text{full}}\,\vec{\vect p}, \vec{\vect q} \rangle &=
%		\rho_p \int^5_{\Omega^{\Gamma}_h} \nabla p \cdot \nabla q \diff{\vect x}, \quad \vect C_{\text{full}} \in \mathbb R^{n_{\vect S} \times n_{\vect S}},
%\end{split}\end{align}
%Notations are as in~\eqref{mtx_exact}. Note that the ``$\tau$-term'' uses~$\vect n_{\Gamma_h^2}$. In order to switch between~\eqref{mtx_exact} and~\eqref{mtx_exact_2}, one modifies JSON input file: 
%\begin{lstlisting}
%// No_Bnd_Condition.json
%"Levelset": {
%// ...
%"NumbOfVirtualSubEdges" : 2,
%"UseExactNormals"       : "yes",
%// ...
%}
%\end{lstlisting}
%Here 2 corresponds to $m = 2$, and ``yes'' corresponds to~\eqref{mtx_exact_2}.
%
%\clearpage
%\begin{figure}[h!]
%	\par\bigskip
%	\centering
%	\begin{subfigure}{.5\linewidth}
%		\centering
%		\includegraphicsw[.55]{patches_2.png}
%	\end{subfigure}%
%	\begin{subfigure}{.5\linewidth}
%		\centering
%		\includegraphicsw[.55]{normals_2.png}
%	\end{subfigure}%
%	\par\bigskip
%	\begin{subfigure}{.5\linewidth}
%		\centering
%		\includegraphicsw[.55]{patches_4.png}
%	\end{subfigure}%
%	\begin{subfigure}{.5\linewidth}
%		\centering
%		\includegraphicsw[.55]{normals_4.png}
%	\end{subfigure}%
%	\par\bigskip
%	\caption{$\Gamma = \sphere$, $\phi(\vect x) = \|\vect x\|^2 - 1$, $h = 8.33\times10^{-1}$. Top-left: $\Gamma_{h/2}^{2 \rightarrow 1} = \Gamma_{h/2}^1$ (different color corresponds to a different \texttt{spatch} $\gamma \in \Gamma_{h/2}^{2 \rightarrow 1}$ as described in section~\ref{subsec:integrals}). Top-right: a patch~$\gamma \in \Gamma_{h/2}^{2 \rightarrow 1}$ and its normals. Bottom-left and bottom-right: same for~$\Gamma_{h/4}^{2 \rightarrow 1} = \Gamma_{h/4}^1$. \textbf{Note that since~$\phi \in P^2$, we have that~$\Gamma_{h/m}^{2 \rightarrow 1} = \Gamma_{h/m}^1 \rightarrow \Gamma$ as $m \rightarrow \infty$ for fixed~$h$}}
%	\label{fig:phi_exact}		
%\end{figure}
%\vfill
%\begin{figure}[h!]
%	\centering
%	\begin{subfigure}{.5\linewidth}
%		\centering
%		\includegraphicsw[.55]{patches_2_inexact.png}
%	\end{subfigure}%
%	\begin{subfigure}{.5\linewidth}
%		\centering
%		\includegraphicsw[.55]{patches_4_inexact.png}
%	\end{subfigure}%
%	\par\bigskip
%	\caption{$\Gamma = \sphere$, $\phi(\vect x) = \|\vect x\|^{1/2} - 1$, $h = 8.33\times10^{-1}$. Left: $\Gamma_{h/2}^{2 \rightarrow 1} = \Gamma_{h/2}^1$ (different color corresponds to a different \texttt{spatch} $\gamma \in \Gamma_{h/2}^{2 \rightarrow 1}$ as described in section~\ref{subsec:integrals}). Right: same for~$\Gamma_{h/4}^{2 \rightarrow 1} \ne \Gamma_{h/4}^1$. \textbf{Note that since~$\phi \not\in \bar{P}^2_h$, we have that~$\Gamma_{h/m}^{2 \rightarrow 1} \ne \Gamma_{h/m}^1$ for~$m > 2$, and $\Gamma_{h/m}^{2 \rightarrow 1} \rightarrow \Gamma_h^2 \ne \Gamma$ as $m \rightarrow \infty$ for fixed~$h$}}
%	\label{fig:phi_inexact}		
%\end{figure}
%\clearpage
%
%\subsection{Quadrature rules for the error computation}\label{subsec:err}
%
%When we first tried to test convergence for~\eqref{mtx_exact_2} in section~\ref{sec:conv}, we noticed that the~$\HOneSpace$-error of the velocity decays much slower than expected, whereas its~$\LTwoSpace$-error behaves as expected. Note that~$\HOneSpace$-error (for e.g. $\vect P_2$\,--\,$P_1$ FE) can be computed as~$\langle \vect w, \vect A_s\,\vect w \rangle^{1/2}$, $\vect w \coloneqq$ vector of d.o.f. corresponding to $\vect P^2_h$ interpolant $I_h^2(\vect u^e) - \vect u_h$, $\vect A_s \coloneqq$ matrix corresponding to the first term of~$\vect A$ in~\eqref{mtx_exact_2}. Thus the errors are approximated as
%\begin{align}\begin{split}
%	\| \vect u - \vect u_h \|_{\HOneSpace} &= \| I^k_h(\vect u^e) - \vect u_h \|_{\HOneSpace[\Gamma_{h/m}^{2 \to 1}]} + O(h^{k}), \\
%	\| \vect u - \vect u_h \|_{\LTwoSpace} &= \| I^k_h(\vect u^e) - \vect u_h \|_{\LTwoSpace[\Gamma_{h/m}^{2 \to 1}]} + O(h^{k+1}), \\
%	\| p - p_h \|_{\LTwoSpace} &= \| I^1_h(p^e) - p_h \|_{\LTwoSpace[\Gamma_{h/m}^{2 \to 1}]} + O(h^2)
%\end{split}\end{align}
%for~$m > 1$. Here $k = 1$ for~$\vect P_1$\,--\,$P_1$ FEM and $k = 2$ for~$\vect P_2$\,--\,$P_1$. For consistent penalty approach matrix~$\vect A_s$ is computed as in~\eqref{mtx_exact_2_cons}.
%
%\textbf{Interestingly enough}, DROPS implementation did not use the assembled matrices to compute errors (and normals that are \textit{different} from the ones in~$\vect A_s$ were used). We corrected it in commit~\texttt{\href{https://github.com/56th/drops/commit/68443b0a678447ba8b3e7e0af1621e6cd402e5d1}{68443b0}}:
%\begin{lstlisting}
%// surfnavierstokes.cpp
%// ...
%VectorCL vSolMinusV = vSol.Data - v.Data, pSolMinusP = pSol.Data - p.Data;
%auto velL2          = sqrt(dot(v.Data, M.Data * v.Data));
%auto velNormalL2    = sqrt(dot(v.Data, S.Data * v.Data));
%auto velH1err       = sqrt(dot(vSolMinusV, A.Data * vSolMinusV));
%auto velL2err       = sqrt(dot(vSolMinusV, M.Data * vSolMinusV));
%auto preL2          = sqrt(dot(p.Data, Schur.Data * p.Data));
%auto preL2err       = sqrt(dot(pSolMinusP, Schur.Data * pSolMinusP));
%// ...
%\end{lstlisting}

\bibliographystyle{plain}
\bibliography{bibl}

\end{document}