\documentclass[12pt]{article}

\usepackage{mathtools}
\usepackage{amssymb}
\usepackage{amsthm}

\usepackage[dvipsnames, table]{xcolor}
\colorlet{DarkRed}{Red!90!black}
\colorlet{LightRed}{Red!10!white}
\colorlet{DarkGreen}{Green!50!black}
\colorlet{LightGreen}{Green!10!white}
\usepackage{colortbl} % https://texblog.org/2011/04/19/highlight-table-rowscolumns-with-color/
% links
\usepackage{hyperref}
\hypersetup{
	colorlinks,
	linkcolor={DarkRed},
	citecolor={DarkRed},
	urlcolor={blue}
}

\usepackage{geometry}
\newgeometry{
	left=1cm, right=1cm, top=.8cm, bottom=.8cm,
	includefoot, heightrounded
}

\usepackage[parfill]{parskip} % https://tex.stackexchange.com/a/16703/135296

% sub figures / grids of pictures
\usepackage{subcaption}
\usepackage{graphicx}
\graphicspath{{img/}} % includegraphics path
% \usepackage[export]{adjustbox} % https://tex.stackexchange.com/questions/20640/how-to-add-border-for-an-image
\newcommand{\includegraphicsw}[2][1.]{\includegraphics[width=#1\linewidth]{#2}}
\newcommand{\svginput}[1]{\input{img/#1}} % pdf_tex path
\newcommand{\svginputw}[2][\linewidth]{\def\svgwidth{#1}\input{img/#2}} % pdf_tex path

% tables
\usepackage{multirow}
\usepackage{hhline}

% bold for everything
\usepackage{bm}
\newcommand{\vect}[1]{\boldsymbol{\mathbf{#1}}}

% differentials
\newcommand*\diff{\mathop{}\!\mathrm{d}}
\newcommand*\Diff[1]{\mathop{}\!\mathrm{d^#1}}

\DeclareMathOperator{\Div}{div}
\newcommand{\sphere}{{\Gamma_{\text{sph}}}}
\newcommand{\tor}{{\Gamma_{\text{tor}}}}

\newcommand{\LTwoSpace}[1][\Gamma]{{\mathbb L^2\left({#1}\right)}}
\newcommand{\HOneSpace}[1][\Gamma]{{\mathbb H^1\left({#1}\right)}}

\usepackage{listings}
\definecolor{mygreen}{rgb}{0,0.6,0}
\lstset{
	language=C++,
	basicstyle=\footnotesize\ttfamily,
	breaklines=true,
	commentstyle=\color{mygreen},
	frame=l,
	xleftmargin=5pt,
	tabsize=2,
	belowskip=-1pt
} 

\title{Some computational results for $\vect{P}_1$\,--\,$P_1$ and $\vect{P}_2$\,--\,$P_1$ Trace\,FEM for the surface Stokes problem}
\author{
	Alexander Zhiliakov\thanks{Department of Mathematics, University of Houston, Houston, Texas 77204 (alex@math.uh.edu).}
}

\begin{document}
	
\maketitle

\tableofcontents
	
\let\oldtabular\tabular
\renewcommand{\tabular}[1][1.5]{\def\arraystretch{#1}\oldtabular}

\section{Inf-sup stability: generalized pressure Schur complement eigenvalues}

\subsection{Bilinear forms and matrices}

We set $n_{\vect A}$ to be the number of velocity d.o.f. and $n_{\vect S}$ to be the number of pressure d.o.f. Vector stiffness, divergence, pressure mass, normal stabilization, and full stabilization matrices resulting from Trace\,FEM discretization of the surface Stokes problem~\cite{surfstokes} are defined via
\begin{align}\begin{split}\label{mtx}
	\langle \vect A\,\vec{\vect u}, \vec{\vect v} \rangle &\approx 
		\int_{\Gamma} \big( E_s(\vect u) : E_s(\vect v) + \vect u\cdot\vect v + \tau\,(\vect u\cdot\vect n)\,(\vect v\cdot\vect n) \big) \diff{s} + 
		\rho_u \int_{\Omega_h^{\Gamma}} \frac{\partial \vect u}{\partial\vect n}\cdot\frac{\partial \vect v}{\partial\vect n} \diff{\vect x}, \quad \vect A \in \mathbb R^{n_{\vect A} \times n_{\vect A}},\\
	\langle \vect B\,\vec{\vect u}, \vec{\vect q} \rangle &\approx 
		-\int_{\Gamma} q\,\Div_{\Gamma} \vect u \diff{s}, \quad \vect B \in \mathbb R^{n_{\vect S} \times n_{\vect A}},\\
	\langle \vect M_0\,\vec{\vect p}, \vec{\vect q} \rangle &\approx
		\int_{\Gamma} p\,q \diff{s}, \quad \vect M_0 \in \mathbb R^{n_{\vect S} \times n_{\vect S}},\\
	\langle \vect C_n\,\vec{\vect p}, \vec{\vect q} \rangle &\approx
		\rho_p \int_{\Omega^{\Gamma}_h} \frac{\partial p}{\partial\vect n} \frac{\partial q}{\partial\vect n} \diff{\vect x}, \quad \vect C_n \in \mathbb R^{n_{\vect S} \times n_{\vect S}},\\
	\langle \vect C_{\text{full}}\,\vec{\vect p}, \vec{\vect q} \rangle &\approx
		\rho_p \int_{\Omega^{\Gamma}_h} \nabla p \cdot \nabla q \diff{\vect x}, \quad \vect C_{\text{full}} \in \mathbb R^{n_{\vect S} \times n_{\vect S}},		 
\end{split}\end{align}
respectively. We use notations as in~\cite{surfstokes}, in particular, $\Omega_\Gamma^h$ is the domain consisting of tetrahedra cut by $\Gamma$. Here~$\vec{\vect u}$ denotes a vector of d.o.f. corresponding to a FE interpolant~$\vect u$ (analogously for $\vec{\vect p}$ and $p$). See~\eqref{mtx_exact} and~\eqref{mtx_exact_2} for the computational details. Mesh-dependent parameters are set as
\begin{equation}
	\tau = h^{-2}, \quad \rho_u = \rho_p = h,
\end{equation}
and $h$ is the typical mesh size for tetrahedra from~$\Omega^{\Gamma}_h$. $\Gamma$ is chosen either as the unit sphere or torus, $\Gamma = \sphere$ or $\Gamma = \tor$ (see Figure~\ref{fig:gamma}).

We also define matrices 
\begin{equation}
	\vect C_0 \coloneqq \vect 0,\quad
	\vect M_n \coloneqq \vect M_0 + \vect C_n,\quad
	\vect M_{\text{full}} \coloneqq \vect M_0 + \vect C_{\text{full}}.
\end{equation}
We are interested in (generalized) extreme eigenvalues of the pressure Schur complement matrices
\begin{align}\label{schur}
	\vect S_0 \coloneqq \vect B\,\vect A^{-1}\,\vect B^{T},\quad
	\vect S_n \coloneqq \vect S_0 + \vect C_n,\quad
	\vect S_{\text{full}} \coloneqq \vect S_0 + \vect C_{\text{full}},
\end{align}
i.e. in solving
\begin{equation}\label{problem}
	\vect S_\star\,\vect x = \lambda\,\vect M_\star\,\vect x,
\end{equation}
where ``$\star$'' stands for ``$0$,'' ``$n$,'' or ``full.'' We denote by~$0 = \lambda_1 < \lambda_2 \le \dots \le \lambda_{n_{\vect S}} = O(1)$ the spectrum of~\eqref{problem}.

\subsection{Solution description}

Computing $\vect A^{-1}$ in~\eqref{schur} becomes troublesome already for $h = 5.21\times10^{-2}$ ($n_{\vect A} = 32736$ for $\vect u \in \vect P_1$ FE space): although $\vect A$ is sparse, $\vect A^{-1}$ is dense and consumes 8.5+ GB in double-precision arithmetic. A quick research \href{https://mathematica.stackexchange.com/questions/189620/matrix-free-arnoldi-method-for-eigensystems}{showed} that \texttt{Mathematica} has no built-in matrix-free eigenvalue routines. \texttt{Intel MKL}'s FEAST algorithm for computing (generalized) eigenvalues in an interval \href{https://software.intel.com/sites/default/files/mkl-2019-developer-reference-c.pdf#_OPENTOPIC_TOC_PROCESSING_d62e853651}{is suitable for matrix-free implementations}; however, it requires some expensive operations to be implemented (e.g. matrix-matrix multiplications $\vect Y \leftarrow \vect S_\star\,\vect X$, $\vect Y \leftarrow \vect M_\star\,\vect X$ and approximating the action of inverses in the form $\vect y \leftarrow (\sigma\,\vect M_\star - \vect S_\star)^{-1}\,\vect x$).

Taking this into account, instead of~\eqref{problem} we consider a perturbed\footnotemark{} problem
\begin{equation}\label{problem_pert}
	\underbrace{\begin{bmatrix}
		\vect A & \phantom{-}\vect B^T \\
		\vect B & -\vect C_\star \\
	\end{bmatrix}}_{\mathcal A_\star \coloneqq}
	\begin{bmatrix}
		\vect x \\
		\vect y
	\end{bmatrix}
	=
	\mu
	\underbrace{\begin{bmatrix}
		\epsilon\,\vect A & \\
		& \vect M_\star
	\end{bmatrix}}_{\mathcal M^\epsilon_\star \coloneqq}
	\begin{bmatrix}
		\vect x \\
		\vect y
	\end{bmatrix}
\end{equation}
with $0 < \epsilon \ll 1$. For $\mathcal A_0$ and $\mathcal M^\epsilon_0$ we have
\begin{equation}
	\mu = -\lambda + o(1)\quad\text{or}\quad\epsilon^{-1} + \lambda + o(1),\qquad\epsilon \rightarrow 0.
\end{equation}
This makes it easy to pick only ``correct'' eigenvalues. To ease the computation further we replace the $(1, 1)$-block of~$\mathcal M^\epsilon_\star$ with $\epsilon\,\vect I$. 

To make sure that results are consistent we solve~\eqref{problem_pert} for~$\epsilon = 10^{-5}$ and~$\epsilon = 10^{-6}$; for the coarse mesh levels we also check that the dense solver for~\eqref{problem} and the iterative one for~\eqref{problem_pert} give solutions that coincide.  

\footnotetext{The majority of generalized eigenvalue solvers require left-hand-side matrix to be Hermitian and right-hand-side matrix to be Hermitian \textbf{positive definite}; that's why we need to introduce $\epsilon > 0$.}     

\subsection{Numerical results: dependency of the spectrum on the mesh size}

\begin{figure}[h]
	\centering
	\begin{subfigure}{.33\linewidth}
		\centering
		\includegraphicsw[.9]{{lvl1.cropped}.png}
		\caption{$h = 8.33\times10^{-1}$}
	\end{subfigure}%
	\begin{subfigure}{.33\linewidth}
		\centering
		\includegraphicsw[.9]{{lvl2.cropped}.png}
		\caption{$h = 4.17\times10^{-1}$}
	\end{subfigure}%
	\begin{subfigure}{.33\linewidth}
		\centering
		\includegraphicsw[.9]{{lvl3.cropped}.png}
		\caption{$h = 2.08\times10^{-1}$}
	\end{subfigure}
	\par\bigskip
	\begin{subfigure}{.33\linewidth}
		\centering
		\includegraphicsw[.9]{{tor_lvl3.cropped}.png}
		\caption{$h = 2.08\times10^{-1}$}
	\end{subfigure}%
	\begin{subfigure}{.33\linewidth}
		\centering
		\includegraphicsw[.9]{{tor_lvl4.cropped}.png}
		\caption{$h = 1.04\times10^{-1}$}
	\end{subfigure}%
	\begin{subfigure}{.33\linewidth}
		\centering
		\includegraphicsw[.9]{{tor_lvl5.cropped}.png}
		\caption{$h = 5.21\times10^{-2}$}
	\end{subfigure}
	\caption{First three mesh levels for~$\sphere$ (top) and $\tor$ (bottom)}
	\label{fig:gamma}		
\end{figure}

\clearpage

\begin{table}[h]
	\centering\small
	\caption{Spectrum of~\eqref{problem} for $\vect P_1$\,--\,$P_1$} 
	\label{tab:p1p1}
	\begin{subtable}{1.\linewidth}
		\centering
		\caption{$\Gamma = \sphere$}
		\label{tab:p1p1:sph}
		\begin{tabular}[1.3]{|c|c|c|c|c|c|c|c|c|}
			\hline
			\multirow{2}{*}{$h$} & \multirow{2}{*}{$n_{\vect A}$} & \multirow{2}{*}{$n_{\vect S}$} & \multicolumn{2}{c|}{$\vect S_0$} & \multicolumn{2}{c|}{$\vect S_n$} & \multicolumn{2}{c|}{$\vect S_{\text{full}}$} \\ 
			\cline{4-9}
			& & & $\lambda_2$ & $\lambda_{n_{\vect S}}$ & $\lambda_2$ & $\lambda_{n_{\vect S}}$ & $\lambda_2$ & $\lambda_{n_{\vect S}}$ \\ 
			\hline
			$8.33\times	10^{-1}$	&	$153$	&	$51$	&	$1.32\times	10^{-2}$	&	$1.42$	&	$7.48\times	10^{-1}$	&	$1.13$	&	$9.58\times	10^{-1}$	&	$1.06$	\\ \hline
$4.17\times	10^{-1}$	&	$570$	&	$190$	&	$5.12\times	10^{-3}$	&	$1.04$	&	$5.77\times	10^{-1}$	&	$1.$	&	$8.54\times	10^{-1}$	&	$1.$	\\ \hline
$2.08\times	10^{-1}$	&	$1992$	&	$664$	&	$4.4\times	10^{-3}$	&	$7.93\times	10^{-1}$	&	$3.87\times	10^{-1}$	&	$1.$	&	$6.71\times	10^{-1}$	&	$1.$	\\ \hline
$1.04\times	10^{-1}$	&	$8292$	&	$2764$	&	$2.01\times	10^{-3}$	&	$7.75\times	10^{-1}$	&	$2.19\times	10^{-1}$	&	$1.$	&	$5.82\times	10^{-1}$	&	$1.$	\\ \hline
$5.21\times	10^{-2}$	&	$32736$	&	$10912$	&	$6.04\times	10^{-5}$	&	$9.81\times	10^{-1}$	&	$1.17\times	10^{-1}$	&	$1.$	&	$5.37\times	10^{-1}$	&	$1.$	\\ \hline
$2.6\times	10^{-2}$	&	$131592$	&	$43864$	&	$3.53\times	10^{-5}$	&	$8.67\times	10^{-1}$	&	$5.72\times	10^{-2}$	&	$1.$	&	$5.16\times	10^{-1}$	&	$1.$	\\ \hline
$1.3\times	10^{-2}$	&	$525864$	&	$175288$	&	$2.16\times	10^{-6}$	&	$7.34\times	10^{-1}$	&	$2.84\times	10^{-2}$	&	$1.$	&	$5.04\times	10^{-1}$	&	$1.$	\\ \hline
			%		\multirow{2}{*}{$h$} & \multirow{2}{*}{$n_{\vect A}$} & \multirow{2}{*}{$n_{\vect S}$} & \multicolumn{2}{c|}{$\vect S_0$} & \multicolumn{2}{c|}{$\vect S_n$} & \multicolumn{2}{c|}{$\vect S_{\text{full}}$} \\ 
			%		\cline{4-9}
			%		& & & $r_2$ & $r_{n_{\vect S}}$ & $r_2$ & $r_{n_{\vect S}}$ & $r_2$ & $r_{n_{\vect S}}$ \\ 
			%		\hline
			%		$2.08\times	10^{-1}$	&	$972$	&	$324$	&	$3.\times	10^{-10}$	&	$3.\times	10^{-17}$	&	$9.\times	10^{-13}$	&	$5.\times	10^{-8}$	&	$1.\times	10^{-13}$	&	$3.\times	10^{-7}$	\\ \hline
$1.04\times	10^{-1}$	&	$4740$	&	$1580$	&	$1.\times	10^{-15}$	&	$2.\times	10^{-18}$	&	$5.\times	10^{-11}$	&	$5.\times	10^{-8}$	&	$4.\times	10^{-10}$	&	$8.\times	10^{-8}$	\\ \hline
$5.21\times	10^{-2}$	&	$19704$	&	$6568$	&	$3.\times	10^{-15}$	&	$1.\times	10^{-18}$	&	$5.\times	10^{-14}$	&	$1.\times	10^{-5}$	&	$1.\times	10^{-10}$	&	$2.\times	10^{-4}$	\\ \hline
$2.6\times	10^{-2}$	&	$80808$	&	$26936$	&	$3.\times	10^{-19}$	&	$4.\times	10^{-19}$	&	$5.\times	10^{-13}$	&	$7.\times	10^{-5}$	&	$9.\times	10^{-13}$	&	$7.\times	10^{-4}$	\\ \hline
$1.3\times	10^{-2}$	&	$327036$	&	$109012$	&	$2.\times	10^{-20}$	&	$2.\times	10^{-22}$	&	$9.\times	10^{-14}$	&	$2.\times	10^{-4}$	&	$1.\times	10^{-12}$	&	$7.\times	10^{-4}$	\\ \hline
		\end{tabular}
	\end{subtable}%
	\vskip 3mm
	\begin{subtable}{1.\linewidth}
		\centering
		\caption{$\Gamma = \tor$}
		\label{tab:p1p1:tor}
		\begin{tabular}[1.3]{|c|c|c|c|c|c|c|c|c|}
			\hline
			\multirow{2}{*}{$h$} & \multirow{2}{*}{$n_{\vect A}$} & \multirow{2}{*}{$n_{\vect S}$} & \multicolumn{2}{c|}{$\vect S_0$} & \multicolumn{2}{c|}{$\vect S_n$} & \multicolumn{2}{c|}{$\vect S_{\text{full}}$} \\ 
			\cline{4-9}
			& & & $\lambda_2$ & $\lambda_{n_{\vect S}}$ & $\lambda_2$ & $\lambda_{n_{\vect S}}$ & $\lambda_2$ & $\lambda_{n_{\vect S}}$ \\ 
			\hline
			$2.08\times	10^{-1}$	&	$972$	&	$324$	&	$5.04\times	10^{-2}$	&	$4.93$	&	$2.84\times	10^{-1}$	&	$1.35$	&	$3.64\times	10^{-1}$	&	$1.19$	\\ \hline
$1.04\times	10^{-1}$	&	$4740$	&	$1580$	&	$2.99\times	10^{-3}$	&	$3.83$	&	$1.58\times	10^{-1}$	&	$1.02$	&	$3.35\times	10^{-1}$	&	$1.01$	\\ \hline
			%		\multirow{2}{*}{$h$} & \multirow{2}{*}{$n_{\vect A}$} & \multirow{2}{*}{$n_{\vect S}$} & \multicolumn{2}{c|}{$\vect S_0$} & \multicolumn{2}{c|}{$\vect S_n$} & \multicolumn{2}{c|}{$\vect S_{\text{full}}$} \\ 
			%		\cline{4-9}
			%		& & & $r_2$ & $r_{n_{\vect S}}$ & $r_2$ & $r_{n_{\vect S}}$ & $r_2$ & $r_{n_{\vect S}}$ \\ 
			%		\hline
			%		$2.08\times	10^{-1}$	&	$972$	&	$324$	&	$3.\times	10^{-10}$	&	$3.\times	10^{-17}$	&	$9.\times	10^{-13}$	&	$5.\times	10^{-8}$	&	$1.\times	10^{-13}$	&	$3.\times	10^{-7}$	\\ \hline
$1.04\times	10^{-1}$	&	$4740$	&	$1580$	&	$1.\times	10^{-15}$	&	$2.\times	10^{-18}$	&	$5.\times	10^{-11}$	&	$5.\times	10^{-8}$	&	$4.\times	10^{-10}$	&	$8.\times	10^{-8}$	\\ \hline
$5.21\times	10^{-2}$	&	$19704$	&	$6568$	&	$3.\times	10^{-15}$	&	$1.\times	10^{-18}$	&	$5.\times	10^{-14}$	&	$1.\times	10^{-5}$	&	$1.\times	10^{-10}$	&	$2.\times	10^{-4}$	\\ \hline
$2.6\times	10^{-2}$	&	$80808$	&	$26936$	&	$3.\times	10^{-19}$	&	$4.\times	10^{-19}$	&	$5.\times	10^{-13}$	&	$7.\times	10^{-5}$	&	$9.\times	10^{-13}$	&	$7.\times	10^{-4}$	\\ \hline
$1.3\times	10^{-2}$	&	$327036$	&	$109012$	&	$2.\times	10^{-20}$	&	$2.\times	10^{-22}$	&	$9.\times	10^{-14}$	&	$2.\times	10^{-4}$	&	$1.\times	10^{-12}$	&	$7.\times	10^{-4}$	\\ \hline
		\end{tabular}
	\end{subtable}
\end{table}
\vfill
\begin{figure}[h]
	\centering\small
	\begin{subfigure}{.49\linewidth}
		\centering
		\includegraphicsw{sphere_2_P1P1.png}
		%\caption{$\vect P_1$\,--\,$P_1$ for $\sphere$}
	\end{subfigure}%
	\hfill
	\begin{subfigure}{.49\linewidth}
		\centering
		\includegraphicsw{torus_P1P1.png}
		%\caption{$\vect P_1$\,--\,$P_1$ for $\tor$}
	\end{subfigure}
	\caption{Log-log plot of~$\lambda_2$ for Tables~\ref{tab:p1p1}\subref{tab:p1p1:sph} (left) and~\ref{tab:p1p1}\subref{tab:p1p1:tor} (right)}
\end{figure}
\vfill

\clearpage

\begin{table}[h]
	\centering\small
	\caption{Spectrum of~\eqref{problem} for $\vect P_2$\,--\,$P_1$} 
	\label{tab:p2p1}
	\begin{subtable}{1.\linewidth}
		\centering
		\caption{$\Gamma = \sphere$}
		\label{tab:p2p1:sph}
		\begin{tabular}[1.3]{|c|c|c|c|c|c|c|c|c|}
			\hline
			\multirow{2}{*}{$h$} & \multirow{2}{*}{$n_{\vect A}$} & \multirow{2}{*}{$n_{\vect S}$} & \multicolumn{2}{c|}{$\vect S_0$} & \multicolumn{2}{c|}{$\vect S_n$} & \multicolumn{2}{c|}{$\vect S_{\text{full}}$} \\ 
			\cline{4-9}
			& & & $\lambda_2$ & $\lambda_{n_{\vect S}}$ & $\lambda_2$ & $\lambda_{n_{\vect S}}$ & $\lambda_2$ & $\lambda_{n_{\vect S}}$ \\ 
			\hline
			$8.33\times	10^{-1}$	&	$789$	&	$51$	&	$3.22\times	10^{-1}$	&	$1.73$	&	$8.27\times	10^{-1}$	&	$1.17$	&	$9.68\times	10^{-1}$	&	$1.07$	\\ \hline
$4.17\times	10^{-1}$	&	$3240$	&	$190$	&	$9.17\times	10^{-2}$	&	$1.08$	&	$6.45\times	10^{-1}$	&	$1.$	&	$8.56\times	10^{-1}$	&	$1.$	\\ \hline
$2.08\times	10^{-1}$	&	$11718$	&	$664$	&	$1.78\times	10^{-1}$	&	$8.31\times	10^{-1}$	&	$5.49\times	10^{-1}$	&	$1.$	&	$6.75\times	10^{-1}$	&	$1.$	\\ \hline
$1.04\times	10^{-1}$	&	$48762$	&	$2764$	&	$1.04\times	10^{-1}$	&	$8.35\times	10^{-1}$	&	$5.14\times	10^{-1}$	&	$1.$	&	$5.82\times	10^{-1}$	&	$1.$	\\ \hline
$5.21\times	10^{-2}$	&	$193014$	&	$10912$	&	$2.99\times	10^{-3}$	&	$9.89\times	10^{-1}$	&	$5.02\times	10^{-1}$	&	$1.$	&	$5.34\times	10^{-1}$	&	$1.$	\\ \hline
$2.6\times	10^{-2}$	&	$775998$	&	$43864$	&	$1.17\times	10^{-3}$	&	$7.9\times	10^{-1}$	&	$4.96\times	10^{-1}$	&	$1.$	&	$5.17\times	10^{-1}$	&	$1.$	\\ \hline
		\end{tabular}
	\end{subtable}%
	\vskip 3mm
	\begin{subtable}{1.\linewidth}
		\centering
		\caption{$\Gamma = \tor$}
		\label{tab:p2p1:tor}
		\begin{tabular}[1.3]{|c|c|c|c|c|c|c|c|c|}
			\hline
			\multirow{2}{*}{$h$} & \multirow{2}{*}{$n_{\vect A}$} & \multirow{2}{*}{$n_{\vect S}$} & \multicolumn{2}{c|}{$\vect S_0$} & \multicolumn{2}{c|}{$\vect S_n$} & \multicolumn{2}{c|}{$\vect S_{\text{full}}$} \\ 
			\cline{4-9}
			& & & $\lambda_2$ & $\lambda_{n_{\vect S}}$ & $\lambda_2$ & $\lambda_{n_{\vect S}}$ & $\lambda_2$ & $\lambda_{n_{\vect S}}$ \\ 
			\hline
			$2.08\times	10^{-1}$	&	$5184$	&	$324$	&	$9.92\times	10^{-2}$	&	$3.89$	&	$1.33\times	10^{-1}$	&	$1.37$	&	$1.75\times	10^{-1}$	&	$1.19$	\\ \hline
$1.04\times	10^{-1}$	&	$27906$	&	$1580$	&	$1.46\times	10^{-2}$	&	$4.35$	&	$2.84\times	10^{-1}$	&	$1.04$	&	$2.99\times	10^{-1}$	&	$1.02$	\\ \hline
$5.21\times	10^{-2}$	&	$116568$	&	$6568$	&	$6.08\times	10^{-3}$	&	$4.85$	&	$3.19\times	10^{-1}$	&	$1.01$	&	$3.24\times	10^{-1}$	&	$1.01$	\\ \hline
$2.6\times	10^{-2}$	&	$477660$	&	$26936$	&	$1.36\times	10^{-3}$	&	$4.92$	&	$3.14\times	10^{-1}$	&	$1.01$	&	$3.16\times	10^{-1}$	&	$1.$	\\ \hline
		\end{tabular}
	\end{subtable}
\end{table}
\vfill
\begin{figure}[h]
	\centering\small
	\begin{subfigure}{.49\linewidth}
		\centering
		\includegraphicsw{sphere_2_P2P1.png}
	\end{subfigure}%
	\hfill
	\begin{subfigure}{.49\linewidth}
		\centering
		\includegraphicsw{torus_P2P1.png}
	\end{subfigure}
	\caption{Log-log plot of~$\lambda_2$ for Tables~\ref{tab:p2p1}\subref{tab:p2p1:sph} (left) and~\ref{tab:p2p1}\subref{tab:p2p1:tor} (right)}
\end{figure}
\vfill

\clearpage

\subsection{Numerical results: sensitivity of the spectrum to levelset shifts}

In this section we investigate the sensitivity of the spectrum to levelset shifts
\begin{equation}\label{shift}
	\Gamma \mapsto \Gamma + \alpha\,\vect s,
\end{equation}
for some $\alpha \in \mathbb R$ and $\vect s \in \mathbb R^3$, $\|\vect s\| = 1$. 

We construct the bulk mesh~$\Omega_h^\Gamma$ and then perform the assembly of matrices~\eqref{mtx} using the shifted levelset~\eqref{shift}. That is, the refinement of~$\Omega_h^\Gamma$ is performed using~$\Gamma$, not~$\Gamma + \alpha\,\vect s$, and~$\Omega_h^{\Gamma + \alpha\,\vect s}$ is never constructed. We choose $\alpha \in [0, h]$ to guarantee the appearance of ``small cuts'' in~$\Omega_h^\Gamma$.

\begin{figure}[h]
	\centering
	\begin{subfigure}{.5\linewidth}
		\centering
		\includegraphicsw[.7]{{shift_0.0.cropped}.png}
		\caption{$\sphere$}
	\end{subfigure}%
	\begin{subfigure}{.5\linewidth}
		\centering
		\includegraphicsw[.7]{{shift_0.2.cropped}.png}
		\caption{$\sphere + \alpha\,\vect s$}
	\end{subfigure}%
	\caption{The unit sphere (left) and the shifted unit sphere (right). Here $\vect s = (0, 1, 1)^T/\sqrt{2}$, $\alpha = 0.2$, and $h = 2.08\times10^{-1}$. The bulk mesh~$\Omega_\Gamma^h$ is computed for~$\sphere$ and then used for~$\sphere + \alpha\,\vect s$}
	\label{fig:shift}		
\end{figure}

\begin{table}[h]
	\centering
	\caption{Spectrum of~\eqref{problem} for perturbed levelset $\sphere + \alpha\,\vect s$. Here $\vect s = (1, 1, 1)^T/\sqrt{3}$, $h = 1.04\times10^{-1}$} 
	\label{tab:p1p1_shift_h=0.104167}
	\small
	\begin{subtable}{1.\linewidth}
		\centering
		\caption{$\vect P_1$\,--\,$P_1$}
		\begin{tabular}[1.3]{|c|c|c|c|c|c|c|}
			\hline
			\multirow{2}{*}{Surface} & \multicolumn{2}{c|}{$\vect S_0$} & \multicolumn{2}{c|}{$\vect S_n$} & \multicolumn{2}{c|}{$\vect S_{\text{full}}$} \\ 
			\cline{2-7}
			& $\lambda_2$ & $\lambda_{n_{\vect S}}$ & $\lambda_2$ & $\lambda_{n_{\vect S}}$ & $\lambda_2$ & $\lambda_{n_{\vect S}}$ \\ 
			\hline
			$\sphere$\phantom{ + 0.1\,h$\,\vect s$}	&	$2.006\times	10^{-3}$	&	$7.79\times	10^{-1}$	&	$2.19\times	10^{-1}$	&	$1.$	&	$5.818\times	10^{-1}$	&	$1.$	\\ \hline
$\sphere + 0.1\,h\,\vect s$	&	$4.832\times	10^{-4}$	&	$8.01\times	10^{-1}$	&	$2.195\times	10^{-1}$	&	$1.$	&	$5.818\times	10^{-1}$	&	$1.$	\\ \hline
$\sphere + 0.3\,h\,\vect s$	&	$7.278\times	10^{-4}$	&	$8.17\times	10^{-1}$	&	$2.203\times	10^{-1}$	&	$1.$	&	$5.818\times	10^{-1}$	&	$1.$	\\ \hline
$\sphere + 0.5\,h\,\vect s$	&	$3.121\times	10^{-4}$	&	$8.67\times	10^{-1}$	&	$2.221\times	10^{-1}$	&	$1.$	&	$5.82\times	10^{-1}$	&	$1.$	\\ \hline
$\sphere + 0.7\,h\,\vect s$	&	$1.438\times	10^{-3}$	&	$1.51$	&	$2.254\times	10^{-1}$	&	$1.$	&	$5.82\times	10^{-1}$	&	$1.$	\\ \hline
$\sphere + \phantom{0.1}\,h\,\vect s$	&	$1.79\times	10^{-3}$	&	$2.07$	&	$2.332\times	10^{-1}$	&	$1.$	&	$5.827\times	10^{-1}$	&	$1.$	\\ \hline
		\end{tabular}
	\end{subtable}
	\vskip 4mm
	\begin{subtable}{1.\linewidth}
		\centering
		\caption{$\vect P_2$\,--\,$P_1$}
		\begin{tabular}[1.3]{|c|c|c|c|c|c|c|}
			\hline
			\multirow{2}{*}{Surface} & \multicolumn{2}{c|}{$\vect S_0$} & \multicolumn{2}{c|}{$\vect S_n$} & \multicolumn{2}{c|}{$\vect S_{\text{full}}$} \\ 
			\cline{2-7}
			& $\lambda_2$ & $\lambda_{n_{\vect S}}$ & $\lambda_2$ & $\lambda_{n_{\vect S}}$ & $\lambda_2$ & $\lambda_{n_{\vect S}}$ \\ 
			\hline
			$\sphere$\phantom{ + 0.1\,h$\,\vect s$}	&	$1.041\times	10^{-1}$	&	$8.35\times	10^{-1}$	&	$5.138\times	10^{-1}$	&	$1.$	&	$5.841\times	10^{-1}$	&	$1.$	\\ \hline
$\sphere + 0.1\,h\,\vect s$	&	$1.705\times	10^{-3}$	&	$8.58\times	10^{-1}$	&	$5.138\times	10^{-1}$	&	$1.$	&	$5.841\times	10^{-1}$	&	$1.$	\\ \hline
$\sphere + 0.3\,h\,\vect s$	&	$2.293\times	10^{-3}$	&	$8.81\times	10^{-1}$	&	$5.137\times	10^{-1}$	&	$1.$	&	$5.841\times	10^{-1}$	&	$1.$	\\ \hline
$\sphere + 0.5\,h\,\vect s$	&	$5.63\times	10^{-3}$	&	$9.35\times	10^{-1}$	&	$5.138\times	10^{-1}$	&	$1.$	&	$5.844\times	10^{-1}$	&	$1.$	\\ \hline
$\sphere + 0.7\,h\,\vect s$	&	$8.188\times	10^{-3}$	&	$1.77$	&	$5.138\times	10^{-1}$	&	$1.$	&	$5.843\times	10^{-1}$	&	$1.$	\\ \hline
$\sphere + \phantom{0.1}\,h\,\vect s$	&	$1.93\times	10^{-2}$	&	$2.21$	&	$5.142\times	10^{-1}$	&	$1.$	&	$5.852\times	10^{-1}$	&	$1.$	\\ \hline
		\end{tabular}%
	\end{subtable}
\end{table}

\clearpage

\section{Convergence results}\label{sec:conv}

We solve model problem from~\cite[p.\,20]{surfstokes}. We set
\begin{align}\begin{split}\label{exact_soln}
	\tilde{\vect u}(x, y, z) &\coloneqq (-z^2, y, x)^T, \\
	\tilde p(x, y, z) &\coloneqq x\,y^2 + z, \\
	\phi(\vect x) &\coloneqq \|\vect x\|^2 - 1.
\end{split}\end{align}
The exact solution on the unit sphere is chosen as 
\begin{equation}\label{exact_soln_2}
	\vect u(\vect x) \coloneqq \vect P\,\tilde{\vect u}\big(\frac{\vect x}{\|\vect x\|}\big), \quad
	p(\vect x) \coloneqq \tilde p\big(\frac{\vect x}{\|\vect x\|}\big).
\end{equation}
This way we ``extend''~\eqref{exact_soln} radially. It is useful since we use \textit{the approximation} of~$\sphere$.

\subsection{$\text{P}_1$\,--\,$P_1$ Trace\,FEM}

Here we compare approaches~\eqref{mtx_exact} and~\eqref{mtx_exact_2}. We use one virtual refinement for surface integrals, $m = 2$, so we have $\Gamma_{h/2}^{2 \to 1} = \Gamma_{h/2}^1$ (see~\eqref{gammah} and~\eqref{gammah2}). We use the full stabilization matrix~$\vect C_{\text{full}}$. Note that~$\phi \in P_2$ in~\eqref{exact_soln}, and hence~$\Gamma_h^2 = \Gamma$. Thus~\eqref{mtx_exact} boils down to
\begin{align}\begin{split}\label{mtx_exact_p1}
	\langle \vect A\,\vec{\vect u}, \vec{\vect v} \rangle &= 
		\int^5_{\Gamma_{h/2}^1} \big( E_{s,\,\Gamma}(\vect u) : E_{s,\,\Gamma}(\vect v) + \vect u\cdot\vect v + \tau\,(\vect u\cdot\vect n_{\Gamma})\,(\vect v\cdot\vect n_{\Gamma}) \big) \diff{s} \\
	&
		+ \rho_u \int^5_{\Omega_h^{\Gamma}} \frac{\partial \vect u}{\partial\vect n_{\Gamma}}\cdot\frac{\partial \vect v}{\partial\vect n_{\Gamma}} \diff{\vect x}, \quad \vect A \in \mathbb R^{n_{\vect A} \times n_{\vect A}},\\
	\langle \vect B\,\vec{\vect u}, \vec{\vect q} \rangle &= 
		-\int^5_{\Gamma_{h/2}^1} q\,\Div_{\Gamma} \vect u \diff{s}, \quad \vect B \in \mathbb R^{n_{\vect S} \times n_{\vect A}},\\
	\langle \vect M_0\,\vec{\vect p}, \vec{\vect q} \rangle &=
		\int^5_{\Gamma_{h/2}^1} p\,q \diff{s}, \quad \vect M_0 \in \mathbb R^{n_{\vect S} \times n_{\vect S}},\\
	\langle \vect C_{\text{full}}\,\vec{\vect p}, \vec{\vect q} \rangle &=
		\rho_p \int^5_{\Omega^{\Gamma}_h} \nabla p \cdot \nabla q \diff{\vect x}, \quad \vect C_{\text{full}} \in \mathbb R^{n_{\vect S} \times n_{\vect S}},		 
\end{split}\end{align}
and e.g. the integrand that involves $E_{s,\,\Gamma} \equiv E_s$ is exact. Similarly, approach~\eqref{mtx_exact_2} boils down to
\begin{align}\begin{split}\label{mtx_exact_2_p1}
	\langle \vect A\,\vec{\vect u}, \vec{\vect v} \rangle &= 
		\int^5_{\Gamma_{h/2}^1} \big( E_{s,\,\Gamma_{h/2}^1}(\vect u) : E_{s,\,\Gamma_{h/2}^1}(\vect v) + \vect u\cdot\vect v + \tau\,(\vect u\cdot\vect n_{\Gamma})\,(\vect v\cdot\vect n_{\Gamma}) \big) \diff{s} \\
	&
		+ \rho_u \int^5_{\Omega_h^{\Gamma}} \frac{\partial \vect u}{\partial\vect n_{\Gamma}}\cdot\frac{\partial \vect v}{\partial\vect n_{\Gamma}} \diff{\vect x}, \quad \vect A \in \mathbb R^{n_{\vect A} \times n_{\vect A}},\\
	\langle \vect B\,\vec{\vect u}, \vec{\vect q} \rangle &= 
		-\int^5_{\Gamma_{h/2}^1} q\,\Div_{\Gamma_{h/2}^1} \vect u \diff{s}, \quad \vect B \in \mathbb R^{n_{\vect S} \times n_{\vect A}},\\
	\langle \vect M_0\,\vec{\vect p}, \vec{\vect q} \rangle &=
		\int^5_{\Gamma_{h/2}^1} p\,q \diff{s}, \quad \vect M_0 \in \mathbb R^{n_{\vect S} \times n_{\vect S}},\\
	\langle \vect C_{\text{full}}\,\vec{\vect p}, \vec{\vect q} \rangle &=
		\rho_p \int^5_{\Omega^{\Gamma}_h} \nabla p \cdot \nabla q \diff{\vect x}, \quad \vect C_{\text{full}} \in \mathbb R^{n_{\vect S} \times n_{\vect S}}.
\end{split}\end{align}

For statistics: using 64 CPUs, computation of the meshlevel~6 ($h = 2.6\times10^{-2}$) takes~${\sim}14$ minutes, meshlevel~7 takes~${\sim}75$ minutes, and meshlevel~8 takes~${\sim}7.3$ hours.

\begin{table}[h!]
	\centering\small
	\caption{Convergence results. Matrices are assembled as in~\eqref{mtx_exact_p1} (top table) and~\eqref{mtx_exact_2_p1} (bottom table)}
	\label{tab:p1p1_conv}
	\begin{subtable}{1.\linewidth}\centering
		\begin{tabular}[1.3]{|c||c||c|c||c|c||c|c|}
			\hline
			$m$ & $h$ & $\|\vect u - \vect u_h\|_{\HOneSpace}$ & Order & $\|\vect u - \vect u_h\|_{\LTwoSpace}$ & Order & $\|p - p_h\|_{\LTwoSpace}$ & Order \\
			\hline
			\multirow{8}{*}{2}&$8.33\times	10^{-1}$	&	$3.4$	&	$\text{}$	&	$2.2$	&	$\text{}$	&	$1.1$	&	$\text{}$	\\ \cline{2-8}
&$4.17\times	10^{-1}$	&	$1.8$	&	$9.39\times	10^{-1}$	&	$1.1$	&	$9.57\times	10^{-1}$	&	$9.3\times	10^{-1}$	&	$2.56\times	10^{-1}$	\\ \cline{2-8}
&$2.08\times	10^{-1}$	&	$7.6\times	10^{-1}$	&	$1.23$	&	$3.6\times	10^{-1}$	&	$1.65$	&	$5.1\times	10^{-1}$	&	$8.75\times	10^{-1}$	\\ \cline{2-8}
&$1.04\times	10^{-1}$	&	$3.1\times	10^{-1}$	&	$1.3$	&	$1.\times	10^{-1}$	&	$1.85$	&	$1.8\times	10^{-1}$	&	$1.46$	\\ \cline{2-8}
&$5.21\times	10^{-2}$	&	$1.3\times	10^{-1}$	&	$1.21$	&	$2.6\times	10^{-2}$	&	$1.95$	&	$5.3\times	10^{-2}$	&	$1.79$	\\ \cline{2-8}
&$2.6\times	10^{-2}$	&	$6.4\times	10^{-2}$	&	$1.05$	&	$6.5\times	10^{-3}$	&	$1.98$	&	$1.5\times	10^{-2}$	&	$1.84$	\\ \cline{2-8}
&$1.3\times	10^{-2}$	&	$3.2\times	10^{-2}$	&	$1.01$	&	$1.7\times	10^{-3}$	&	$1.97$	&	$6.6\times	10^{-3}$	&	$1.17$	\\ \cline{2-8}
\rowcolor{LightRed}
&$6.51\times	10^{-3}$	&	$1.6\times	10^{-2}$	&	$9.93\times	10^{-1}$	&	$5.1\times	10^{-4}$	&	$1.72$	&	$5.5\times	10^{-3}$	&	$2.59\times	10^{-1}$	\\ \hline
\rowcolor{LightGreen}
4&$6.51\times	10^{-3}$	&	$1.7\times	10^{-2}$	&	$9.2\times	10^{-1}$	&	$4.1\times	10^{-4}$	&	$2.02$	&	$1.6\times	10^{-3}$	&	$2.06$	\\ \hline
		\end{tabular}
	\end{subtable}
	\vskip 4mm	
	\begin{subtable}{1.\linewidth}\centering
		\begin{tabular}[1.3]{|c||c|c||c|c||c|c|}
			\hline
			$h$ & $\|\vect u - \vect u_h\|_{\HOneSpace}$ & Order & $\|\vect u - \vect u_h\|_{\LTwoSpace}$ & Order & $\|p - p_h\|_{\LTwoSpace}$ & Order \\
			\hline
			$8.33\times	10^{-1}$	&	$3.4$	&	$\text{}$	&	$2.2$	&	$\text{}$	&	$1.1$	&	$\text{}$	\\ \hline
$4.17\times	10^{-1}$	&	$1.8$	&	$9.28\times	10^{-1}$	&	$1.1$	&	$9.3\times	10^{-1}$	&	$9.3\times	10^{-1}$	&	$2.58\times	10^{-1}$	\\ \hline
$2.08\times	10^{-1}$	&	$7.6\times	10^{-1}$	&	$1.23$	&	$3.6\times	10^{-1}$	&	$1.65$	&	$5.1\times	10^{-1}$	&	$8.62\times	10^{-1}$	\\ \hline
$1.04\times	10^{-1}$	&	$3.1\times	10^{-1}$	&	$1.29$	&	$1.\times	10^{-1}$	&	$1.85$	&	$1.9\times	10^{-1}$	&	$1.44$	\\ \hline
$5.21\times	10^{-2}$	&	$1.3\times	10^{-1}$	&	$1.22$	&	$2.6\times	10^{-2}$	&	$1.94$	&	$5.5\times	10^{-2}$	&	$1.77$	\\ \hline
$2.6\times	10^{-2}$	&	$6.4\times	10^{-2}$	&	$1.07$	&	$6.7\times	10^{-3}$	&	$1.97$	&	$1.5\times	10^{-2}$	&	$1.89$	\\ \hline
$1.3\times	10^{-2}$	&	$3.1\times	10^{-2}$	&	$1.02$	&	$1.7\times	10^{-3}$	&	$1.98$	&	$4.\times	10^{-3}$	&	$1.89$	\\ \hline
\rowcolor{LightGreen}
$6.51\times	10^{-3}$	&	$1.5\times	10^{-2}$	&	$1.02$	&	$4.3\times	10^{-4}$	&	$1.99$	&	$1.2\times	10^{-3}$	&	$1.77$	\\ \hline
		\end{tabular}
	\end{subtable}
\end{table}

\begin{table}[h!]
	\centering\small
	\caption{Solver statistics for~\eqref{mtx_exact_p1} (left table) and~\eqref{mtx_exact_2_p1} (right table)}
	\label{tab:p1p1_iters}
	\begin{subtable}{.5\linewidth}\centering
		\begin{tabular}[1.3]{|c|c|c|}
			\hline
			$h$ & Outer iterations & Residual norm \\
			\hline
			$8.33\times	10^{-1}$	&	$14$	&	$1.\times	10^{-8}$	\\ \hline
$4.17\times	10^{-1}$	&	$20$	&	$9.2\times	10^{-9}$	\\ \hline
$2.08\times	10^{-1}$	&	$26$	&	$8.8\times	10^{-9}$	\\ \hline
$1.04\times	10^{-1}$	&	$29$	&	$5.4\times	10^{-9}$	\\ \hline
$5.21\times	10^{-2}$	&	$29$	&	$7.1\times	10^{-9}$	\\ \hline
$2.6\times	10^{-2}$	&	$29$	&	$4.5\times	10^{-9}$	\\ \hline
$1.3\times	10^{-2}$	&	$27$	&	$9.5\times	10^{-9}$	\\ \hline
$6.51\times	10^{-3}$	&	$30$	&	$6.5\times	10^{-9}$	\\ \hline
		\end{tabular}
	\end{subtable}%
	\begin{subtable}{.5\linewidth}\centering
		\begin{tabular}[1.3]{|c|c|c|}
			\hline
			$h$ & Outer iterations & Residual norm \\
			\hline
			$8.33\times	10^{-1}$	&	$15$	&	$1.7\times	10^{-9}$	\\ \hline
$4.17\times	10^{-1}$	&	$20$	&	$7.7\times	10^{-9}$	\\ \hline
$2.08\times	10^{-1}$	&	$26$	&	$7.\times	10^{-9}$	\\ \hline
$1.04\times	10^{-1}$	&	$29$	&	$4.3\times	10^{-9}$	\\ \hline
$5.21\times	10^{-2}$	&	$29$	&	$5.2\times	10^{-9}$	\\ \hline
$2.6\times	10^{-2}$	&	$27$	&	$8.5\times	10^{-9}$	\\ \hline
$1.3\times	10^{-2}$	&	$27$	&	$3.6\times	10^{-9}$	\\ \hline
$6.51\times	10^{-3}$	&	$29$	&	$7.4\times	10^{-9}$	\\ \hline
		\end{tabular}
	\end{subtable}
\end{table}

%\begin{table}[h!]
%	\centering\small
%	\caption{$\vect P_1$\,--\,$P_1$, $\Gamma = \sphere$. Here we use $\Gamma + \alpha_h\,\vect s$ to build (refine) the bulk mesh, and then solve the actual problem using~$\Gamma$. We set~$\alpha_h \coloneqq 0.3\,h$, $\vect s \coloneqq (1, 1, 1)^T/\sqrt{3}$}
%	\label{tab:p1p1_conv_shift}
%	\begin{subtable}{1.\linewidth}\centering
%		\begin{tabular}[1.3]{|c||c|c||c|c||c|c|}
%			\hline
%			$h$ & $\|\vect u - \vect u_h\|_{\HOneSpace}$ & Order & $\|\vect u - \vect u_h\|_{\LTwoSpace}$ & Order & $\|p - p_h\|_{\LTwoSpace}$ & Order \\
%			\hline
%			$8.33\times	10^{-1}$	&	$5.6$	&	$\text{}$	&	$2.4$	&	$\text{}$	&	$1.4$	&	$\text{}$	\\ \hline
$4.17\times	10^{-1}$	&	$3.8$	&	$5.39\times	10^{-1}$	&	$1.2$	&	$9.79\times	10^{-1}$	&	$9.9\times	10^{-1}$	&	$4.86\times	10^{-1}$	\\ \hline
$2.08\times	10^{-1}$	&	$2.$	&	$9.62\times	10^{-1}$	&	$3.9\times	10^{-1}$	&	$1.65$	&	$5.2\times	10^{-1}$	&	$9.24\times	10^{-1}$	\\ \hline
$1.04\times	10^{-1}$	&	$9.8\times	10^{-1}$	&	$1.$	&	$1.1\times	10^{-1}$	&	$1.86$	&	$1.9\times	10^{-1}$	&	$1.47$	\\ \hline
$5.21\times	10^{-2}$	&	$4.9\times	10^{-1}$	&	$9.97\times	10^{-1}$	&	$2.8\times	10^{-2}$	&	$1.95$	&	$5.4\times	10^{-2}$	&	$1.79$	\\ \hline
$2.6\times	10^{-2}$	&	$2.5\times	10^{-1}$	&	$1.$	&	$7.\times	10^{-3}$	&	$1.98$	&	$1.5\times	10^{-2}$	&	$1.82$	\\ \hline
$1.3\times	10^{-2}$	&	$1.2\times	10^{-1}$	&	$9.9\times	10^{-1}$	&	$1.8\times	10^{-3}$	&	$1.97$	&	$6.7\times	10^{-3}$	&	$1.19$	\\ \hline
$6.51\times	10^{-3}$	&	$6.6\times	10^{-2}$	&	$9.12\times	10^{-1}$	&	$5.3\times	10^{-4}$	&	$1.74$	&	$5.5\times	10^{-3}$	&	$2.86\times	10^{-1}$	\\ \hline
%		\end{tabular}
%	\end{subtable}
%	\vskip 4mm
%	\begin{subtable}{1.\linewidth}\centering
%		\begin{tabular}[1.3]{|c|c|c|}
%			\hline
%			$h$ & Outer iterations & Residual norm \\
%			\hline
%			$8.33\times	10^{-1}$	&	$14$	&	$1.\times	10^{-8}$	\\ \hline
$4.17\times	10^{-1}$	&	$20$	&	$9.2\times	10^{-9}$	\\ \hline
$2.08\times	10^{-1}$	&	$26$	&	$8.7\times	10^{-9}$	\\ \hline
$1.04\times	10^{-1}$	&	$29$	&	$5.2\times	10^{-9}$	\\ \hline
$5.21\times	10^{-2}$	&	$29$	&	$7.4\times	10^{-9}$	\\ \hline
$2.6\times	10^{-2}$	&	$29$	&	$5.6\times	10^{-9}$	\\ \hline
$1.3\times	10^{-2}$	&	$29$	&	$3.4\times	10^{-9}$	\\ \hline
$6.51\times	10^{-3}$	&	$30$	&	$6.7\times	10^{-9}$	\\ \hline
%		\end{tabular}
%	\end{subtable}
%\end{table}

The errors in Table~\ref{tab:p1p1_conv} are computed as explained in section~\ref{subsec:err}.

\clearpage

\subsection{$\text{P}_2$\,--\,$P_1$ Trace\,FEM}

We use the normal stabilization matrix~$\vect C_n$. We stick to approach~\eqref{mtx_exact_2}. We choose~$m = O(h^{-1/2})$ so that the geometric error is~$O(h^3)$. Thus $m = 2, 2, 4, 4, 6, 8, 10, 14$ for meshlevel~1, 2 and so forth, respectively.

\begin{table}[h!]
	\centering\small
	\caption{Convergence results. $\tau = h^{-2}$, $\rho_u = \rho_p = h$}
	\label{tab:p2p1_conv_old}
	\begin{subtable}{1.\linewidth}\centering
		\begin{tabular}[1.3]{|c||c|c||c|c||c||c|c|}
			\hline
			$h$ & $\|\vect u - \vect u_h\|_{\HOneSpace}$ & Order & $\|\vect u - \vect u_h\|_{\LTwoSpace}$ & Order & $\|p - p_h\|_{\LTwoSpace}$ & Order \\
			\hline
			$8.33\times	10^{-1}$	&	$2.8$	&	$\text{}$	&	$1.7$	&	$\text{}$	&	$2.1$	&	$\text{}$	\\ \hline
$4.17\times	10^{-1}$	&	$1.9$	&	$5.74\times	10^{-1}$	&	$9.\times	10^{-1}$	&	$9.15\times	10^{-1}$	&	$1.7$	&	$2.57\times	10^{-1}$	\\ \hline
$2.08\times	10^{-1}$	&	$7.2\times	10^{-1}$	&	$1.37$	&	$3.4\times	10^{-1}$	&	$1.39$	&	$7.1\times	10^{-1}$	&	$1.31$	\\ \hline
$1.04\times	10^{-1}$	&	$2.2\times	10^{-1}$	&	$1.7$	&	$1.\times	10^{-1}$	&	$1.78$	&	$2.1\times	10^{-1}$	&	$1.76$	\\ \hline
$5.21\times	10^{-2}$	&	$6.2\times	10^{-2}$	&	$1.85$	&	$2.6\times	10^{-2}$	&	$1.92$	&	$5.1\times	10^{-2}$	&	$2.04$	\\ \hline
$2.6\times	10^{-2}$	&	$1.8\times	10^{-2}$	&	$1.81$	&	$6.5\times	10^{-3}$	&	$2.01$	&	$1.3\times	10^{-2}$	&	$1.91$	\\ \hline
		\end{tabular}
	\end{subtable}
	\vskip 4mm	
	\begin{subtable}{1.\linewidth}\centering
		\begin{tabular}[1.3]{|c||c|c||c||c|}
			\hline
			$h$ & $\| \vect u_h\cdot\vect n \|_{\LTwoSpace}$ & Order & Outer iterations & Residual norm \\
			\hline
			$8.33\times	10^{-1}$	&	$1.7$	&	$\text{}$	&	$24$	&	$9.9\times	10^{-9}$	\\ \hline
$4.17\times	10^{-1}$	&	$8.9\times	10^{-1}$	&	$8.87\times	10^{-1}$	&	$31$	&	$4.6\times	10^{-9}$	\\ \hline
$2.08\times	10^{-1}$	&	$3.4\times	10^{-1}$	&	$1.38$	&	$31$	&	$4.5\times	10^{-9}$	\\ \hline
$1.04\times	10^{-1}$	&	$1.\times	10^{-1}$	&	$1.78$	&	$29$	&	$3.5\times	10^{-9}$	\\ \hline
$5.21\times	10^{-2}$	&	$2.6\times	10^{-2}$	&	$1.95$	&	$27$	&	$6.9\times	10^{-9}$	\\ \hline
$2.6\times	10^{-2}$	&	$6.5\times	10^{-3}$	&	$1.99$	&	$28$	&	$7.3\times	10^{-9}$	\\ \hline
		\end{tabular}
	\end{subtable}
\end{table}

\begin{table}[h!]
	\centering\small
	\caption{Convergence results. $\tau = h^{-3}$, $\rho_u = \rho_p = h$}
	\label{tab:p2p1_conv}
	\begin{subtable}{1.\linewidth}\centering
		\begin{tabular}[1.3]{|c||c|c||c|c||c||c|c|}
			\hline
			$h$ & $\|\vect u - \vect u_h\|_{\HOneSpace}$ & Order & $\|\vect u - \vect u_h\|_{\LTwoSpace}$ & Order & $\|p - p_h\|_{\LTwoSpace}$ & Order \\
			\hline
			$8.33\times	10^{-1}$	&	$3.4$	&	$\text{}$	&	$2.1$	&	$\text{}$	&	$2.6$	&	$\text{}$	\\ \hline
$4.17\times	10^{-1}$	&	$3.$	&	$1.96\times	10^{-1}$	&	$1.5$	&	$5.02\times	10^{-1}$	&	$2.9$	&	$-1.44\times	10^{-1}$	\\ \hline
$2.08\times	10^{-1}$	&	$1.2$	&	$1.29$	&	$6.\times	10^{-1}$	&	$1.32$	&	$1.2$	&	$1.25$	\\ \hline
$1.04\times	10^{-1}$	&	$2.3\times	10^{-1}$	&	$2.42$	&	$1.\times	10^{-1}$	&	$2.53$	&	$2.2\times	10^{-1}$	&	$2.5$	\\ \hline
$5.21\times	10^{-2}$	&	$4.2\times	10^{-2}$	&	$2.46$	&	$1.4\times	10^{-2}$	&	$2.85$	&	$2.6\times	10^{-2}$	&	$3.05$	\\ \hline
$2.6\times	10^{-2}$	&	$1.2\times	10^{-2}$	&	$1.83$	&	$1.7\times	10^{-3}$	&	$3.06$	&	$3.9\times	10^{-3}$	&	$2.73$	\\ \hline
$1.3\times	10^{-2}$	&	$4.7\times	10^{-3}$	&	$1.31$	&	$2.2\times	10^{-4}$	&	$2.96$	&	$7.9\times	10^{-4}$	&	$2.31$	\\ \hline
		\end{tabular}
	\end{subtable}
	\vskip 4mm	
	\begin{subtable}{1.\linewidth}\centering
		\begin{tabular}[1.3]{|c||c|c||c||c|}
			\hline
			$h$ & $\| \vect u_h\cdot\vect n \|_{\LTwoSpace}$ & Order & Outer iterations & Residual norm \\
			\hline
			$8.33\times	10^{-1}$	&	$2.1$	&	$\text{}$	&	$25$	&	$5.8\times	10^{-9}$	\\ \hline
$4.17\times	10^{-1}$	&	$1.5$	&	$4.69\times	10^{-1}$	&	$30$	&	$4.8\times	10^{-9}$	\\ \hline
$2.08\times	10^{-1}$	&	$6.\times	10^{-1}$	&	$1.32$	&	$30$	&	$5.6\times	10^{-9}$	\\ \hline
$1.04\times	10^{-1}$	&	$1.\times	10^{-1}$	&	$2.53$	&	$29$	&	$3.5\times	10^{-9}$	\\ \hline
		\end{tabular}
	\end{subtable}
\end{table}

For statistics: using 64 CPUs, computation of the meshlevel~3 ($h = 2.08\times10^{-1}$) takes~${\sim}1$ minute, meshlevel~4 takes~${\sim}7$ minutes, meshlevel~5 takes~${\sim}50$ minutes, meshlevel~6 takes~$4.8$ hours, and meshlevel~7 takes~${\sim}21.3$ hours.

\section{Notes on DROPS implementation}

\subsection{Notations}\label{subsec:not}

We denote by~$P_h^n \subset \bar P_h^n$ spaces of continuous and discontinuous nodal~$P_n$ interpolants defined on~$\Omega_\Gamma^h$, respectively. For a function~$f$, $I_h^n(f) \in P_h^n$ is the corresponding interpolant; we will use the notation~$f_h^n$ to emphasize that~$f_h^n \in P_h^n$ and~$f_h^n$ approximates~$f$ in some sense, but~$I_h^n(f) \ne f_h^n$.

We set
\begin{align}\label{gammah}
	\Gamma_h^n &\coloneqq \{ \vect x \in \mathbb{R}^3 : \big(I_h^n(\phi)\big)(\vect x) = 0 \}, \\
	\vect n_{\Gamma_h^n} &= \frac{\nabla I_h^n(\phi)}{\|\nabla I_h^n(\phi)\|} \not\in \bar{P}_h^m\text{ for any $m$ if $n > 1$}. \label{gammah:n}
\end{align}  
Note that~$\Gamma_h^n$ is a continuous piecewise $P_n$ surface in~$\Omega_\Gamma^h$, and $\Gamma_h^n \ne I_h^n(\Gamma)$. The unit normal~$\vect n_{\Gamma_h^n}$ is not a rational function; it is continuous in~$T \in \Omega_\Gamma^h$ and discontinuous on faces. We also define
\begin{equation}\label{gammah2}
	\Gamma_{h/m}^{2 \rightarrow 1} \coloneqq \{ \vect x \in \mathbb{R}^3 : \Big(I_{h/m}^1\big(I_h^2(\phi)\big)\Big)(\vect x) = 0 \}.
\end{equation}  
Note that~$I_{h/2}^1\big(I_{h}^2(\phi)\big) = I_{h/2}^1(\phi)$ (since in order to build both~$I_{h/2}^1$ and~$I_{h}^2$ the same values of~$\phi$ are used), and~$I_{h/m}^1\big(I_{h}^2(\phi)\big) \ne I_{h/m}^1(\phi)$ for~$m > 2$. Thus we have~$\Gamma_{h/2}^{2 \rightarrow 1} = \Gamma_{h/2}^1$, and~$\Gamma_{h/m}^{2 \rightarrow 1} \ne \Gamma_{h/m}^1$ for~$m > 2$.
%Note also that~$\Gamma_{h/m}^{2 \rightarrow 1} \ne I_{h/m}^1(\Gamma_h^2)$ since $\Gamma_{h/m}^{2 \rightarrow 1}$ connects roots of piecewise \textbf{linear} functions, and~$I_{h/m}^1(\Gamma_h^2)$ connects roots  

\subsection{Approximation of integrands involving~$\text{n}_\Gamma$}\label{subsec:app}

We start with description of the continuous levelset~$\phi$ of~$\Gamma = \left\{ \vect x \in \mathbb{R}^3 : \phi(\vect x) = 0 \right\}$. It is stored in~\texttt{levelset\_fun} variable. For example, for the unit sphere we have:  
\begin{lstlisting}
// surfnavierstokes_funcs.h
DROPS::Point3DCL sphere_2_shift(0.);
double sphere_2 (const DROPS::Point3DCL& p, double) {
	return pow(p[0] - sphere_2_shift[0], 2.) + 
	       pow(p[1] - sphere_2_shift[1], 2.) + 
	       pow(p[2] - sphere_2_shift[2], 2.) - 1.;
}

// surfnavierstokes.cpp
instat_scalar_fun_ptr levelset_fun;
// ...
levelset_fun = &sphere_2;
\end{lstlisting}
\vskip .2cm
Continuous piecewise $P_2$ interpolant~$I_h^2(\phi)$ of $\phi$ is built on~$\Omega_\Gamma^h$ via iterating over vertices and edges of~$\Omega_\Gamma^h$. It is stored in~\texttt{lset} object:
\begin{lstlisting}
// levelset.cpp
void LevelsetP2ContCL::Init( instat_scalar_fun_ptr phi0, double t) {
	const Uint lvl= Phi.GetLevel(),
	idx= Phi.RowIdx->GetIdx();
	for (auto it = MG_.GetTriangVertexBegin(lvl), end = MG_.GetTriangVertexEnd(lvl); it != end; ++it) {
		if (it->Unknowns.Exist(idx))
			Phi.Data[it->Unknowns(idx)]= phi0( it->GetCoord(), t);
	}
	for (auto it = MG_.GetTriangEdgeBegin(lvl), end = MG_.GetTriangEdgeEnd(lvl); it != end; ++it) {
		if (it->Unknowns.Exist(idx))
			Phi.Data[it->Unknowns(idx)]= phi0( GetBaryCenter( *it), t);
	}
}

// surfnavierstokes.cpp
DROPS::LevelsetP2CL& lset(*DROPS::LevelsetP2CL::Create(mg, lsbnd, sf));
// ...
lset.Init(levelset_fun);
\end{lstlisting}
\vskip .2cm
In order to assemble matrices in~\eqref{mtx} for e.g. $\vect P_1$\,--\,$P_1$ elements, one calls
\begin{lstlisting}
SetupNavierStokesIF_P1P1(mg, &A, /* ... */ lset.Phi, /* ... */);
\end{lstlisting}
(\textbf{Interestingly enough}, this function does not get~\texttt{lset} object that represents the interpolant; it gets only~\texttt{lset.Phi}, which is the object of type~\texttt{VecDescCL}. \texttt{lset.Phi} is essentially just a vector of values of~$\phi$ at interpolation points (i.e. vertices and edges' centroids of~$\Omega_\Gamma^h$). That is, the assembling function above has no idea what~\texttt{lset.Phi} actually represents: one may interpret it as an element of~$P_h^2$ or e.g. $P_{h/2}^1$. Who knows?..)
 
\textbf{No} interpolation is built explicitly for~$\vect n_{\Gamma_h^2}$ in~\eqref{gammah:n}; it is implicitly represented via \texttt{qnormal} data field:
\begin{lstlisting}
// ifacetransp.cpp
class LocalStokesCL {
	// ...
	GridFunctionCL<Point3DCL> qnormal;
	// ...
}
\end{lstlisting}
\texttt{qnormal} object is essentially a set of values of type~\texttt{Point3DCL} which are obtained by mapping a (vector valued) function to suitable quadrature nodes. This is how it is constructed:   
\begin{lstlisting}
// ifacetransp.cpp
void LocalStokesCL::Get_Normals(const LocalP2CL<>& ls, LocalP1CL<Point3DCL>& Normals) {
	for(int i=0; i<10 ; ++i)
		Normals+=ls[i]*P2Grad[i];
}
// ...
void LocalStokesCL::calcIntegrands(const SMatrixCL<3,3>& T, const LocalP2CL<>& ls, const TetraCL& tet) {
	// ...
	LocalP1CL<Point3DCL> Normals;
	Get_Normals(ls, Normals);
	resize_and_evaluate_on_vertexes (Normals, q2Ddomain, qnormal);
	for(Uint i=0; i<qnormal.size(); ++i) 
		qnormal[i]= qnormal[i]/qnormal[i].norm();
	// ...
}
\end{lstlisting}
First $\nabla I_h^2(\phi|_T)$ is built (locally for a tetrahedron~$T \in \Omega_\Gamma^h$ represented by~\texttt{tet}) and saved to \texttt{Normals} object. \texttt{ls[i]} gives the value of~$\phi|_T$ at $i$th node (vertices and edges' centroids\,---\,there are 10 of them for tetrahedra), and \texttt{P2Grad[i]} represents the gradient of quadratic basis function which itself is linear. (Actually, it is sufficient to have $4 < 10$ linear functions to represent $\nabla I_h^2(\phi|_T)$, but this is how it is implemented here.) Finally, \texttt{qnormal} object is built via evaluating \texttt{Normals} at quadrature nodes and normalization.

Objects \texttt{qnormal} for surface integrals and \texttt{q3Dnormal} for volume integrals are used in approximation of~$\vect P = \vect I - \vect n\,\vect n^T$, normal derivatives, and taking-normal-components in~\eqref{mtx}. \texttt{q3Dnormal} is constructed as~\texttt{qnormal} but for quadrature points of tetrahedrons, not triangles. 

For one, $\vect P\,\nabla f_h^2$, $f_h^2 \coloneqq P_2$ basis function on~$\Omega_\Gamma^h$, is approximated via~\texttt{qsurfP2grad} object:
\begin{lstlisting}
// ifacetransp.cpp
void LocalStokesCL::calcIntegrands(/* ... */) {
    // ...
    for(int j=0; j<10 ;++j) {
		resize_and_evaluate_on_vertexes(P2Grad[j], q2Ddomain, qsurfP2grad[j]);
		qsurfP2grad[j]-= dot(qsurfP2grad[j], qnormal)*qnormal;
	}
	// ...
}
\end{lstlisting}  
The term~$\int_{\Omega^{\Gamma}_h} \frac{\partial p}{\partial\vect n} \frac{\partial q}{\partial\vect n} \diff{\vect x}$ in~\eqref{mtx} is computed as
\begin{lstlisting}
// ifacetransp.cpp
void LocalStokesCL::setupA_P1_stab(double A_P1_stab[4][4], double absdet) {
	for (int i=0; i<4; ++i) 
		for (int j=0; j<4; ++j) 
			A_P1_stab[i][j] = quad(dot(q3Dnormal, q3DP1Grad[i])*dot(q3Dnormal, q3DP1Grad[j]), absdet, q3Ddomain, AllTetraC);
}
\end{lstlisting}

\subsection{Quadrature rules for~$\int_{\Gamma}$ and $\int_{\Omega_\Gamma^h}$}\label{subsec:integrals}

All the 3D integrals in~\eqref{mtx} are computed via iteration over~$T \in \Omega_\Gamma^h$ without any virtual refinements. \texttt{q3Ddomain} object represents the set of quadrature nodes and weights:
\begin{lstlisting}
// ifacetransp.cpp
void LocalStokesCL::calc3DIntegrands(/* ... */) {
	make_SimpleQuadDomain<Quad5DataCL> (q3Ddomain, AllTetraC);
	// ...
}
\end{lstlisting} 
It is used e.g. in~\texttt{setupA\_P1\_stab} above. 15 nodes and weights are used, and the quadrature is exact for functions in~$\bar P_h^5$.

All the surface integrals are also computed via iteration over~$T \in \Omega_\Gamma^h$, but using~$\Gamma_{h/2}^1$. One extra ``virtual'' refinement is achieved via setting 
\begin{lstlisting}
// ifacetransp.cpp
LocalStokesCL(bool fullGradient) 
	: lat(PrincipalLatticeCL::instance(2))
	, /* ... */ { /* ... */ }
\end{lstlisting}
\texttt{PrincipalLatticeCL::instance(2)} means that each edge of the tetrahedron~$T \in \Omega_\Gamma^h$ is split into 2 edges, and~$T$ is split into 8 smaller tetrahedrons. Changing 2 to 4 will give us~$\Gamma_{h/4}^{2 \rightarrow 1}$ from~\eqref{gammah2} and so forth. \texttt{q2Ddomain} object represents the set of quadrature nodes and weights:
\begin{lstlisting}
// ifacetransp.cpp
void LocalStokesCL::calcIntegrands(/* ... */) {
	// ...
	evaluate_on_vertexes( ls, lat, Addr( ls_loc));
	spatch.make_patch<MergeCutPolicyCL>( lat, ls_loc);
	make_CompositeQuad5Domain2D ( q2Ddomain, spatch, tet);
	// ...
}
\end{lstlisting} 
Each linear subsurface in~$T \in \Omega_\Gamma^h$ has 7 quadrature nodes and weights, and the quadrature rule is again exact for functions in~$\bar P_h^5$.

\texttt{spatch} represents a set of triangles that form~$\Gamma_{h/2}^1$ inside~$T$. That is, in order to approximate zeros of~$\phi$, $I_{h/2}^1\big(I_{h}^2(\phi)\big) = I_{h/2}^1(\phi)$ is used:
\begin{lstlisting}
// subtriangulation.h
// ...
const double edge_bary1_cut= ls0/(ls0 - ls1); // the root of the level set function on the edge
// ...
\end{lstlisting}
Here~$l(x) := \texttt{ls0}\,(1-x) + \texttt{ls1}\,x$ is a linear function defined on the master edge~$[0, 1]$. Indeed, its root is~$x = \texttt{ls0/(ls0 - ls1)}$.

\subsection{Summary on the matrix assembly}

The matrices in~\eqref{mtx} are assembled as
\begin{align}\begin{split}\label{mtx_exact}
	\langle \vect A\,\vec{\vect u}, \vec{\vect v} \rangle &= 
		\int^5_{\Gamma_{h/m}^{2 \to 1}} \big( E_{s,\,\Gamma_{h}^2}(\vect u) : E_{s,\,\Gamma_{h}^2}(\vect v) + \vect u\cdot\vect v + \tau\,(\vect u\cdot\vect n_{\Gamma_{h}^2})\,(\vect v\cdot\vect n_{\Gamma_{h}^2}) \big) \diff{s} \\
	&
		+ \rho_u \int^5_{\Omega_h^{\Gamma}} \frac{\partial \vect u}{\partial\vect n_{\Gamma_{h}^2}}\cdot\frac{\partial \vect v}{\partial\vect n_{\Gamma_{h}^2}} \diff{\vect x}, \quad \vect A \in \mathbb R^{n_{\vect A} \times n_{\vect A}},\\
	\langle \vect B\,\vec{\vect u}, \vec{\vect q} \rangle &= 
		-\int^5_{\Gamma_{h/m}^{2 \to 1}} q\,\Div_{\Gamma_{h}^2} \vect u \diff{s}, \quad \vect B \in \mathbb R^{n_{\vect S} \times n_{\vect A}},\\
	\langle \vect M_0\,\vec{\vect p}, \vec{\vect q} \rangle &=
		\int^5_{\Gamma_{h/m}^{2 \to 1}} p\,q \diff{s}, \quad \vect M_0 \in \mathbb R^{n_{\vect S} \times n_{\vect S}},\\
	\langle \vect C_n\,\vec{\vect p}, \vec{\vect q} \rangle &=
		\rho_p \int^5_{\Omega^{\Gamma}_h} \frac{\partial p}{\partial\vect n_{\Gamma_{h}^2}} \frac{\partial q}{\partial\vect n_{\Gamma_{h}^2}} \diff{\vect x}, \quad \vect C_n \in \mathbb R^{n_{\vect S} \times n_{\vect S}},\\
	\langle \vect C_{\text{full}}\,\vec{\vect p}, \vec{\vect q} \rangle &=
		\rho_p \int^5_{\Omega^{\Gamma}_h} \nabla p \cdot \nabla q \diff{\vect x}, \quad \vect C_{\text{full}} \in \mathbb R^{n_{\vect S} \times n_{\vect S}},		 
\end{split}\end{align}
Comments:
\begin{itemize}
	\item $\int^5_{\Gamma_{h/m}^{2 \to 1}} \cdot \diff{s}$ denotes a composite quadrature rule that is exact for~$\bar P_h^5(\Gamma_{h/m}^{2 \to 1})$, i.e. this quadrature is exact for piecewise polynomials up to degree~5 on each triangular patch~$\gamma \in \Gamma_{h/m}^{2 \to 1}$, 
	\item $\int^5_{\Omega^{\Gamma}_h} \cdot \diff{\vect x}$ denotes a composite quadrature rule that is exact for~$\bar P_h^5(\Omega^{\Gamma}_h)$, i.e. this quadrature is exact for piecewise polynomials up to degree~5 on each tetrahedron~$T \in \Omega^{\Gamma}_h$, 
	\item $E_{s,\,\Gamma_{h}^2}$ and~$\Div_{\Gamma_{h}^2}$ are defined as their continuous analogues with~$\vect n_{\Gamma}$ in~$\vect P$ replaced with~$\vect n_{\Gamma_{h}^2}$,
	\item It is always the case that integrands use~$\vect n_{\Gamma_{h}^2} \ne \vect n_{\Gamma_{h/m}^{2 \to 1}}$, and the actual domain of integration is~$\Gamma_{h/m}^{2 \to 1} \ne \Gamma_{h}^2$,
	\item $\vect n_{\Gamma_{h}^2}$ is defined in~\eqref{gammah:n} and it is not a polynomial even locally, thus quadrature rules are never exact (although for $\vect P_2$\,--\,$P_1$ shape functions alone these quadratures are exact).
	%\item In our examples we know~$\phi$ and~$\nabla \phi$ exactly, and thus it is super easy to feed the \textbf{exact} normal~$\vect n_{\Gamma} \ne \vect n_{\Gamma_{h}^2}$ to quadratures. Shall we do this?
\end{itemize}

\subsection{Using exact normals in integrands of~$\int_{\Gamma_{h/m}^{2 \to 1}}$ (updated summary)}

It was quite easy to update~\eqref{mtx_exact} such that the exact normals w.r.t. piecewise linear surface domain of integration are used. \texttt{spatch} (section~\ref{subsec:integrals}) has a member function that gives \textbf{physical} normals to its triangles straightaway. Thus implementation of updated quadratures boiled down to constructing~\texttt{GridFunction} object (described in section~\ref{subsec:app}) out of these physical normals. Details of the implementation can be found in commit~\texttt{\href{https://github.com/56th/drops/commit/dacc440587a3f1ea56186fd8c1ff6b6e3ea4b730}{dacc440}}.

Comments:
\begin{itemize}
	\item It is also easy to extend this approach s.t. $\Gamma_{h/m}^1 \ne \Gamma_{h/m}^{2 \rightarrow 1}$  is used (please see section~\ref{subsec:not} and then figures~\ref{fig:phi_exact} and~\ref{fig:phi_inexact}). There is no difference between~$\Gamma_{h/2}^{2 \rightarrow 1}$ and $\Gamma_{h/2}^1$. For $m > 2$, there is no difference between~$\Gamma_{h/m}^{2 \rightarrow 1}$ and $\Gamma_{h/m}^1$ if $\phi \in P^2$. Right now~$\Gamma_{h/m}^{2 \rightarrow 1}$ is implemented.
	\item It is \textbf{not} that easy to use the exact normal w.r.t. piecewise linear surface domain of integration in volume integrals. Our guess is that it shall not be a problem (we can leave~$\vect n_{\Gamma_h^2}$ there),
	\item It is \textbf{not} easy to build~$I_{h}^n(\phi)$ for~$n > 2$. It is \textbf{not} implemented in DROPS as for now.
\end{itemize}
As for now, the matrices in~\eqref{mtx} can also be assembled as
\begin{align}\begin{split}\label{mtx_exact_2}
	\langle \vect A\,\vec{\vect u}, \vec{\vect v} \rangle &= 
		\int^5_{\Gamma_{h/m}^{2 \to 1}} \big( E_{s,\,\textcolor{DarkGreen}{\Gamma_{h/m}^{2 \to 1}}}(\vect u) : E_{s,\,\textcolor{DarkGreen}{\Gamma_{h/m}^{2 \to 1}}}(\vect v) + \vect u\cdot\vect v + \tau\,(\vect u\cdot\vect n_{\Gamma_{h}^2})\,(\vect v\cdot\vect n_{\Gamma_{h}^2}) \big) \diff{s} \\
	&
		+ \rho_u \int^5_{\Omega_h^{\Gamma}} \frac{\partial \vect u}{\partial\vect n_{\Gamma_{h}^2}}\cdot\frac{\partial \vect v}{\partial\vect n_{\Gamma_{h}^2}} \diff{\vect x}, \quad \vect A \in \mathbb R^{n_{\vect A} \times n_{\vect A}},\\
	\langle \vect B\,\vec{\vect u}, \vec{\vect q} \rangle &= 
		-\int^5_{\Gamma_{h/m}^{2 \to 1}} q\,\Div_{\textcolor{DarkGreen}{\Gamma_{h/m}^{2 \to 1}}} \vect u \diff{s}, \quad \vect B \in \mathbb R^{n_{\vect S} \times n_{\vect A}},\\
	\langle \vect M_0\,\vec{\vect p}, \vec{\vect q} \rangle &=
		\int^5_{\Gamma_{h/m}^{2 \to 1}} p\,q \diff{s}, \quad \vect M_0 \in \mathbb R^{n_{\vect S} \times n_{\vect S}},\\
	\langle \vect C_n\,\vec{\vect p}, \vec{\vect q} \rangle &=
		\rho_p \int^5_{\Omega^{\Gamma}_h} \frac{\partial p}{\partial\vect n_{\Gamma_{h}^2}} \frac{\partial q}{\partial\vect n_{\Gamma_{h}^2}} \diff{\vect x}, \quad \vect C_n \in \mathbb R^{n_{\vect S} \times n_{\vect S}},\\
	\langle \vect C_{\text{full}}\,\vec{\vect p}, \vec{\vect q} \rangle &=
		\rho_p \int^5_{\Omega^{\Gamma}_h} \nabla p \cdot \nabla q \diff{\vect x}, \quad \vect C_{\text{full}} \in \mathbb R^{n_{\vect S} \times n_{\vect S}},
\end{split}\end{align}
Notations are as in~\eqref{mtx_exact}. Note that the ``$\tau$-term'' uses~$\vect n_{\Gamma_h^2}$. In order to switch between~\eqref{mtx_exact} and~\eqref{mtx_exact_2}, one modifies JSON input file: 
\begin{lstlisting}
// No_Bnd_Condition.json
"Levelset": {
// ...
"NumbOfVirtualSubEdges" : 2,
"UseExactNormals"       : "yes",
// ...
}
\end{lstlisting}
Here 2 corresponds to $m = 2$, and ``yes'' corresponds to~\eqref{mtx_exact_2}.

\clearpage
\begin{figure}[h!]
	\par\bigskip
	\centering
	\begin{subfigure}{.5\linewidth}
		\centering
		\includegraphicsw[.6]{patches_2.png}
	\end{subfigure}%
	\begin{subfigure}{.5\linewidth}
		\centering
		\includegraphicsw[.6]{normals_2.png}
	\end{subfigure}%
	\par\bigskip
	\begin{subfigure}{.5\linewidth}
		\centering
		\includegraphicsw[.6]{patches_4.png}
	\end{subfigure}%
	\begin{subfigure}{.5\linewidth}
		\centering
		\includegraphicsw[.6]{normals_4.png}
	\end{subfigure}%
	\par\bigskip
	\caption{$\Gamma = \sphere$, $\phi(\vect x) = \|\vect x\|^2 - 1$, $h = 8.33\times10^{-1}$. Top-left: $\Gamma_{h/2}^{2 \rightarrow 1} = \Gamma_{h/2}^1$ (different color corresponds to a different \texttt{spatch} $\gamma \in \Gamma_{h/2}^{2 \rightarrow 1}$ as described in section~\ref{subsec:integrals}). Top-right: a patch~$\gamma \in \Gamma_{h/2}^{2 \rightarrow 1}$ and its normals. Bottom-left and bottom-right: same for~$\Gamma_{h/4}^{2 \rightarrow 1} = \Gamma_{h/4}^1$. \textbf{Note that since~$\phi \in P^2$, we have that~$\Gamma_{h/m}^{2 \rightarrow 1} = \Gamma_{h/m}^1 \rightarrow \Gamma$ as $m \rightarrow \infty$ independent of~$h$}}
	\label{fig:phi_exact}		
\end{figure}
\vfill
\begin{figure}[h!]
	\centering
	\begin{subfigure}{.5\linewidth}
		\centering
		\includegraphicsw[.6]{patches_2_inexact.png}
	\end{subfigure}%
	\begin{subfigure}{.5\linewidth}
		\centering
		\includegraphicsw[.6]{patches_4_inexact.png}
	\end{subfigure}%
	\par\bigskip
	\caption{$\Gamma = \sphere$, $\phi(\vect x) = \|\vect x\|^{1/2} - 1$, $h = 8.33\times10^{-1}$. Left: $\Gamma_{h/2}^{2 \rightarrow 1} = \Gamma_{h/2}^1$ (different color corresponds to a different \texttt{spatch} $\gamma \in \Gamma_{h/2}^{2 \rightarrow 1}$ as described in section~\ref{subsec:integrals}). Right: same for~$\Gamma_{h/4}^{2 \rightarrow 1} \ne \Gamma_{h/4}^1$. \textbf{Note that since~$\phi \not\in \bar{P}^2_h$, we have that~$\Gamma_{h/m}^{2 \rightarrow 1} \ne \Gamma_{h/m}^1$ for~$m > 2$, and $\Gamma_{h/m}^{2 \rightarrow 1} \rightarrow \Gamma_h^2 \ne \Gamma$ as $m \rightarrow \infty$ for fixed~$h$}}
	\label{fig:phi_inexact}		
\end{figure}
\clearpage

\subsection{Quadrature rules for the error computation}\label{subsec:err}

When we first tried to test convergence for~\eqref{mtx_exact_2} in section~\ref{sec:conv}, we noticed that the~$\HOneSpace$-error of the velocity decays much slower than expected, whereas its~$\LTwoSpace$-error behaves as expected. Note that~$\HOneSpace$-error (for e.g. $\vect P_2$\,--\,$P_1$ FE) can be computed as~$\langle \vect w, \vect A_s\,\vect w \rangle^{1/2}$, $\vect w \coloneqq$ vector of d.o.f. corresponding to $\vect P^2_h$ interpolant $I_h^2(\vect u) - \vect u_h$, $\vect A_s \coloneqq$ matrix corresponding to the first term of~$\vect A$ in~\eqref{mtx_exact_2}. Thus the errors are approximated as
\begin{align*}
	\| \vect u - \vect u_h \|_{\HOneSpace} &= \| I^k_h(\vect u) - \vect u_h \|_{\HOneSpace[\Gamma_{h/m}^{2 \to 1}]} + O(h^{k}), \\
	\| \vect u - \vect u_h \|_{\LTwoSpace} &= \| I^k_h(\vect u) - \vect u_h \|_{\LTwoSpace[\Gamma_{h/m}^{2 \to 1}]} + O(h^{k+1}), \\
	\| p - p_h \|_{\LTwoSpace} &= \| I^1_h(p) - p_h \|_{\LTwoSpace[\Gamma_{h/m}^{2 \to 1}]} + O(h^2)
\end{align*}
for large enough~$m$. Here $k = 1$ for~$\vect P_1$\,--\,$P_1$ FEM and $k = 2$ for~$\vect P_2$\,--\,$P_1$.

\textbf{Interestingly enough}, DROPS implementation did not use the assembled matrices to compute errors (and normals that are \textit{different} from the ones in~$\vect A_s$ were used). We corrected it in commit~\texttt{\href{https://github.com/56th/drops/commit/68443b0a678447ba8b3e7e0af1621e6cd402e5d1}{68443b0}}:
\begin{lstlisting}
// surfnavierstokes.cpp
// ...
VectorCL vSolMinusV = vSol.Data - v.Data, pSolMinusP = pSol.Data - p.Data;
auto velL2          = sqrt(dot(v.Data, M.Data * v.Data));
auto velNormalL2    = sqrt(dot(v.Data, S.Data * v.Data));
auto velH1err       = sqrt(dot(vSolMinusV, A.Data * vSolMinusV));
auto velL2err       = sqrt(dot(vSolMinusV, M.Data * vSolMinusV));
auto preL2          = sqrt(dot(p.Data, Schur.Data * p.Data));
auto preL2err       = sqrt(dot(pSolMinusP, Schur.Data * pSolMinusP));
// ...
\end{lstlisting}

\bibliographystyle{plain}
\bibliography{bibl}

\end{document}