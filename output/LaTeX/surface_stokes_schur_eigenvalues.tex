\documentclass[12pt]{article}

\usepackage{mathtools}
\usepackage{amssymb}
\usepackage{amsthm}

\usepackage[dvipsnames, table]{xcolor}
% links
\usepackage{hyperref}
\hypersetup{
	colorlinks,
	linkcolor={Red!90!black},
	citecolor={Red!90!black},
	urlcolor={blue}
}

\usepackage{geometry}
\newgeometry{
	left=1cm, right=1cm, top=.8cm, bottom=.8cm,
	includefoot, heightrounded
}

\usepackage[parfill]{parskip} % https://tex.stackexchange.com/a/16703/135296

% sub figures / grids of pictures
\usepackage{subcaption}
\usepackage{graphicx}
\graphicspath{{img/}} % includegraphics path
\newcommand{\includegraphicsw}[2][1.]{\includegraphics[width=#1\linewidth]{#2}}
\newcommand{\svginput}[1]{\input{img/#1}} % pdf_tex path
\newcommand{\svginputw}[2][\linewidth]{\def\svgwidth{#1}\input{img/#2}} % pdf_tex path

% tables
\usepackage{multirow}
\usepackage{hhline}

% bold for everything
\usepackage{bm}
\newcommand{\vect}[1]{\boldsymbol{\mathbf{#1}}}

% differentials
\newcommand*\diff{\mathop{}\!\mathrm{d}}
\newcommand*\Diff[1]{\mathop{}\!\mathrm{d^#1}}

\DeclareMathOperator{\Div}{div}
\newcommand{\sphere}{{\Gamma_{\text{sph}}}}
\newcommand{\tor}{{\Gamma_{\text{tor}}}}

\usepackage{listings}
\definecolor{mygreen}{rgb}{0,0.6,0}
\lstset{
	language=C++,
	basicstyle=\footnotesize\ttfamily,
	breaklines=true,
	commentstyle=\color{mygreen},
	frame=l,
	xleftmargin=5pt,
	tabsize=2,
	belowskip=-1pt
} 

\title{Some computational results for generalized pressure Schur complement eigenvalues of the surface Stokes problem}
\author{
	Alexander Zhiliakov\thanks{Department of Mathematics, University of Houston, Houston, Texas 77204 (alex@math.uh.edu).}
}

\begin{document}
	
\maketitle
	
\let\oldtabular\tabular
\renewcommand{\tabular}[1][1.5]{\def\arraystretch{#1}\oldtabular}

\section{Bilinear forms and matrices}

We set $n_{\vect A}$ to be the number of velocity d.o.f. and $n_{\vect S}$ to be the number of pressure d.o.f. Vector stiffness, divergence, pressure mass, normal stabilization, and full stabilization matrices resulting from Trace\,FEM discretization of the surface Stokes problem~\cite{surfstokes} are defined via
\begin{align}\begin{split}\label{mtx}
	\langle \vect A\,\bar{\vect u}, \bar{\vect v} \rangle &\approx 
		\int_{\Gamma} \big( E_s(\vect u) : E_s(\vect v) + \vect u\cdot\vect v + \tau\,(\vect u\cdot\vect n)\,(\vect v\cdot\vect n) \big) \diff{s} + 
		\rho_u \int_{\Omega_h^{\Gamma}} \frac{\partial \vect u}{\partial\vect n}\cdot\frac{\partial \vect v}{\partial\vect n} \diff{\vect x}, \quad \vect A \in \mathbb R^{n_{\vect A} \times n_{\vect A}},\\
	\langle \vect B\,\bar{\vect u}, \bar{\vect q} \rangle &\approx 
		-\int_{\Gamma} q\,\Div_{\Gamma} \vect u \diff{s}, \quad \vect B \in \mathbb R^{n_{\vect S} \times n_{\vect A}},\\
	\langle \vect M_0\,\bar{\vect p}, \bar{\vect q} \rangle &\approx
		\int_{\Gamma} p\,q \diff{s}, \quad \vect M_0 \in \mathbb R^{n_{\vect S} \times n_{\vect S}},\\
	\langle \vect C_n\,\bar{\vect p}, \bar{\vect q} \rangle &\approx
		\rho_p \int_{\Omega^{\Gamma}_h} \frac{\partial p}{\partial\vect n} \frac{\partial q}{\partial\vect n} \diff{\vect x}, \quad \vect C_n \in \mathbb R^{n_{\vect S} \times n_{\vect S}},\\
	\langle \vect C_{\text{full}}\,\bar{\vect p}, \bar{\vect q} \rangle &\approx
		\rho_p \int_{\Omega^{\Gamma}_h} \nabla p \cdot \nabla q \diff{\vect x}, \quad \vect C_{\text{full}} \in \mathbb R^{n_{\vect S} \times n_{\vect S}},		 
\end{split}\end{align}
respectively. We use notations as in~\cite{surfstokes}, in particular, $\Omega_\Gamma^h$ is the domain consisting of tetrahedra cut by $\Gamma$. Here~$\bar{\vect u}$ denotes a vector of d.o.f. corresponding to a FE interpolant~$\vect u$ (analogously for $\bar{\vect p}$ and $p$). See~\eqref{mtx_exact} for the computational details. Mesh-dependent parameters are set as
\begin{equation}
	\tau = h^{-2}, \quad \rho_u = \rho_p = h,
\end{equation}
and $h$ is the typical mesh size for tetrahedra from~$\Omega^{\Gamma}_h$. $\Gamma$ is chosen either as the unit sphere or torus, $\Gamma = \sphere$ or $\Gamma = \tor$ (see Figure~\ref{fig:gamma}).

We also define matrices 
\begin{align}\begin{split}
	\vect C_0 &\coloneqq \vect 0,\\
	\vect M_n &\coloneqq \vect M_0 + \vect C_n,\\
	\vect M_{\text{full}} &\coloneqq \vect M_0 + \vect C_{\text{full}}.
\end{split}\end{align}
We are interested in (generalized) extreme eigenvalues of the pressure Schur complement matrices
\begin{align}\label{schur}\begin{split}
	\vect S_0 &\coloneqq \vect B\,\vect A^{-1}\,\vect B^{T},\\
	\vect S_n &\coloneqq \vect S_0 + \vect C_n,\\
	\vect S_{\text{full}} &\coloneqq \vect S_0 + \vect C_{\text{full}},
\end{split}\end{align}
i.e. in solving
\begin{equation}\label{problem}
	\vect S_\star\,\vect x = \lambda\,\vect M_\star\,\vect x,
\end{equation}
where ``$\star$'' stands for ``$0$,'' ``$n$,'' or ``full.'' We denote by~$0 = \lambda_1 < \lambda_2 \le \dots \le \lambda_{n_{\vect S}} = O(1)$ the spectrum of~\eqref{problem}.

\section{Solution description}

Computing $\vect A^{-1}$ in~\eqref{schur} becomes troublesome already for $h = 5.21\times10^{-2}$ ($n_{\vect A} = 32736$ for $\vect u \in \vect P_1$ FE space): although $\vect A$ is sparse, $\vect A^{-1}$ is dense and consumes 8.5+ GB in double-precision arithmetic. A quick research \href{https://mathematica.stackexchange.com/questions/189620/matrix-free-arnoldi-method-for-eigensystems}{showed} that \texttt{Mathematica} has no built-in matrix-free eigenvalue routines. \texttt{Intel MKL}'s FEAST algorithm for computing (generalized) eigenvalues in an interval \href{https://software.intel.com/sites/default/files/mkl-2019-developer-reference-c.pdf#_OPENTOPIC_TOC_PROCESSING_d62e853651}{is suitable for matrix-free implementations}; however, it requires some expensive operations to be implemented (e.g. matrix-matrix multiplications $\vect Y \leftarrow \vect S_\star\,\vect X$, $\vect Y \leftarrow \vect M_\star\,\vect X$ and approximating the action of inverses in the form $\vect y \leftarrow (\sigma\,\vect M_\star - \vect S_\star)^{-1}\,\vect x$).

Taking this into account, instead of~\eqref{problem} we consider a perturbed\footnotemark{} problem
\begin{equation}\label{problem_pert}
	\underbrace{\begin{bmatrix}
		\vect A & \phantom{-}\vect B^T \\
		\vect B & -\vect C_\star \\
	\end{bmatrix}}_{\mathcal A_\star \coloneqq}
	\begin{bmatrix}
		\vect x \\
		\vect y
	\end{bmatrix}
	=
	\mu
	\underbrace{\begin{bmatrix}
		\epsilon\,\vect A & \\
		& \vect M_\star
	\end{bmatrix}}_{\mathcal M^\epsilon_\star \coloneqq}
	\begin{bmatrix}
		\vect x \\
		\vect y
	\end{bmatrix}
\end{equation}
with $0 < \epsilon \ll 1$. For $\mathcal A_0$ and $\mathcal M^\epsilon_0$ we have
\begin{equation}
	\mu = -\lambda + o(1)\quad\text{or}\quad\epsilon^{-1} + \lambda + o(1),\qquad\epsilon \rightarrow 0.
\end{equation}
This makes it easy to pick only ``correct'' eigenvalues. To ease the computation further we replace the $(1, 1)$-block of~$\mathcal M^\epsilon_\star$ with $\epsilon\,\vect I$. 

To make sure that results are consistent we solve~\eqref{problem_pert} for~$\epsilon = 10^{-5}$ and~$\epsilon = 10^{-6}$; for the coarse mesh levels we also check that the dense solver for~\eqref{problem} and the iterative one for~\eqref{problem_pert} give solutions that coincide.  

\footnotetext{The majority of generalized eigenvalue solvers require left-hand-side matrix to be Hermitian and right-hand-side matrix to be Hermitian \textbf{positive definite}; that's why we need to introduce $\epsilon > 0$.}     

\section{Numerical results: dependency of the spectrum on the mesh size}

\begin{figure}[h]
	\centering
	\begin{subfigure}{.33\linewidth}
		\centering
		\includegraphicsw{lvl1.png}
		\caption{$h = 8.33\times10^{-1}$}
	\end{subfigure}%
	\begin{subfigure}{.33\linewidth}
		\centering
		\includegraphicsw{lvl2.png}
		\caption{$h = 4.17\times10^{-1}$}
	\end{subfigure}%
	\begin{subfigure}{.33\linewidth}
		\centering
		\includegraphicsw{lvl3.png}
		\caption{$h = 2.08\times10^{-1}$}
	\end{subfigure}
	\par
	\begin{subfigure}{.33\linewidth}
		\centering
		\includegraphicsw{tor_lvl3.png}
		\caption{$h = 2.08\times10^{-1}$}
	\end{subfigure}%
	\begin{subfigure}{.33\linewidth}
		\centering
		\includegraphicsw{tor_lvl4.png}
		\caption{$h = 1.04\times10^{-1}$}
	\end{subfigure}%
	\begin{subfigure}{.33\linewidth}
		\centering
		\includegraphicsw{tor_lvl5.png}
		\caption{$h = 5.21\times10^{-2}$}
	\end{subfigure}
	\caption{First three mesh levels for~$\sphere$ (top) and $\tor$ (bottom)}
	\label{fig:gamma}		
\end{figure}

\clearpage

\begin{table}[h]
	\centering\small
	\caption{Spectrum of~\eqref{problem} for $\vect P_1$\,--\,$P_1$} 
	\label{tab:p1p1}
	\begin{subtable}{1.\linewidth}
		\centering
		\caption{$\Gamma = \sphere$}
		\label{tab:p1p1:sph}
		\begin{tabular}[1.3]{|c|c|c|c|c|c|c|c|c|}
			\hline
			\multirow{2}{*}{$h$} & \multirow{2}{*}{$n_{\vect A}$} & \multirow{2}{*}{$n_{\vect S}$} & \multicolumn{2}{c|}{$\vect S_0$} & \multicolumn{2}{c|}{$\vect S_n$} & \multicolumn{2}{c|}{$\vect S_{\text{full}}$} \\ 
			\cline{4-9}
			& & & $\lambda_2$ & $\lambda_{n_{\vect S}}$ & $\lambda_2$ & $\lambda_{n_{\vect S}}$ & $\lambda_2$ & $\lambda_{n_{\vect S}}$ \\ 
			\hline
			\input{sphere_2_P1P1_table.tex}
			%		\multirow{2}{*}{$h$} & \multirow{2}{*}{$n_{\vect A}$} & \multirow{2}{*}{$n_{\vect S}$} & \multicolumn{2}{c|}{$\vect S_0$} & \multicolumn{2}{c|}{$\vect S_n$} & \multicolumn{2}{c|}{$\vect S_{\text{full}}$} \\ 
			%		\cline{4-9}
			%		& & & $r_2$ & $r_{n_{\vect S}}$ & $r_2$ & $r_{n_{\vect S}}$ & $r_2$ & $r_{n_{\vect S}}$ \\ 
			%		\hline
			%		$2.08\times	10^{-1}$	&	$972$	&	$324$	&	$3.\times	10^{-10}$	&	$3.\times	10^{-17}$	&	$9.\times	10^{-13}$	&	$5.\times	10^{-8}$	&	$1.\times	10^{-13}$	&	$3.\times	10^{-7}$	\\ \hline
$1.04\times	10^{-1}$	&	$4740$	&	$1580$	&	$1.\times	10^{-15}$	&	$2.\times	10^{-18}$	&	$5.\times	10^{-11}$	&	$5.\times	10^{-8}$	&	$4.\times	10^{-10}$	&	$8.\times	10^{-8}$	\\ \hline
		\end{tabular}
	\end{subtable}%
	\vskip 3mm
	\begin{subtable}{1.\linewidth}
		\centering
		\caption{$\Gamma = \tor$}
		\label{tab:p1p1:tor}
		\begin{tabular}[1.3]{|c|c|c|c|c|c|c|c|c|}
			\hline
			\multirow{2}{*}{$h$} & \multirow{2}{*}{$n_{\vect A}$} & \multirow{2}{*}{$n_{\vect S}$} & \multicolumn{2}{c|}{$\vect S_0$} & \multicolumn{2}{c|}{$\vect S_n$} & \multicolumn{2}{c|}{$\vect S_{\text{full}}$} \\ 
			\cline{4-9}
			& & & $\lambda_2$ & $\lambda_{n_{\vect S}}$ & $\lambda_2$ & $\lambda_{n_{\vect S}}$ & $\lambda_2$ & $\lambda_{n_{\vect S}}$ \\ 
			\hline
			$2.08\times	10^{-1}$	&	$972$	&	$324$	&	$5.04\times	10^{-2}$	&	$4.93$	&	$2.84\times	10^{-1}$	&	$1.35$	&	$3.64\times	10^{-1}$	&	$1.19$	\\ \hline
$1.04\times	10^{-1}$	&	$4740$	&	$1580$	&	$2.99\times	10^{-3}$	&	$3.83$	&	$1.58\times	10^{-1}$	&	$1.02$	&	$3.35\times	10^{-1}$	&	$1.01$	\\ \hline
			%		\multirow{2}{*}{$h$} & \multirow{2}{*}{$n_{\vect A}$} & \multirow{2}{*}{$n_{\vect S}$} & \multicolumn{2}{c|}{$\vect S_0$} & \multicolumn{2}{c|}{$\vect S_n$} & \multicolumn{2}{c|}{$\vect S_{\text{full}}$} \\ 
			%		\cline{4-9}
			%		& & & $r_2$ & $r_{n_{\vect S}}$ & $r_2$ & $r_{n_{\vect S}}$ & $r_2$ & $r_{n_{\vect S}}$ \\ 
			%		\hline
			%		$2.08\times	10^{-1}$	&	$972$	&	$324$	&	$3.\times	10^{-10}$	&	$3.\times	10^{-17}$	&	$9.\times	10^{-13}$	&	$5.\times	10^{-8}$	&	$1.\times	10^{-13}$	&	$3.\times	10^{-7}$	\\ \hline
$1.04\times	10^{-1}$	&	$4740$	&	$1580$	&	$1.\times	10^{-15}$	&	$2.\times	10^{-18}$	&	$5.\times	10^{-11}$	&	$5.\times	10^{-8}$	&	$4.\times	10^{-10}$	&	$8.\times	10^{-8}$	\\ \hline
		\end{tabular}
	\end{subtable}
\end{table}
\vfill
\begin{figure}[h]
	\centering\small
	\begin{subfigure}{.49\linewidth}
		\centering
		\includegraphicsw{sphere_2_P1P1.png}
		%\caption{$\vect P_1$\,--\,$P_1$ for $\sphere$}
	\end{subfigure}%
	\hfill
	\begin{subfigure}{.49\linewidth}
		\centering
		\includegraphicsw{torus_P1P1.png}
		%\caption{$\vect P_1$\,--\,$P_1$ for $\tor$}
	\end{subfigure}
	\caption{Log-log plot of~$\lambda_2$ for Tables~\ref{tab:p1p1}\subref{tab:p1p1:sph} (left) and~\ref{tab:p1p1}\subref{tab:p1p1:tor} (right)}
\end{figure}
\vfill

\clearpage

\clearpage

\begin{table}[h]
	\centering\small
	\caption{Spectrum of~\eqref{problem} for $\vect P_2$\,--\,$P_1$} 
	\label{tab:p2p1}
	\begin{subtable}{1.\linewidth}
		\centering
		\caption{$\Gamma = \sphere$}
		\label{tab:p2p1:sph}
		\begin{tabular}[1.3]{|c|c|c|c|c|c|c|c|c|}
			\hline
			\multirow{2}{*}{$h$} & \multirow{2}{*}{$n_{\vect A}$} & \multirow{2}{*}{$n_{\vect S}$} & \multicolumn{2}{c|}{$\vect S_0$} & \multicolumn{2}{c|}{$\vect S_n$} & \multicolumn{2}{c|}{$\vect S_{\text{full}}$} \\ 
			\cline{4-9}
			& & & $\lambda_2$ & $\lambda_{n_{\vect S}}$ & $\lambda_2$ & $\lambda_{n_{\vect S}}$ & $\lambda_2$ & $\lambda_{n_{\vect S}}$ \\ 
			\hline
			\input{sphere_2_P2P1_table.tex}
		\end{tabular}
	\end{subtable}%
	\vskip 3mm
	\begin{subtable}{1.\linewidth}
		\centering
		\caption{$\Gamma = \tor$}
		\label{tab:p2p1:tor}
		\begin{tabular}[1.3]{|c|c|c|c|c|c|c|c|c|}
			\hline
			\multirow{2}{*}{$h$} & \multirow{2}{*}{$n_{\vect A}$} & \multirow{2}{*}{$n_{\vect S}$} & \multicolumn{2}{c|}{$\vect S_0$} & \multicolumn{2}{c|}{$\vect S_n$} & \multicolumn{2}{c|}{$\vect S_{\text{full}}$} \\ 
			\cline{4-9}
			& & & $\lambda_2$ & $\lambda_{n_{\vect S}}$ & $\lambda_2$ & $\lambda_{n_{\vect S}}$ & $\lambda_2$ & $\lambda_{n_{\vect S}}$ \\ 
			\hline
			$2.08\times	10^{-1}$	&	$5184$	&	$324$	&	$9.92\times	10^{-2}$	&	$3.89$	&	$1.33\times	10^{-1}$	&	$1.37$	&	$1.75\times	10^{-1}$	&	$1.19$	\\ \hline
$1.04\times	10^{-1}$	&	$27906$	&	$1580$	&	$1.46\times	10^{-2}$	&	$4.35$	&	$2.84\times	10^{-1}$	&	$1.04$	&	$2.99\times	10^{-1}$	&	$1.02$	\\ \hline
		\end{tabular}
	\end{subtable}
\end{table}
\vfill
\begin{figure}[h]
	\centering\small
	\begin{subfigure}{.49\linewidth}
		\centering
		\includegraphicsw{sphere_2_P2P1.png}
	\end{subfigure}%
	\hfill
	\begin{subfigure}{.49\linewidth}
		\centering
		\includegraphicsw{torus_P2P1.png}
	\end{subfigure}
	\caption{Log-log plot of~$\lambda_2$ for Tables~\ref{tab:p2p1}\subref{tab:p2p1:sph} (left) and~\ref{tab:p2p1}\subref{tab:p2p1:tor} (right)}
\end{figure}
\vfill

\clearpage

\section{Numerical results: sensitivity of the spectrum to levelset shifts}

In this section we investigate the sensitivity of the spectrum to levelset shifts
\begin{equation}\label{shift}
	\Gamma \mapsto \Gamma + \alpha\,\vect s,
\end{equation}
for some $\alpha \in \mathbb R$ and $\vect s \in \mathbb R^3$, $\|\vect s\| = 1$. 

We construct the bulk mesh~$\Omega_h^\Gamma$ and then perform the assembly of matrices~\eqref{mtx} using the shifted levelset~\eqref{shift}. That is, the refinement of~$\Omega_h^\Gamma$ is performed using~$\Gamma$, not~$\Gamma + \alpha\,\vect s$, and~$\Omega_h^{\Gamma + \alpha\,\vect s}$ is never constructed. We choose $\alpha \in [0, h]$ to guarantee the appearance of ``small cuts'' in~$\Omega_h^\Gamma$.

\begin{figure}[h]
	\centering
	\begin{subfigure}{.5\linewidth}
		\centering
		\includegraphicsw{{shift_0.0}.png}
		\caption{$\sphere$}
	\end{subfigure}%
	\begin{subfigure}{.5\linewidth}
		\centering
		\includegraphicsw{{shift_0.2}.png}
		\caption{$\sphere + \alpha\,\vect s$}
	\end{subfigure}%
	\caption{The unit sphere (left) and the shifted unit sphere (right). Here $\vect s = (0, 1, 1)^T/\sqrt{2}$, $\alpha = 0.2$, and $h = 2.08\times10^{-1}$. The bulk mesh~$\Omega_\Gamma^h$ is computed for~$\sphere$ and then used for~$\sphere + \alpha\,\vect s$}
	\label{fig:shift}		
\end{figure}

\begin{table}[h]
	\centering
	\caption{Spectrum of~\eqref{problem} for perturbed levelset $\sphere + \alpha\,\vect s$. Here $\vect s = (1, 1, 1)^T/\sqrt{3}$, $h = 1.04\times10^{-1}$} 
	\label{tab:p1p1_shift_h=0.104167}
	\small
	\begin{subtable}{1.\linewidth}
		\centering
		\caption{$\vect P_1$\,--\,$P_1$}
		\begin{tabular}[1.3]{|c|c|c|c|c|c|c|}
			\hline
			\multirow{2}{*}{Surface} & \multicolumn{2}{c|}{$\vect S_0$} & \multicolumn{2}{c|}{$\vect S_n$} & \multicolumn{2}{c|}{$\vect S_{\text{full}}$} \\ 
			\cline{2-7}
			& $\lambda_2$ & $\lambda_{n_{\vect S}}$ & $\lambda_2$ & $\lambda_{n_{\vect S}}$ & $\lambda_2$ & $\lambda_{n_{\vect S}}$ \\ 
			\hline
			$\sphere$\phantom{ + 0.1\,h$\,\vect s$}	&	$2.01\times	10^{-3}$	&	$7.79\times	10^{-1}$	&	$2.19\times	10^{-1}$	&	$1.$	&	$5.82\times	10^{-1}$	&	$1.$	\\ \hline
$\sphere + 0.1\,h\,\vect s$	&	$4.83\times	10^{-4}$	&	$8.01\times	10^{-1}$	&	$2.2\times	10^{-1}$	&	$1.$	&	$5.82\times	10^{-1}$	&	$1.$	\\ \hline
$\sphere + 0.3\,h\,\vect s$	&	$7.28\times	10^{-4}$	&	$8.17\times	10^{-1}$	&	$2.2\times	10^{-1}$	&	$1.$	&	$5.82\times	10^{-1}$	&	$1.$	\\ \hline
$\sphere + 0.5\,h\,\vect s$	&	$3.12\times	10^{-4}$	&	$8.67\times	10^{-1}$	&	$2.22\times	10^{-1}$	&	$1.$	&	$5.82\times	10^{-1}$	&	$1.$	\\ \hline
$\sphere + 0.7\,h\,\vect s$	&	$1.44\times	10^{-3}$	&	$1.51$	&	$2.25\times	10^{-1}$	&	$1.$	&	$5.82\times	10^{-1}$	&	$1.$	\\ \hline
$\sphere + \phantom{0.1}\,h\,\vect s$	&	$1.79\times	10^{-3}$	&	$2.07$	&	$2.33\times	10^{-1}$	&	$1.$	&	$5.83\times	10^{-1}$	&	$1.$	\\ \hline
		\end{tabular}
	\end{subtable}
	\vskip 4mm
	\begin{subtable}{1.\linewidth}
		\centering
		\caption{$\vect P_2$\,--\,$P_1$}
		\begin{tabular}[1.3]{|c|c|c|c|c|c|c|}
			\hline
			\multirow{2}{*}{Surface} & \multicolumn{2}{c|}{$\vect S_0$} & \multicolumn{2}{c|}{$\vect S_n$} & \multicolumn{2}{c|}{$\vect S_{\text{full}}$} \\ 
			\cline{2-7}
			& $\lambda_2$ & $\lambda_{n_{\vect S}}$ & $\lambda_2$ & $\lambda_{n_{\vect S}}$ & $\lambda_2$ & $\lambda_{n_{\vect S}}$ \\ 
			\hline
			\input{sphere_2_P2P1_h=0.104167_table.tex}
		\end{tabular}%
	\end{subtable}
\end{table}

\clearpage

\section{Notes on DROPS implementation}

\subsection{Notations}

We denote by~$P_h^n \subset \bar P_h^n$ spaces of continuous and discontinuous nodal~$P_n$ interpolants defined on~$\Omega_\Gamma^h$, respectively. For a function~$f$, $I_h^n(f) \in P_h^n$ is the corresponding interpolant; we will use the notation~$f_h^n$ to emphasize that~$f_h^n \in P_h^n$ and~$f_h^n$ approximates~$f$ in some sense, but~$I_h^n(f) \ne f_h^n$.

We set
\begin{align}\label{gammah}
	\Gamma_h^n &\coloneqq \{ \vect x \in \mathbb{R}^3 : \big(I_h^n(\phi)\big)(\vect x) = 0 \}, \\
	\vect n_{\Gamma_h^n} &= \frac{\nabla I_h^n(\phi)}{\|\nabla I_h^n(\phi)\|} \not\in \bar{P}_h^m\text{ for any }m. \label{gammah:n}
\end{align}  
Note that~$\Gamma_h^n$ is a continuous piecewise $P_n$ surface in~$\Omega_\Gamma^h$, and $\Gamma_h^n \ne I_h^n(\Gamma)$. The unit normal~$\vect n_{\Gamma_h^n}$ is not a rational function; it is continuous in~$T \in \Omega_\Gamma^h$ and discontinuous on faces.   

\subsection{Approximation of integrands involving~$\text{n}_\Gamma$}

We start with description of the continuous levelset~$\phi$ of~$\Gamma = \left\{ \vect x \in \mathbb{R}^3 : \phi(\vect x) = 0 \right\}$. It is stored in~\texttt{levelset\_fun} variable. For example, for the unit sphere we have:  
\begin{lstlisting}
// surfnavierstokes_funcs.h
DROPS::Point3DCL sphere_2_shift(0.);
double sphere_2 (const DROPS::Point3DCL& p, double) {
	return pow(p[0] - sphere_2_shift[0], 2.) + 
	       pow(p[1] - sphere_2_shift[1], 2.) + 
	       pow(p[2] - sphere_2_shift[2], 2.) - 1.;
}

// surfnavierstokes.cpp
instat_scalar_fun_ptr levelset_fun;
// ...
levelset_fun = &sphere_2;
\end{lstlisting}
\vskip .2cm
Continuous piecewise $P_2$ interpolant~$I_h^2(\phi)$ of $\phi$ is built on~$\Omega_\Gamma^h$ via iterating over vertices and edges of~$\Omega_\Gamma^h$. It is stored in~\texttt{lset} object:
\begin{lstlisting}
// levelset.cpp
void LevelsetP2ContCL::Init( instat_scalar_fun_ptr phi0, double t) {
	const Uint lvl= Phi.GetLevel(),
	idx= Phi.RowIdx->GetIdx();
	for (auto it = MG_.GetTriangVertexBegin(lvl), end = MG_.GetTriangVertexEnd(lvl); it != end; ++it) {
		if (it->Unknowns.Exist(idx))
			Phi.Data[it->Unknowns(idx)]= phi0( it->GetCoord(), t);
	}
	for (auto it = MG_.GetTriangEdgeBegin(lvl), end = MG_.GetTriangEdgeEnd(lvl); it != end; ++it) {
		if (it->Unknowns.Exist(idx))
			Phi.Data[it->Unknowns(idx)]= phi0( GetBaryCenter( *it), t);
	}
}

// surfnavierstokes.cpp
DROPS::LevelsetP2CL& lset(*DROPS::LevelsetP2CL::Create(mg, lsbnd, sf));
// ...
lset.Init(levelset_fun);
\end{lstlisting}
\vskip .2cm
In order to assemble matrices in~\eqref{mtx} for e.g. $\vect P_1$\,--\,$P_1$ elements, one calls
\begin{lstlisting}
SetupNavierStokesIF_P1P1(mg, &A, /* ... */ lset.Phi, /* ... */);
\end{lstlisting}
(\textbf{Interestingly enough}, this function does not get~\texttt{lset} object that represents the interpolant; it gets only~\texttt{lset.Phi}, which is the object of type~\texttt{VecDescCL}. \texttt{lset.Phi} is essentially just a vector of values of~$\phi$ at interpolation points (i.e. vertices and edges' centroids of~$\Omega_\Gamma^h$). That is, the assembling function above has no idea what~\texttt{lset.Phi} actually represents: one may interpret it as an element of~$P_h^2$ or e.g. $P_{h/2}^1$. Who knows?..)
 
\textbf{No} interpolation is built explicitly for~$\vect n_{\Gamma_h^2}$ in~\eqref{gammah:n}; it is implicitly represented via \texttt{qnormal} data field:
\begin{lstlisting}
// ifacetransp.cpp
class LocalStokesCL {
	// ...
	GridFunctionCL<Point3DCL> qnormal;
	// ...
}
\end{lstlisting}
\texttt{qnormal} object is essentially a set of values of type~\texttt{Point3DCL} which are obtained by mapping a (vector valued) function to suitable quadrature nodes. This is how it is constructed:   
\begin{lstlisting}
// ifacetransp.cpp
void LocalStokesCL::Get_Normals(const LocalP2CL<>& ls, LocalP1CL<Point3DCL>& Normals) {
	for(int i=0; i<10 ; ++i)
		Normals+=ls[i]*P2Grad[i];
}
// ...
void LocalStokesCL::calcIntegrands(const SMatrixCL<3,3>& T, const LocalP2CL<>& ls, const TetraCL& tet) {
	// ...
	LocalP1CL<Point3DCL> Normals;
	Get_Normals(ls, Normals);
	resize_and_evaluate_on_vertexes (Normals, q2Ddomain, qnormal);
	for(Uint i=0; i<qnormal.size(); ++i) 
		qnormal[i]= qnormal[i]/qnormal[i].norm();
	// ...
}
\end{lstlisting}
First $\nabla I_h^2(\phi|_T)$ is built (locally for a tetrahedron~$T \in \Omega_\Gamma^h$ represented by~\texttt{tet}) and saved to \texttt{Normals} object. \texttt{ls[i]} gives the value of~$\phi|_T$ at $i$th node (vertices and edges' centroids\,---\,there are 10 of them for tetrahedra), and \texttt{P2Grad[i]} represents the gradient of quadratic basis function which itself is linear. (Actually, it is sufficient to have $4 < 10$ linear functions to represent $\nabla I_h^2(\phi|_T)$, but this is how it is implemented here.) Finally, \texttt{qnormal} object is built via evaluating \texttt{Normals} at quadrature nodes and normalization.

Objects \texttt{qnormal} for surface integrals and \texttt{q3Dnormal} for volume integrals are used in approximation of~$\vect P = \vect I - \vect n\,\vect n^T$, normal derivatives, and taking-normal-components in~\eqref{mtx}. \texttt{q3Dnormal} is constructed as~\texttt{qnormal} but for quadrature points of tetrahedrons, not triangles. 

For one, $\vect P\,\nabla f_h^2$, $f_h^2 \coloneqq P_2$ basis function on~$\Omega_\Gamma^h$, is approximated via~\texttt{qsurfP2grad} object:
\begin{lstlisting}
// ifacetransp.cpp
void LocalStokesCL::calcIntegrands(/* ... */) {
    // ...
    for(int j=0; j<10 ;++j) {
		resize_and_evaluate_on_vertexes(P2Grad[j], q2Ddomain, qsurfP2grad[j]);
		qsurfP2grad[j]-= dot(qsurfP2grad[j], qnormal)*qnormal;
	}
	// ...
}
\end{lstlisting}  
The term~$\int_{\Omega^{\Gamma}_h} \frac{\partial p}{\partial\vect n} \frac{\partial q}{\partial\vect n} \diff{\vect x}$ in~\eqref{mtx} is computed as
\begin{lstlisting}
// ifacetransp.cpp
void LocalStokesCL::setupA_P1_stab(double A_P1_stab[4][4], double absdet) {
	for (int i=0; i<4; ++i) 
		for (int j=0; j<4; ++j) 
			A_P1_stab[i][j] = quad(dot(q3Dnormal, q3DP1Grad[i])*dot(q3Dnormal, q3DP1Grad[j]), absdet, q3Ddomain, AllTetraC);
}
\end{lstlisting}

\subsection{Quadrature rules for~$\int_{\Gamma}$ and $\int_{\Omega_\Gamma^h}$}

All the 3D integrals in~\eqref{mtx} are computed via iteration over~$T \in \Omega_\Gamma^h$ without any virtual refinements. \texttt{q3Ddomain} object represents the set of quadrature nodes and weights:
\begin{lstlisting}
// ifacetransp.cpp
void LocalStokesCL::calc3DIntegrands(/* ... */) {
	make_SimpleQuadDomain<Quad5DataCL> (q3Ddomain, AllTetraC);
	// ...
}
\end{lstlisting} 
It is used e.g. in~\texttt{setupA\_P1\_stab} above. 15 nodes and weights are used, and the quadrature is exact for functions in~$\bar P_h^5$.

All the surface integrals are computed using~$\Gamma_{h/2}^1$. One extra ``virtual'' refinement is achieved via setting 
\begin{lstlisting}
// ifacetransp.cpp
LocalStokesCL(bool fullGradient) 
	: lat(PrincipalLatticeCL::instance(2))
	, /* ... */ { /* ... */ }
\end{lstlisting}
(Changing 2 to 4 will give us~$\Gamma_{h/4}^1$ and so forth.) \texttt{q2Ddomain} object represents the set of quadrature nodes and weights:
\begin{lstlisting}
// ifacetransp.cpp
void LocalStokesCL::calcIntegrands(/* ... */) {
	// ...
	evaluate_on_vertexes( ls, lat, Addr( ls_loc));
	spatch.make_patch<MergeCutPolicyCL>( lat, ls_loc);
	make_CompositeQuad5Domain2D ( q2Ddomain, spatch, tet);
	// ...
}
\end{lstlisting} 
Each linear subsurface in~$T \in \Omega_\Gamma^h$ has 7 quadrature nodes and weights, and the quadrature rule is again exact for functions in~$\bar P_h^5$.

\subsection{Summary}

The matrices in~\eqref{mtx} are assembled as
\begin{align}\begin{split}\label{mtx_exact}
	\langle \vect A\,\bar{\vect u}, \bar{\vect v} \rangle &= 
		\int^5_{\Gamma_{h/2}^1} \big( E_{s,\,\Gamma_{h}^2}(\vect u) : E_{s,\,\Gamma_{h}^2}(\vect v) + \vect u\cdot\vect v + \tau\,(\vect u\cdot\vect n_{\Gamma_{h}^2})\,(\vect v\cdot\vect n_{\Gamma_{h}^2}) \big) \diff{s} \\
	&
		+ \rho_u \int^5_{\Omega_h^{\Gamma}} \frac{\partial \vect u}{\partial\vect n_{\Gamma_{h}^2}}\cdot\frac{\partial \vect v}{\partial\vect n_{\Gamma_{h}^2}} \diff{\vect x}, \quad \vect A \in \mathbb R^{n_{\vect A} \times n_{\vect A}},\\
	\langle \vect B\,\bar{\vect u}, \bar{\vect q} \rangle &= 
		-\int^5_{\Gamma_{h/2}^1} q\,\Div_{\Gamma_{h}^2} \vect u \diff{s}, \quad \vect B \in \mathbb R^{n_{\vect S} \times n_{\vect A}},\\
	\langle \vect M_0\,\bar{\vect p}, \bar{\vect q} \rangle &=
		\int^5_{\Gamma_{h/2}^1} p\,q \diff{s}, \quad \vect M_0 \in \mathbb R^{n_{\vect S} \times n_{\vect S}},\\
	\langle \vect C_n\,\bar{\vect p}, \bar{\vect q} \rangle &=
		\rho_p \int^5_{\Omega^{\Gamma}_h} \frac{\partial p}{\partial\vect n_{\Gamma_{h}^2}} \frac{\partial q}{\partial\vect n_{\Gamma_{h}^2}} \diff{\vect x}, \quad \vect C_n \in \mathbb R^{n_{\vect S} \times n_{\vect S}},\\
	\langle \vect C_{\text{full}}\,\bar{\vect p}, \bar{\vect q} \rangle &=
		\rho_p \int^5_{\Omega^{\Gamma}_h} \nabla p \cdot \nabla q \diff{\vect x}, \quad \vect C_{\text{full}} \in \mathbb R^{n_{\vect S} \times n_{\vect S}},		 
\end{split}\end{align}
Comments:
\begin{itemize}
	\item $\int^5$ denotes a composite quadrature rule that is exact for~$\bar P_h^5$,
	\item $E_{s,\,\Gamma_{h}^2}$ and~$\Div_{\Gamma_{h}^2}$ are defined as their continuous analogues with~$\vect n_{\Gamma}$ in~$\vect P$ replaced with~$\vect n_{\Gamma_{h}^2}$,
	\item It is always the case that integrands use~$\vect n_{\Gamma_{h}^2} \ne \vect n_{\Gamma_{h/2}^1}$, and the actual domain of integration is~$\Gamma_{h/2}^1 \ne \Gamma_{h}^2$,
	\item $\vect n_{\Gamma_{h}^2}$ is defined in~\eqref{gammah:n} and it is not a polynomial even locally, thus quadrature rules are never exact (although for $\vect P_2$\,--\,$P_1$ shape functions alone these quadratures are exact),
	\item In our examples we know~$\phi$ and~$\nabla \phi$ exactly, and thus it is super easy to feed the \textbf{exact} normal~$\vect n_{\Gamma} \ne \vect n_{\Gamma_{h}^2}$ to quadratures. Shall we do this?
\end{itemize}

\bibliographystyle{plain}
\bibliography{bibl}

\end{document}